
%%%%%%%%%%%%%%%%%%%
%%%%%%%%%%%%%%%%%%%
%%%%%
%%%%% Etapa 01
%%%%%
%%%%%%%%%%%%%%%%%%%
%%%%%%%%%%%%%%%%%%%




%%%%%%
%%%%%%
%
\classdef[BaseGraf]{INF01202}{6}{ALGORÍTMOS E PROGRAMAÇÃO - CIC}

     \csummary{Noção de algoritmo, dado, variável, instrução e programa. Construções básicas: atribuição, leitura e escrita. Estruturas de controle: seqüência, seleção e iteração. Tipos de dados escalares: inteiros, reais, caracteres, intervalos e enumerações. Tipos estruturados básicos: vetores, matrizes registros e strings. Subprogramas: funções, procedimentos e recursão. Arquivos.}

     \Orgbibdef{Damas, Luis. Linguagem C. Rio de Janeiro: LTC, c2007. ISBN 9788521615194}
     \Orgbibdef{Nina Edelweiss e Maria Aparecida Castro Livi.. Algoritmos e Programação: com exemplos em Pascal e C. Série de Livros Didáticos Informática UFRGS. Porto Alegre: Bookman, 2014. ISBN 9788582601891}
     \Orgbibdef{Salvetti, Dirceu Douglas; Barbosa, Lisbete Madsen. Algoritmos. Sao Paulo: Makron Books, c1998. ISBN 853460715X}
     \Orgbibdef[basic]{Deitel, Harvey M.. C How to Program. Estados Unidos: Prentice-Hall, 2007. ISBN 9780132404167}
     \Orgbibdef[basic]{Goodrich, Michael T.; Tamassia, Roberto. Projeto de algoritmos :fundamentos, análise e exemplos da internet. Porto Alegre: Bookman, 2004. ISBN 8536303034}
     \Orgbibdef[basic]{Harbison, Samuel P., III. Steele, Guy L., Jr.. C: manual de referência. Rio de Janeiro: Ciência Moderna, 2002. ISBN 8573932295}
     \Orgbibdef[basic]{Kernighan, Brian W.; Ritchie, Dennis M.. The C programming language. Englewood Cliffs: Prentice Hall, c1988. ISBN 0131103628}
     \Orgbibdef[basic]{Orth, Afonso Inacio. Algoritmos e programação :com resumo das linguagens pascal e C. Porto Alegre: AIO, c2001}
     \Orgbibdef[basic]{Senne, Edson Luiz França. Curso de programação em C. São Paulo: Visual Books, 2009. ISBN 9788575022450}
     \Orgbibdef[basic]{Ziviani, N.. Projeto de Algoritmos Com Implementações em Pascal e C. THOMSON PIONEIRA, 2004. ISBN 8522103909}


%%%%%%
%%%%%%
%
\classdef[BaseMatematica]{MAT01353}{6}{CÁLCULO E GEOMETRIA ANALÍTICA I - A}

     \csummary{Estudo da reta e de curvas planas. Cálculo diferencial de uma variável real. Cálculo integral das funções de uma variável real.}

      \Orgbibdef{Howard Anton; Irl C. Bivens; Stephen L. Davis. Cálculo - Volume 1. Porto Alegre: Bookman, 2014. ISBN 9788582602256}
      \Orgbibdef[basic]{Rogawski, Jon;. Cálculo - Vol. 1. Porto Alegre: Bookman, 2009. ISBN 9788577802708}
      \Orgbibdef[compl]{Avila, Geraldo Severo de Souza. Cálculo. Rio de Janeiro: LTC, 2003 - 2006. ISBN 8521613709 (v. 1); 8521613997 (v. 2); 8521615019 (v. 3)}
      \Orgbibdef[compl]{Hughes-Hallet, Deborah. Cálculo. Rio de Janeiro: LTC, c1997. ISBN 8521611021}
      \Orgbibdef[compl]{Larson, Roland E.; Hostetler, Robert P.; Edwards, Bruce H.. Cálculo com geometria analítica. Rio de Janeiro: Livros Técnicos e Científicos, c1998. ISBN 8521611080}
      \Orgbibdef[compl]{Shenk, al. Calculo com geometria analitica. Rio de Janeiro: Campus, 1984. ISBN 8570011229; 8570011237; 8570011245; 8570012535}
      \Orgbibdef[compl]{Simmons, George F.. Cálculo com geometria analítica. São Paulo: Mcgraw-Hill, c1987. ISBN 0074504118}
      \Orgbibdef[compl]{Stewart, James. Cálculo. São Paulo: Thomson Learning, 2006, c2005. ISBN 8522104794; 9788522104796}
      \Orgbibdef[compl]{Strang, Gilbert. Calculus. Cambridge: Wellesley-Cambridge Press, 1991. ISBN 0961408820}


%%%%%%
%%%%%%
%
\classdef[BaseMatematica]{FIS01181}{6}{FÍSICA I-C}

     \csummary{Medidas físicas. Cinemática, estática e dinâmica do ponto e do corpo rígido. Gravitação.}

      \Orgbibdef{Halliday, David; Resnick, Robert; Walker, Jearl. Fundamentos de Fí­sica. Rio de Janeiro: GEN-LTC, 2016. ISBN 978-8521630371 (v1); 978-8521630364 (v2)}
      \Orgbibdef{Tipler, Paul A.; Mosca, Gene. Física :para cientistas e engenheiros. Rio de Janeiro: LTC, 2009. ISBN 9788521617105 (v.1); 9788521617112 (v.2)}
      \Orgbibdef{Young, Hugh D; Freedman, Roger A. Fí­sica, Sears. São Paulo: Pearson Education do Brasil, 2016. ISBN 978-8543005683 (v1); 978-8543005737 (v2)}
      \Orgbibdef[basic]{Alaor Chaves e J. F. Sampaio. Física Básica - Mecânica. Rio de Janeiro: Livros Técnicos e Científicos, 2007. ISBN 978-85-216-1549-1}
      \Orgbibdef[basic]{Nussenzveig, Hersh Moyses. Curso de física básica. Sao Paulo: Ed. Edgar Blucher, c2002. ISBN 8521202989 (v.1); 8521202997 (v.2)}
      \Orgbibdef[basic]{Resnick, Robert; Halliday, David; Krane, Kenneth S.. Física. Rio de Janeiro: LTC Editora, c2003. ISBN 8521613520 (V.1); 9788521613527 (V.1); 8521613687 (V.2); 9788521613688 (V.2); 9788521614067}
      \Orgbibdef[basic]{Serway, Raymond A.; Jewett, Jr., John W.. Princípios de física :. São Paulo: Pioneira Thomson Learning, [2004]. ISBN 8522103828 (v.1); 9788522103829 (v.1); 8522104131 (v.2); 9788522104130 (v.2);}
      \Orgbibdef[compl]{Wagner Corradi, Rodrigo Tárcia e colaboradores. Fundamentos de Física I. Belo Horizonte: UFMG, 2010}


%%%%%%
%%%%%%
%
\classdef[BaseGraf]{ARQ03317}{2}{GEOMETRIA DESCRITIVA II-A}

     \csummary{Fundamentos da expressão gráfica. Métodos atuais de representação. Representação da forma e posição. Deslocamentos. Vistas auxiliares. Seções.}

      \Orgbibdef{Borges, Gladys Cabral de Mello; Barreto, Deli Garcia Olle; Martins, Enio Zago. Noções de geometria descritiva :teoria e exercícios. Porto Alegre: Sagra-Dc Luzzatto, 1998. ISBN 8572370072}

%%%%%%
%%%%%%
%
\classdef[Transv.integ]{CCA99001}{4}{INTRODUÇÃO À ENGENHARIA DE CONTROLE E AUTOMAÇÃO}

     \csummary{Descrição da área de Engenharia de Controle e Automação e do Perfil dos profissionais atuantes na área. Metodologia Científica aplicada à Engenharia de Controle e Automação. Organização do curso e compreensão das atividades de ensino, pesquisa e extensão desenvolvidos nos Departamentos e Laboratórios ligados ao curso. Figura do Engenheiro Cidadão na Sociedade moderna, questões étnico-sociais históricas, acessibilidade e segurança. Inserção da abordagem científica e soluções de engenharia na resolução de problemas, de forma segura, no contexto étnico-social atual.}


      \Orgbibdef{  BAZZO, Walter Antônio, PEREIRA, Luiz Teixeira do Vale. INTRODUÇÃO À ENGENHARIA: CONCEITOS, FERRAMENTAS E COMPORTAMENTOS. Florianópolis: Editora da UFSC, 2008. ISBN 9788532803566}
      \Orgbibdef{	SEVERINO, Antônio Joaquim. Metodologia do trabalho científico. Editora Cortez Morães, ISBN 978-85-249-1311-2}
      \Orgbibdef[basic]{CRESWELL, John W - (ISBN: 978-85-363-0892-0). Projeto de pesquisa: método qualitativo, quantitativo e misto. Editora ARTMED, ISBN 978-85-363-0892-0)}
      \Orgbibdef[compl]{Moraes, Cícero Couto de; Castrucci, Plínio de Lauro. Engenharia de Automação industrial.. Rio de Janeiro: Livros Técnicos e Científicos - LTC, 2001}


%%%%%%
%%%%%%
%
\classdef[BaseMatematica]{QUI01009}{4}{QUIMICA FUNDAMENTAL A}

     \csummary{Estequiometria. Soluções. Cinética Química. Equilíbrio químico e iônico. Colóides. Estrutura atômica. Propriedades periódicas. Ligação química: covalente, iônica e metálica.}

      \Orgbibdef{Brown, L.S.; Holme, T.A.. Química geral aplicada à engenharia. São Paulo: Cengage Leaning, 2009. ISBN 9788522106882}
      \Orgbibdef{Brown, T.L.; Lemay, H. E.; Bursten, B. E.; Murphy, C. J.; Woodward, P. M.; Stoltzfus, M. W.. Química : a ciência central. São Paulo: Pearson Prentice Hall, 2016. ISBN 9788543005652}
      \Orgbibdef{Kotz, J.C.; Treichel, Junior P.; Townsend J.R.; Treichel, D.A.. Química geral e reações químicas. São paulo: Cengage Learning, 2010. ISBN 9788522106912 (v. 1) / 9788522107544 (v. 2)}

      \Orgbibdef[basic]{Atkins, P.; Jones L.. Princípios de Química; questionando a vida moderna e o meio ambiente. Porto Alegre: Bookman, 2012. ISBN 9781429219556}
      \Orgbibdef[basic]{Chang R.. Chemistry. New York: Mc Graw Hill, 2010. ISBN 9780077274313}
      \Orgbibdef[basic]{Chang R.; Goldsby K.A.. Química. Porto Alegre: McGraw-Hill, 2013. ISBN 8580552559}
      \Orgbibdef[basic]{Russell, J. B.. Química geral. São Paulo: Pearson Makron Books, 1994. ISBN 8534601925 (v.1) / 8534601518 (v.2)}
      \Orgbibdef[basic]{Tro, N. J.. Química - Uma abordagem molecular. São Paulo: LTC-Livros Técnicos e Científicos Editora Ltda., 2017. ISBN 9788521633372 (v. 1) / 9788521633396 (v. 2)}
      \Orgbibdef[compl]{Brady, J. E.; Russell, J. W.; Holum, J. R.. Química a matéria e suas transformações. Rio de Janeiro: LTC Livros Técnicos e Científicos Editora S.A., 2009. ISBN 9788521617204 (V.1)/9788521617211 (V.2)}
      \Orgbibdef[compl]{Brady, J.E.; Humiston, G.E.. Química geral. Rio Janeiro: Livros Técnicos e Científicos, 1990. ISBN 9788521604488 (v.1)/9788521604495 (v.2)}
      \Orgbibdef[compl]{Ebbing, D.D.. Química Geral. Rio de Janeiro: LTC Livros Técnicos e Científicos, 1988. ISBN 8521611153 (V.1)/ 8521611277 (V.2)}
      \Orgbibdef[compl]{Mahan, B.M.; Myers, R.J.. Química: um curso universitário. São Paulo: Edgard Blücher, 1995. ISBN 9788521200369}
      \Orgbibdef[compl]{Masterton, W.L.; Slowinski, E.J.; Stanitski, C.L.. Princípios de química. Rio de Janeiro: Guanabara Koogan, 1990. ISBN 8527701561}




%%%%%%%%%%%%%%%%%%%
%%%%%%%%%%%%%%%%%%%
%%%%%
%%%%% Etapa 02
%%%%%
%%%%%%%%%%%%%%%%%%%
%%%%%%%%%%%%%%%%%%%





%%%%%%
%%%%%%
%
\classdef[BaseMatematica]{MAT01355}{4}{ÁLGEBRA LINEAR I - A}

     \csummary{Sistema de equações lineares. Matrizes. Fatoração LU. Vetores. Espaços vetoriais. Ortogonalidade. Valores próprios. Aplicações.}

    	\Orgbibdef{David C. Lay. Álgebra Linear e suas aplicações. Rio de Janeiro: LTC, 2018. ISBN 9788521634959}

        \Orgbibdef[basic]{Anton, Howard; Rorres, Chris; Doering, Claus Ivo. Álgebra linear :com aplicações. Porto Alegre: Bookman, 2001-2002. ISBN 8573078472; 0471170526 (broch.); 9798573078472}
    	\Orgbibdef[basic]{Gilbert Strang. Introdução à Álgebra Linear. Rio de Janeiro: LTC, 2013. ISBN 9788521623571}
    	\Orgbibdef[basic]{W. Keith Nicholson. Álgebra Linear. São Paulo: Mcgraw-Hill do Brasil, 2006. ISBN 9788586804922}

    	\Orgbibdef[compl]{Boldrini, Jose Luiz; Costa, Sueli I. Rodrigues; Figueiredo, Vera Lucia; Wetzler, Henry G.. Álgebra linear. São Paulo: Harbra, c1986. ISBN 8529402022; 9788529402024}
    	\Orgbibdef[compl]{Lima, Elon Lages. Álgebra linear. Rio de Janeiro: Impa/CNPq, 2006, c2004. ISBN 978-85-244-0089-6}
    	\Orgbibdef[compl]{Lipschutz, Seymour. Algebra linear :teoria e problemas. Sao Paulo: Makron Books do Brasil, c1994. ISBN 8534601976; 9788534601979}


%%%%%%
%%%%%%
%
\classdef[BaseMatematica]{MAT01354}{6}{CÁLCULO E GEOMETRIA ANALÍTICA II - A}

     \csummary{Geometria analítica espacial. Derivadas parciais. Integrais múltiplas. Séries.}

    	\Orgbibdef{Anton, Howard; Bivens, Irl; Davis, Stephen. Cálculo 10ª Edição. Porto Alegre: Bookman, 2014. ISBN 9788582602454 (v.2)}

    	\Orgbibdef[basic]{Avila, Geraldo Severo de Souza. Cálculo. Rio de Janeiro: LTC, 2003 - 2006. ISBN 8521613997 (v. 2); 8521615019 (v. 3)}
    	\Orgbibdef[basic]{Rogawski, Jon; Adams, Colin. Cálculo, 3ª edição. Porto Alegre, RS: Bookman, 2018. ISBN 9788582604571 (v.2)}
    	\Orgbibdef[basic]{Simmons, George F.. Cálculo com geometria analítica. São Paulo: Mcgraw-Hill, c1987. ISBN 0074504118}
    	\Orgbibdef[basic]{Stewart, James. Cálculo, 4ª edição. São Paulo: Cengage Learning, 2017. ISBN 9788522125845}

    	\Orgbibdef[compl]{Anton, Howard; Bivens, Irl; Davis, Stephen. Cálculo 8ª Edição. Porto Alegre: Bookman, 2007. ISBN 9788560031801 (v.2)}
    	\Orgbibdef[compl]{Rogawski, Jon. Cálculo. Porto Alegre, RS: Bookman, 2009. ISBN 9788577802715 (v.2)}

%%%%%%
%%%%%%
%
\classdef[BaseGraf]{ARQ03319}{4}{DESENHO TÉCNICO II-A}

     \csummary{Extensão do processo de representação em vistas ortogonais. Vistas auxiliares primárias e secundárias. Cortes e secções. Dimensionamento dos desenhos. Desenho convencional. Aplicação da normalização.}

      \Orgbibdef{Associação Brasileira de Normas Técnicas - ABNT. Coletânea de normas de desenho técnico. São Paulo, 1990}
      \Orgbibdef{GIESECKE, Frederick; MITCHELL, Alva; SPENCER, Henry C.; HILL, Ivan Leroy; DYGDON, John Thomas e NOVAK, James E.. Tecnical Drawing. Upper Saddle River, N.J.: Pearson Prentice Hall, 2009. ISBN 9780135135273}

      \Orgbibdef[basic]{Associação Brasileira de Normas Técnicas - ABNT. NBR 10.126 Cotagem em desenho Técnico. Rio de Janeiro, 1987}
      \Orgbibdef[basic]{Associação Brasileira de Normas Técnicas - ABNT. NBR 10647 - Desenho Técnico. Rio de Janeiro, 1989}
      \Orgbibdef[basic]{Associação Brasileira de Normas Técnicas - ABNT. NBR 12.298 - Representação de área de corte por meio de hachuras em desenho técnico. Rio de Janeiro, 1995}
      \Orgbibdef[basic]{Associação Brasileira de Normas Técnicas - ABNT. NBR 16.861 - Desenho Técnico - Requisitos para representação de linhas e escrita. Rio de Janeiro, 2020}
      \Orgbibdef[basic]{Associação Brasileira de Normas Técnicas - ABNT. NBR 6.492 - Representação de Projetos de Arquitetura. Rio de Janeiro, 1994}
      \Orgbibdef[basic]{Cunha, Luis Veiga da. Desenho técnico. Lisboa: Fundação Calouste Gulbenkian, 2004. ISBN 9723110660}
      \Orgbibdef[basic]{French, Thomas E. Desenho técnico. Rio de Janeiro: Globo, 1969}

      \Orgbibdef[compl]{Associação Brasileira de Normas Técnicas - ABNT. NBR 13.273 - Desenho Técnico - Referência a itens. Rio de Janeiro, 1999}
      \Orgbibdef[compl]{BACHMANN, Albert; FORBERG, Richard. Desenho técnico.. Porto Alegre: Globo, 1969}
      \Orgbibdef[compl]{GIESECKE, Frederick E; SPENCER, Alva H. C.; DYGDON, John T.; NOVAK, James; LOCKHART, Shawna. Comunicação Gráfica Moderna. Porto Alegre: Grupo A, 2002. ISBN 9798573078441}

%%%%%%
%%%%%%
%
\classdef[BaseMatematica]{FIS01182}{6}{FÍSICA GERAL - ELETROMAGNETISMO}

     \csummary{Eletrostática. Eletrodinâmica. Magnetismo. Eletromagnetismo.}

      \Orgbibdef{Chabay, Ruth W. Física básica : matéria e interações, v. 2. RJ: LTC, 2018. ISBN 9788521635031}
      \Orgbibdef{Halliday, David; Resnick, Robert; Walker, Jearl. Fundamentos de Física. Rio de Janeiro: Livros Técnicos e Científicos, 2009. ISBN 97885216166078 (V.3)}
      \Orgbibdef{Serway, Raymond A.; Jewett, Jr., John W.. Princípios de física :. São Paulo: Pioneira Thomson Learning, c2004-2005. ISBN 8522103828 (v.1); 9788522103829 (v.1); 8522104131 (v.2); 9788522104130 (v.2); 852210414X (v.3); 9788522104147 (v.3); 8522104379 (v.4); 9788522104376 (v.4)}

      \Orgbibdef[compl]{Tipler, Paul A.. Física : para cientistas e engenheiros. Rio de Janeiro: Livros Técnicos e Científicos, 2000. ISBN 852161215X}


%%%%%%
%%%%%%
%
\classdef[BaseMec]{ENG03041}{4}{MECÂNICA APLICADA I}

     \csummary{Estática de pontos materiais. Sistemas equivalentes de forças. Equilíbrio de corpos rígidos. Forças distribuídas, centróides e baricentros. Treliças. Estruturas. Esforços internos em vigas. Atrito. Momentos de inércia de área e de volume.}

      \Orgbibdef[basic]{			Ferdinand P. Beer, E. Russell Johnston, David F. Mazurek. Mecânica vetorial para engenheiros: Estática. Rio de Janeiro: AMGH, 2019. ISBN 9788580550467}
      \Orgbibdef[basic]{Hibbeler, Russell Charles. Estática - Mecânica para Engenharia. São Paulo: Pearson, 2011. ISBN 978-85-7605-815-1}
      \Orgbibdef[basic]{M. W. Plesha; G. L. Gray; F. Costanzo. Mecânica para Engenharia - Estática. Porto Alegre: Bookman, 2014. ISBN 978-85-65837-01-9}
      \Orgbibdef[basic]{Shames, Irving H.. Mecânica para engenharia. São Paulo: Prentice Hall, c2002-2003. ISBN 8587918133 (V.1); 8587918214 (V.2)}

%%%%%%
%%%%%%
%
\classdef[BaseGraf]{INF01057}{4}{PROGRAMAÇÃO ORIENTADA A OBJETO}

     \csummary{Abstração e encapsulamento de dados. Conceitos de orientação a objeto: classes, instância, herança, polimorfismo. Ferramentas de desenvolvimento e modelagem, usando orientação a objetos. Aplicação dos conceitos e ferramentas a partir da utilização de uma linguagem de programação específica.}

      \Orgbibdef[basic]{  Santos, Rafael. Introdução à programação orientada a objetos usando Java. Rio de Janeiro: Campus, 2003. ISBN 853521206X}

      \Orgbibdef[compl]{Arnold, Ken; Gosling, James A.; Holmes, David. A linguagem de programação Java. Porto Alegre: Bookman, 2007. ISBN 9788560031641}


