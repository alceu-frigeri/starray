%%%%%%%%%%%%%%%%%%%
%%%%%%%%%%%%%%%%%%%
%%%%%
%%%%% Etapa 01
%%%%%
%%%%%%%%%%%%%%%%%%%
%%%%%%%%%%%%%%%%%%%

%%%%%%
%%%%%%
%
\classdef[BaseGraf]{INF01202}{6}{ALGORÍTMOS E PROGRAMAÇÃO - CIC}

     \csummary{Noção de algoritmo, dado, variável, instrução e programa. Construções básicas: atribuição, leitura e escrita. Estruturas de controle: seqüência, seleção e iteração. Tipos de dados escalares: inteiros, reais, caracteres, intervalos e enumerações. Tipos estruturados básicos: vetores, matrizes registros e strings. Subprogramas: funções, procedimentos e recursão. Arquivos.}

     \bibdef{damas2007}
     \bibdef{edelweiss2014}
     \bibdef{salvetti1998}
     \bibdef[basic]{deitel2007}
     \bibdef[basic]{goodrich2004}
     \bibdef[basic]{harbison2002}
     \bibdef[basic]{kernighan1988}
     \bibdef[basic]{orth2001}
     \bibdef[basic]{senne2009}
     \bibdef[basic]{ziviani2004}

%%%%%%
%%%%%%
%
\classdef[BaseMatematica]{MAT01353}{6}{CÁLCULO E GEOMETRIA ANALÍTICA I - A}

     \csummary{Estudo da reta e de curvas planas. Cálculo diferencial de uma variável real. Cálculo integral das funções de uma variável real.}

     \bibdef{anton2014}
     \bibdef[basic]{rogawski2009}
     \bibdef[compl]{avila2003}
     \bibdef[compl]{hugheshallet1997}
     \bibdef[compl]{larson1998}
     \bibdef[compl]{shenk1984}
     \bibdef[compl]{simmons1987}
     \bibdef[compl]{stewart2006}
     \bibdef[compl]{strang1991}

%%%%%%
%%%%%%
%
\classdef[BaseMatematica]{FIS01181}{6}{FÍSICA I-C}

     \csummary{Medidas físicas. Cinemática, estática e dinâmica do ponto e do corpo rígido. Gravitação.}

     \bibdef{halliday2016}
     \bibdef{tipler2009}
     \bibdef{young2016}
     \bibdef[basic]{chaves2007}
     \bibdef[basic]{nussenzveig2002}
     \bibdef[basic]{resnick2003}
     \bibdef[basic]{serway2004}
     \bibdef[compl]{corradi2010}

%%%%%%
%%%%%%
%
\classdef[BaseGraf]{ARQ03317}{2}{GEOMETRIA DESCRITIVA II-A}

     \csummary{Fundamentos da expressão gráfica. Métodos atuais de representação. Representação da forma e posição. Deslocamentos. Vistas auxiliares. Seções.}

     \bibdef{borges1998}

%%%%%%
%%%%%%
%
\classdef[Transv.integ]{CCA99001}{4}{INTRODUÇÃO À ENGENHARIA DE CONTROLE E AUTOMAÇÃO}

     \csummary{Descrição da área de Engenharia de Controle e Automação e do Perfil dos profissionais atuantes na área. Metodologia Científica aplicada à Engenharia de Controle e Automação. Organização do curso e compreensão das atividades de ensino, pesquisa e extensão desenvolvidos nos Departamentos e Laboratórios ligados ao curso. Figura do Engenheiro Cidadão na Sociedade moderna, questões étnico-sociais históricas, acessibilidade e segurança. Inserção da abordagem científica e soluções de engenharia na resolução de problemas, de forma segura, no contexto étnico-social atual.}

     \bibdef{bazzo2008}
     \bibdef{severino}
     \bibdef[basic]{creswell}
     \bibdef[compl]{moraes2001}

%%%%%%
%%%%%%
%
\classdef[BaseMatematica]{QUI01009}{4}{QUIMICA FUNDAMENTAL A}

     \csummary{Estequiometria. Soluções. Cinética Química. Equilíbrio químico e iônico. Colóides. Estrutura atômica. Propriedades periódicas. Ligação química: covalente, iônica e metálica.}

     \bibdef{brown2009}
     \bibdef{brown2016}
     \bibdef{kotz2010}
     \bibdef[basic]{atkins2012}
     \bibdef[basic]{chang2010}
     \bibdef[basic]{chang2013}
     \bibdef[basic]{russell1994}
     \bibdef[basic]{tro2017}
     \bibdef[compl]{brady2009}
     \bibdef[compl]{brady1990}
     \bibdef[compl]{ebbing1988}
     \bibdef[compl]{mahan1995}
     \bibdef[compl]{masterton1990}

%%%%%%%%%%%%%%%%%%%
%%%%%%%%%%%%%%%%%%%
%%%%%
%%%%% Etapa 02
%%%%%
%%%%%%%%%%%%%%%%%%%
%%%%%%%%%%%%%%%%%%%

%%%%%%
%%%%%%
%
\classdef[BaseMatematica]{MAT01355}{4}{ÁLGEBRA LINEAR I - A}

     \csummary{Sistema de equações lineares. Matrizes. Fatoração LU. Vetores. Espaços vetoriais. Ortogonalidade. Valores próprios. Aplicações.}

     \bibdef{lay2018}
     \bibdef[basic]{anton2001}
     \bibdef[basic]{strang2013}
     \bibdef[basic]{nicholson2006}
     \bibdef[compl]{boldrini1986}
     \bibdef[compl]{lima2006}
     \bibdef[compl]{lipschutz1994}

%%%%%%
%%%%%%
%
\classdef[BaseMatematica]{MAT01354}{6}{CÁLCULO E GEOMETRIA ANALÍTICA II - A}

     \csummary{Geometria analítica espacial. Derivadas parciais. Integrais múltiplas. Séries.}

     \bibdef{anton2014b}
     \bibdef[basic]{avila2003b}
     \bibdef[basic]{rogawski2018}
     \bibdef[basic]{simmons1987b}
     \bibdef[basic]{stewart2017}
     \bibdef[compl]{anton2007}
     \bibdef[compl]{rogawski2009b}

%%%%%%
%%%%%%
%
\classdef[BaseGraf]{ARQ03319}{4}{DESENHO TÉCNICO II-A}

     \csummary{Extensão do processo de representação em vistas ortogonais. Vistas auxiliares primárias e secundárias. Cortes e secções. Dimensionamento dos desenhos. Desenho convencional. Aplicação da normalização.}

     \bibdef{abnt1990}
     \bibdef{giesecke2009}
     \bibdef[basic]{abnt1987}
     \bibdef[basic]{abnt1989}
     \bibdef[basic]{abnt1995}
     \bibdef[basic]{abnt2020}
     \bibdef[basic]{abnt1994}
     \bibdef[basic]{cunha2004}
     \bibdef[basic]{french1969}
     \bibdef[compl]{abnt1999}
     \bibdef[compl]{bachmann1969}
     \bibdef[compl]{giesecke2002}

%%%%%%
%%%%%%
%
\classdef[BaseMatematica]{FIS01182}{6}{FÍSICA GERAL - ELETROMAGNETISMO}

     \csummary{Eletrostática. Eletrodinâmica. Magnetismo. Eletromagnetismo.}

     \bibdef{chabay2018}
     \bibdef{halliday2009}
     \bibdef{serway2004b}
     \bibdef[compl]{tipler2000}

%%%%%%
%%%%%%
%
\classdef[BaseMec]{ENG03041}{4}{MECÂNICA APLICADA I}

     \csummary{Estática de pontos materiais. Sistemas equivalentes de forças. Equilíbrio de corpos rígidos. Forças distribuídas, centróides e baricentros. Treliças. Estruturas. Esforços internos em vigas. Atrito. Momentos de inércia de área e de volume.}

     \bibdef[basic]{beer2019}
     \bibdef[basic]{hibbeler2011}
     \bibdef[basic]{plesha2014}
     \bibdef[basic]{shames2002}

%%%%%%
%%%%%%
%
\classdef[BaseGraf]{INF01057}{4}{PROGRAMAÇÃO ORIENTADA A OBJETO}

     \csummary{Abstração e encapsulamento de dados. Conceitos de orientação a objeto: classes, instância, herança, polimorfismo. Ferramentas de desenvolvimento e modelagem, usando orientação a objetos. Aplicação dos conceitos e ferramentas a partir da utilização de uma linguagem de programação específica.}

     \bibdef[basic]{santos2003}
     \bibdef[compl]{arnold2007} 