%%%%%%%%%%%%%%%%%
%%%%%%%%%%%%%%%%%
%%%
%%%  Etapa 03
%%%
%%%%%%%%%%%%%%%%%
%%%%%%%%%%%%%%%%%

%%%%%%
%%%%%%
%
\classdef[BaseEletro]{ENG10001}{4}{CIRCUITOS ELÉTRICOS I - C}

     \csummary{Análise de circuitos resistivos. Quadripolos resistivos. Análise de circuitos de primeira e segunda ordem de domínio do tempo.}

     \bibdef{alexander_fundamentos}
     \bibdef{nilsson_circuitos}
     \bibdef[basic]{irwin_analise}
     \bibdef[basic]{scott_elements}
     \bibdef[compl]{desoer_teoria}
     \bibdef[compl]{dorf_introducao}
     \bibdef[compl]{foerster_circuitos}

%%%%%%
%%%%%%
%
\classdef[BaseMatematica]{MAT01167}{6}{EQUAÇÕES DIFERENCIAIS II}

     \csummary{Equações diferenciais ordinárias e lineares. Elementos de séries de Fourier, polinômios de Legendre e funções de Bessel. Equações diferenciais lineares a derivadas parciais (problemas de contorno: equações da Física Clássica).}

     \bibdef{edwards_equacoes}
     \bibdef{boyce2015equacoes}
     \bibdef{zill2003equacoes}
     \bibdef[basic]{brietzke_notas}
     \bibdef[compl]{asmar2005partial}
     \bibdef[compl]{boyce2006equacoes}
     \bibdef[compl]{churchill1978series}
     \bibdef[compl]{figueiredo2003analise}
     \bibdef[compl]{kreyszig2006advanced}
     \bibdef[compl]{simmons1972differential}
     \bibdef[compl]{solow1998differential}
     \bibdef[compl]{spiegel1976analise}
     \bibdef[compl]{tenenbaum1963ordinary}
     \bibdef[compl]{zill2001equacoes}
     \bibdef[compl]{zill2001equacoesb}

%%%%%%
%%%%%%
%
\classdef[BaseMatematica]{FIS01183}{6}{FÍSICA III-C}

     \csummary{Temperatura. Calor. Teoria cinética dos gases. Termodinâmica. Física ondulatória: ondas mecânicas e eletro-magnéticas. Reflexão e refração.}

     \bibdef{edwards_equacoesb}
     \bibdef{boyce2015equacoesb}
     \bibdef{zill2003equacoesb}
     \bibdef[basic]{alonso1999fisica}
     \bibdef[basic]{alonso1967fisica}
     \bibdef[basic]{halliday2006fundamentos}
     \bibdef[basic]{mckelvey1979fisica}
     \bibdef[basic]{nussenzveig2002curso}
     \bibdef[basic]{sears1983fisica}
     \bibdef[basic]{tipler2009fisica}

%%%%%%
%%%%%%
%
\classdef[Base.Mat]{ENG03043}{4}{MATERIAIS PARA ENGENHARIA A}

     \csummary{Materiais e aplicações principais em engenharia. Correlação entre estrutura e propriedades dos materiais. Microestrutura e suas relações com o comportamento mecânico. Materiais metálicos: classificação e aplicações específicas, metalografia, tratamentos térmicos e termoquímicos. Influência da microestrutura no comportamento mecânico. Processamento, microestrutura e comportamento mecânico dos materiais cerâmicos, poliméricos e conjugados.}

     \bibdef{callister2020ciencia}
     \bibdef[basic]{askeland2019ciencia}
     \bibdef[compl]{shackelford2008ciencia}

%%%%%%
%%%%%%
%
\classdef[BaseMec]{ENG03042}{4}{MECÂNICA APLICADA II}

     \csummary{Cinemática do ponto material. 2ª. Lei de Newton. Energia e quantidade de movimento. Sistemas de pontos materiais. Cinemática de corpos rígidos. Princípios de conservação de energia e quantidade de movimento. Movimento de corpos rígidos em duas e três dimensões.}

     \bibdef{beer2019mecanica}
     \bibdef{hibbeler2017dinamica}
     \bibdef{meriam2016mecanica}
     \bibdef[basic]{gray2014mecanica}
     \bibdef[basic]{nelson2013engenharia}
     \bibdef[basic]{rade2017cinematica}
     \bibdef[basic]{tenenbaum2016dinamica}
     \bibdef[basic]{tongue2007dinamica}

%%%%%%
%%%%%%
%
\classdef[BaseMatematica]{MAT02219}{4}{PROBABILIDADE E ESTATÍSTICA}

     \csummary{Probabilidade: definições e axiomas. Probabilidade condicional e independência. Variáveis aleatórias e funções de distribuição. Esperança matemática. Distribuições discretas e contínuas. Distribuições conjunta e marginal. Estimação pontual. Intervalos de confiança. Testes de hipóteses. Regressão linear simples e múltipla. Planejamento de experimentos. Análise de variância. Controle estatístico de processos.}

     \bibdef{barbetta2008estatistica}
     \bibdef{devore_probabilidade}
     \bibdef{montgomery2009estatistica}
     \bibdef[basic]{costaneto2002estatistica}
     \bibdef[basic]{fonseca1996curso}
     \bibdef[basic]{magalhaes2005noc}
     \bibdef[basic]{meyer2000probabilidade}
     \bibdef[basic]{morettin2009estatistica}
     \bibdef[basic]{spiegel2004probabilidade}

%%%%%%
%%%%%%
%
\classdef[Base.Digit]{ENG10042}{4}{SISTEMAS DIGITAIS}

     \csummary{Conceitos básicos de sistemas digitais. Álgebra Booleana e portas lógicas. Sistemas combinacionais. Sistemas sequenciais. Memórias. Síntese de circuitos digitais: circuitos aritméticos, contadores, registradores e máquinas de estados. Ferramentas computacionais de projeto e simulação. Circuitos integrados Digitais. Arranjos lógicos programáveis.}

     \bibdef{brown2008fundamentals}
     \bibdef{floyd_sistemas}
     \bibdef{wakerly2005digital}
     \bibdef[basic]{mano_logic}
     \bibdef[basic]{tocci2011sistemas}
     \bibdef[basic]{tokheim_fundamentos}
     \bibdef[basic]{vahid_digital}
     \bibdef[compl]{fletcher_an}
     \bibdef[compl]{karris_digital}
     \bibdef[compl]{sandige_fundamentals}
     \bibdef[compl]{wagner_fundamentos}

%%%%%%%%%%%%%%%%%
%%%%%%%%%%%%%%%%%
%%%
%%%  Etapa 04
%%%
%%%%%%%%%%%%%%%%%
%%%%%%%%%%%%%%%%%

%%%%%%
%%%%%%
%
\classdef[BaseMatematica]{MAT01169}{6}{CÁLCULO NUMÉRICO}

     \csummary{Erros numéricos. Resolução de equações não lineares. Sistemas de equações lineares. Aproximação de funções. Derivação e integração numérica. Solução numérica de equações diferenciais ordinárias. Aplicações. Implementação computacional de métodos numéricos.}

     \bibdef{borche2008metodos}
     \bibdef{burden2003analise}
     \bibdef[basic]{bortoli2001introducao}
     \bibdef[basic]{ruggiero1996calculo}
     \bibdef[compl]{barroso1987calculo}
     \bibdef[compl]{conte1965elementos}
     \bibdef[compl]{burden2005numerical}
     \bibdef[compl]{roque2000introducao}
     \bibdef[compl]{sperandio2003calculo}

%%%%%%
%%%%%%
%
\classdef[BaseEletro]{ENG10002}{4}{CIRCUITOS ELÉTRICOS II - C}

     \csummary{Análise sinusoidal de circuitos. Análise de circuitos no domínio da frequência. Potência em circuitos AC. Redes trifásicas. Redes magneticamente acopladas. Transformadores. Quadripolos. Análise de circuitos usando transformada de Laplace.}

     \bibdef{alexander2003fundamentos}
     \bibdef{irwin2003analise}
     \bibdef{nilsson2003circuitos}
     \bibdef[basic]{bird2009circuitos}
     \bibdef[compl]{desoer1979teoria}
     \bibdef[compl]{dorf2003introducao}
     \bibdef[compl]{hayt2008analise}

%%%%%%
%%%%%%
%
\classdef[Base.Digit]{ENG10043}{2}{LABORATÓRIO DE SISTEMAS DIGITIAIS}

     \csummary{Análise e projeto de circuitos lógicos combinacionais e sequenciais. Utilização de circuitos integrados lógicos de pequena, média e grande escala de integração. Utilização de simuladores e ferramentas de apoio ao projeto de circuitos digitais. Implementação de circuitos lógicos em FPGA.}

     \bibdef{brown_fundamentals2}
     \bibdef{floyd_sistemas2}
     \bibdef{wakerly_digital}
     \bibdef[basic]{mano_logic2}
     \bibdef[basic]{tocci_sistemas}
     \bibdef[basic]{vahid_digital2}
     \bibdef[compl]{karris_digital2}

%%%%%%
%%%%%%
%
\classdef[BaseMatematica]{MAT01168}{6}{MATEMÁTICA APLICADA II}

     \csummary{Séries de Fourier. Integral de Fourier. Transformadas de Fourier e de Laplace. Análise vetorial.}

     \bibdef{anton2007calculo}
     \bibdef{hsu2011sinais}
     \bibdef[basic]{hsu1973analise}
     \bibdef[basic]{strauch_notas}
     \bibdef[basic]{kreyszig1983matematica}
     \bibdef[basic]{spiegel1972analise}
     \bibdef[basic]{spiegel_schaum}
     \bibdef[basic]{zill2003equacoes3}
     \bibdef[compl]{asmar2005partial2}
     \bibdef[compl]{oneil2003advanced}
     \bibdef[compl]{spiegel1978transformadas}
     \bibdef[compl]{strang1991calculus2}
     \bibdef[compl]{stroud2003advanced}
     \bibdef[compl]{zill2001equacoes3}

%%%%%%
%%%%%%
%
\classdef[Base.Mat]{ENG03092}{4}{MECÂNICA DOS SÓLIDOS I-A}

     \csummary{Introdução à Mecânica dos Sólidos. Solicitações internas. Tensões e deformações. Esforço axial. Torção. Flexão simples. Cisalhamento em vigas. Solicitações compostas. Análise e transformação de tensões. Análise e transformação de deformações. Critérios de falha. Noções de coeficiente de segurança.}

     \bibdef{beer2013estatica}
     \bibdef{popov1978introducao}
     \bibdef{hibbeler2010resistencia}
     \bibdef[basic]{gere_mecanica}
     \bibdef[basic]{reddy_principles}
     \bibdef[basic]{mendonca2019metodo}
     \bibdef[compl]{gordon_structures}
     \bibdef[compl]{pilkey_modern}
     \bibdef[compl]{shames_introducao}

%%%%%%
%%%%%%
%
\classdef[BaseMec]{ENG03316}{4}{MECANISMOS I}

     \csummary{Introdução a análise de mecanismos: Conceito e classificação de mecanismos. Cadeias Cinemáticas. Análise cinemática dos mecanismos. Cames. Teoria das engrenagens. Forças de inércia em máquinas. Balanceamento estático e dinâmico. Aplicações Industriais ou em Equipamentos.}

     \bibdef[basic]{norton_cinematica}
     \bibdef[basic]{uicker2011theory}
     \bibdef[compl]{mabie1987mechanisms}
     \bibdef[compl]{norton2007design}

%%%%%%
%%%%%%
%
\classdef[BaseMec]{ENG03044}{4}{MODELAGEM DE SISTEMAS MECÂNICOS}

     \csummary{Modelagem e modelos. Tipos de modelos. Modelagem em computador. Estimativas e aproximações. Modelagem sistemática de sistemas mecânicos, elétricos, fluídicos e térmicos. Analogias elétricas. Sistemas híbridos. Técnicas de representação de modelos matemáticos. Respostas transitória e permanente de sistemas dinâmicos. Análise no domínio frequência. Simulação de resposta de sistemas dinâmicos a excitações típicas.}

     \bibdef{close2001modeling}
     \bibdef{kluever2015dynamic}
     \bibdef{palm2013system}
     \bibdef[basic]{kulakowski2012dynamic}
     \bibdef[basic]{lu2014modeling}
     \bibdef[basic]{palm1999modeling}
     \bibdef[basic]{rao2008vibracoes}
     \bibdef[compl]{das2009mechatronic}
     \bibdef[compl]{franklin2013sistemas}
     \bibdef[compl]{golnaraghi_sistemas}
     \bibdef[compl]{ogata_engenharia} 