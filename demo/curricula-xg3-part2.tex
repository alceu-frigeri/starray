%%%%%%%%%%%%%%%%%
%%%%%%%%%%%%%%%%%
%%%
%%%  Etapa 05
%%%
%%%%%%%%%%%%%%%%%
%%%%%%%%%%%%%%%%%



%%%%%%
%%%%%%
%
\classdef[BaseEletro]{ENG10044}{4}{ELETRÔNICA FUNDAMENTAL I-B}

     \csummary{Amplificadores operacionais, diodos, circuitos conformadores, transistores de junção e efeito de campo: características, polarização, estabilidade térmica e resposta em frequência. Amplificadores de um ou mais estágios, realimentação e o teorema do elemento extra: princípios de análise de estabilidade e resposta em frequência.}

      \bibdef{sedra2007microeletronica}

      \bibdef[basic]{millman2010integrated}
      \bibdef[basic]{schilling_circuitos}

      \bibdef[compl]{razavi_fundamentos}
      \bibdef[compl]{boylestad_dispositivos}

%%%%%%
%%%%%%
%
\classdef[BaseEletro]{ENG10003}{2}{LABORATÓRIO DE CIRCUITOS ELÉTRICOS}

     \csummary{Instrumentos de medida e conceitos fundamentais de medição. Ferramentas Computacionais de análise, e simulação. Aplicação de análise de circuitos resistivos. Aplicação de análise de circuitos de 1ª e 2ª ordem. Resposta em frequência de circuitos. Análise fasorial. Medição de potência em Circuitos Trifásicos. Transformadores.}

      \bibdef{alexander2008circuitos}
      \bibdef{nilsson1999circuitos}

      \bibdef{irwin2003analise}

      \bibdef[compl]{desoer1979teoria}
      \bibdef[compl]{bird2009circuitos}
      \bibdef[compl]{hayt2008analise}
      \bibdef[compl]{tsividis2001lab}

%%%%%%
%%%%%%
%
\classdef[Base.Mat]{ENG03004}{4}{MECÂNICA DOS SÓLIDOS II}

     \csummary{Análise de tensões. Teorias estruturais. Análise de flexão de vigas. Métodos clássicos de análise de vigas. Métodos de solução de problemas estaticamente indeterminados. Introdução à análise limite em vigas. Princípios energéticos. Flambagem de colunas. Introdução à elasticidade.}

      \bibdef{hibbeler2017analise}

      \bibdef[basic]{beer2006resistencia}
      \bibdef[basic]{salvadori1987estructuras}
      \bibdef[basic]{steffen1982pratica}
      \bibdef[basic]{sussekind1994curso}

%%%%%%
%%%%%%
%
\classdef[Pro.Control]{ENG10017}{6}{SISTEMAS E SINAIS}

     \csummary{Técnicas de modelagem e análise de sistemas lineares e sistemas amostrados. Introdução a sistemas não lineares.}

      \bibdef{haykin2001sinais}

      \bibdef[basic]{franklin2006feedback}
      \bibdef[basic]{lathi2007sinais}
      \bibdef[basic]{oppenheim1997signals}

      \bibdef[compl]{geromel2004analise}
      \bibdef[compl]{hsu2004sinais}
      \bibdef[compl]{olivier2019linear}

%%%%%%
%%%%%%
%
\classdef[Base.FenTrans]{ENG07086}{5}{TERMODINÂMICA E TRANSFERÊNCIA DE CALOR}

     \csummary{Propriedades termodinâmicas de substâncias puras e misturas. Energia, trabalho e calor e as Leis da Termodinâmica. Termodinâmica dos sistemas abertos. Eficiência de processos térmicos. Mecanismos de transferência de calor. Condução de calor em regime estacionário e transiente.}

      \bibdef{borgnakke2009termodinamica}
      \bibdef{smith2005termodinamica}

      \bibdef[basic]{koretsky2017termodinamica}

%%%%%%
%%%%%%
%
\classdef[Pro.Robotica]{ENG10026}{4}{ROBÓTICA-A}

     \csummary{Estrutura de robô: características, acionamento, controle, manipuladores e sensores. Capacidade do robô. Aplicações do robô. Noções de cinemática e dinâmica. Programação do robô. Sistemas de programação. Sistema controlador - periféricos-robô.}

      \bibdef[basic]{craig2005robotics}
      \bibdef[basic]{fu1987robotics}

      \bibdef[compl]{asada1986robot}
      \bibdef[compl]{goebel2015ros}
      \bibdef[compl]{martinez2015ros}
      \bibdef[compl]{OKane2013a}
      \bibdef[compl]{romano2002}
      
%      \bibdef[compl]{spong2005robot}
%      \bibdef[compl]{groover_industrial}
%%TODO: those are wrong... missing some
%%TODO: MISSED ENG03386

\classdef[Pro.Robotica]{ENG03380}{4}{ROBÓTICA}

     \csummary{Configurações físicas de robôs, movimentos básicos, características técnicas, programação elementar, tipos de linguagens, efetuadores finais, controle da célula de trabalho. Aplicação, dados de projeto.}


      \bibdef[basic]{craig2013}
      \bibdef[basic]{niku2013}
      \bibdef[compl]{siciliano_robotics}
      \bibdef[compl]{pazos2002}
      \bibdef[compl]{rosario2005}
      \bibdef[compl]{spong2005robot}
      \bibdef[compl]{groover_industrial}


%%%%%%%%%%%%%%%%%
%%%%%%%%%%%%%%%%%
%%%
%%%  Etapa 06
%%%
%%%%%%%%%%%%%%%%%
%%%%%%%%%%%%%%%%%




%%%%%%
%%%%%%
%
\classdef[Pro.Maquinas]{ENG10047}{4}{FUNDAMENTOS DE MÁQUINAS ELÉTRICAS}

     \csummary{Princípios de conversão eletromecânica de energia. Dispositivos eletromagnéticos. Máquinas de corrente contínua. Máquinas de corrente alternada. Modelos de dispositivos em regime permanente. Características operacionais em regime permanente.}

      \bibdef{fitzgerald_maquinas}
      \bibdef{white_electromechanical}
      \bibdef{bim_maquinas}

      \bibdef[basic]{ivanovsmolensky1980machines}
      \bibdef[basic]{kostenko1979maquinas}
      \bibdef[basic]{krause_analysis}
      \bibdef[basic]{nasar_electric}

      \bibdef[compl]{gross_electric}
      \bibdef[compl]{hameyer_numerical}
      \bibdef[compl]{chapman_electric}

%%%%%%
%%%%%%
%
\classdef[Pro.Maquinas]{ENG10022}{4}{INSTRUMENTAÇÃO FUNDAMENTAL PARA CONTROLE E AUTOMAÇÃO}

     \csummary{Medidas em processos industriais. Precisão, erros e sua propagação. Transdutores para medição de grandezas físicas. Condicionamento de sinais e interfaceamento. Métodos indiretos de medida.}

      \bibdef{balbinot2011instrumentacao}
      \bibdef{balbinot_instrumentacao}

      \bibdef[basic]{doebelin_measurement}
      \bibdef[basic]{fraden2010sensors}
      \bibdef[basic]{pallasareny_sensors}

      \bibdef[compl]{considine_process}
      \bibdef[compl]{holman_experimental}

%%%%%%
%%%%%%
%
\classdef[BaseEletro]{ENG10045}{2}{LABORATÓRIO DE ELETRÔNICA}

     \csummary{Instrumentos de medida e conceitos fundamentais de medição. Ferramentas computacionais de análise e simulação de circuitos não-lineares: diodos, transistores de junção e efeito de campo. Resposta em frequência de circuitos ativos. Circuitos conformadores, amplificadores de um e de diversos estágios realimentados. Amplificadores operacionais.}

      \bibdef{sedra_microeletronica}
      \bibdef{silva2008circuitos}

      \bibdef[basic]{desoer2006circuit}

      \bibdef[compl]{cordell_amplifiers}

%%%%%%
%%%%%%
%
\classdef[Base.FenTrans]{ENG07069}{2}{PRINCÍPIOS DA MECÂNICA DE FLUIDOS}

     \csummary{Princípios de transferência de quantidade de movimento. Equações de conservação nas formas integral e diferencial. Estática dos fluidos. Camada limite. Equações de projeto para sistemas de transporte de fluidos.}

      \bibdef{potter2013mecanica}
      \bibdef{fox2017mecanica}
      \bibdef{welty2017fundamentos}

      \bibdef[basic]{bird2004fenomenos}

%%%%%%
%%%%%%
%
\classdef[Pro.Automacao]{ENG10023}{4}{SISTEMAS DE AUTOMAÇÃO}

     \csummary{Sistemas de automação industrial e de controle de processos. Técnicas de Modelagem e Metodologia de Desenvolvimento de Sistemas de Automação Industrial (Clássica e Orientada a objetos), Sistemas de Tempo Real (Linguagens de Programação, Sistemas Operacionais).}

      \bibdef{ward_structured}
      \bibdef{selic1994realtime}

      \bibdef[compl]{awad1996object}

%%%%%%
%%%%%%
%
\classdef[Pro.Control]{ENG10004}{4}{SISTEMAS DE CONTROLE I - B}

     \csummary{Modelagem e identificação de sistemas dinâmicos. Conceitos básicos e problemas fundamentais em sistemas de controle. Controladores PID: Teoria e ajuste. Projeto de controladores para sistemas monovariáveis via método do lugar das raízes. Aspectos não-lineares em sistemas de controle.}

      \bibdef{bazanella2005controle}

      \bibdef[basic]{astrom1995pid}
      \bibdef[basic]{franklin_feedback}
      \bibdef[basic]{ogata_controle}

      \bibdef[compl]{dorf_controle}
      \bibdef[compl]{kuo_controle} 