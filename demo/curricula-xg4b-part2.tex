%%%%%%%%%%%%%%%%%
%%%%%%%%%%%%%%%%%
%%%
%%%  Etapa 07
%%%
%%%%%%%%%%%%%%%%%
%%%%%%%%%%%%%%%%%

%%%%%%
%%%%%%
%
\classdef[Pro.Maquinas]{ENG10049}{4}{ACIONAMENTO DE MÁQUINAS ELÉTRICAS}

     \csummary{Seleção de motores elétricos. Comportamento e modelos dinâmicos de máquinas elétricas. Controle de velocidade e torque. Princípios de eletrônica potência, operação e componentes básicos de conversores estáticos. Acionamento de máquinas com conversores estáticos.}

      \bibdef{bose1997}
      \bibdef{bin2009}
      \bibdef{krause2013}
      \bibdef[basic]{hughes2006}
      \bibdef[basic]{boldea1999}
      \bibdef[basic]{pillai1989}
      \bibdef[basic]{leonhard1990}
      \bibdef[compl]{white1959}
      \bibdef[compl]{slemon1992}
      \bibdef[compl]{cathey2000}
      \bibdef[compl]{murphy1988}

%%%%%%
%%%%%%
%
\classdef[Pro.Control]{ENG10005}{2}{LABORATÓRIO DE CONTROLE}

     \csummary{Métodos experimentais para ajuste de controladores PID (Ziegler-Nichols e similares). Verificação experimental de desempenho de malhas de controle. Projeto de controladores por métodos de controle clássico. Projeto prático de controladores para: processo térmico, processo mecânico, controle de velocidade de motor elétrico, controle de posição de motor elétrico.}

      \bibdef{bazanella2005}
      \bibdef[basic]{ogata2003}
      \bibdef[compl]{astrom1995}
      \bibdef[compl]{franklin2006}

%%%%%%
%%%%%%
%
\classdef[Pro.Automacao]{ENG04475}{5}{MICROPROCESSADORES I}

     \csummary{Arquitetura de microprocessadores. Endereçamento e conjunto de instruções. Memória e adaptadores de interface de entrada e saída. Projeto lógico e elétrico de sistemas microprocessados. Sistemas supervisores. Programação e algoritmos.}

      \bibdef{cady2010}
      \bibdef{pont2002}
      \bibdef{susnea2005}
      \bibdef[basic]{balch2003}
      \bibdef[basic]{mcfarland2006}
      \bibdef[basic]{nicolosi2013}
      \bibdef[basic]{nicolosi2014}
      \bibdef[basic]{sengupta2010}
      \bibdef[basic]{silvajr2004}
      \bibdef[basic]{silvajr1990}
      \bibdef[compl]{hennessy2013a}
      \bibdef[compl]{hennessy2012}
      \bibdef[compl]{patterson2013}
      \bibdef[compl]{patterson2013b}

%%%%%%
%%%%%%
%
\classdef[Pro.Fabricacao]{ENG03021}{4}{PROCESSOS DISCRETOS DE PRODUÇÃO}

     \csummary{Introdução aos Materiais Empregados em Engenharia e Seleção de materiais. Fundição: princípios; principais tipos de moldes utilizados e processos de fabricação empregados, tais como fornos elétricos, por indução, etc. Conformação Mecânica: princípios; principais processos 
     empregados, tais como laminação, forjamento, etc., `a quente' e `a frio'. Usinagem: princípios; principais métodos empregados, tais como torno, retífica, etc. Soldagem e Técnicas Conexas: princípios; principais processos empregados, tais como eletrodo revestido, MIG/MAG, etc. Introdução 
     ao Planejamento das Operações de Manufatura, considerações econômicas e comparações de custos entre os processos descritos.}

      \bibdef{groover2012}
      \bibdef[basic]{diniz2006}
      \bibdef[basic]{ferreira1999}
      \bibdef[basic]{cetlin2005}
      \bibdef[basic]{marques2009}
      \bibdef[compl]{lefteri2009}
      \bibdef[compl]{dieter1981}

%%%%%%
%%%%%%
%
\classdef[Pro.Automacao]{ENG10048}{4}{PROTOCOLOS DE COMUNICAÇÃO}

     \csummary{Conceitos básicos de redes de computadores. Definição de sistemas abertos (modelo ISO/OSI). Nível físico, enlace de dados; algoritmos de detecção e correção de erro, redes, transporte e aplicação. Barramentos industriais para automação e instrumentação: IEEE448, Profibus, Fieldbus, CAN-BUS e outros protocolos de chão-de-fábrica.}

      \bibdef{tanenbaum2003}
      \bibdef[basic]{soares1996}

%%%%%%
%%%%%%
%
\classdef[Pro.Control]{ENG10018}{4}{SISTEMAS DE CONTROLE II}

     \csummary{Análise e projeto de sistemas de controle por métodos freqüenciais. Sensibilidade e robustez de sistemas de controle. Análise de ciclo-limite em sistemas não-lineares. Modelagem, análise e projeto de sistemas de controle por variáveis de estado.}

      \bibdef{bazanella2005b}
      \bibdef{ogata2003b}
      \bibdef[basic]{franklin2002}
      \bibdef[compl]{boldrini1986}
      \bibdef[compl]{chen1998}
      \bibdef[compl]{campestrini2006}
      \bibdef[compl]{zanchin2003}

%%%%%%
%%%%%%
%
\classdef[Pro.Maquinas]{ENG03027}{4}{SISTEMAS HIDRÁULICOS E PNEUMÁTICOS}

     \csummary{Introdução à hidráulica e pneumática industrial. descrição de componentes para circuitos de comando e controle: atuadores, válvulas, cilindros, bombas e compressores. Características e propriedades dos fluidos hidráulicos. Elementos de mecatrônica.}

      \bibdef{prudente2013}
      \bibdef{rabie2009}
      \bibdef{watton2012}
      \bibdef[basic]{capuamo2009}
      \bibdef[basic]{cundiff2011}
      \bibdef[basic]{linsingen2001}
      \bibdef[compl]{bollmann1996}
      \bibdef[compl]{bolton1996}
      \bibdef[compl]{martin1990}
      \bibdef[compl]{merritt1967}
      \bibdef[compl]{parr2007}
      \bibdef[compl]{yeaple1996}

%%%%%%%%%%%%%%%%%
%%%%%%%%%%%%%%%%%
%%%
%%%  Etapa 08
%%%
%%%%%%%%%%%%%%%%%
%%%%%%%%%%%%%%%%%

%%%%%%
%%%%%%
%
\classdef[Pro.Fabricacao]{ENG03045}{4}{ELEMENTOS DE MÁQUINAS}

     \csummary{Noções básicas sobre projeto mecânico. Fadiga dos materiais. Eixos de transmissão. Dimensionamento, seleção e aplicação de molas, 
     rolamentos, mancais de escorregamento, engrenagens, freios e embreagens, elementos flexíveis, parafusos de fixação, acoplamentos elásticos, elementos de transmissão de movimento.}

      \bibdef{juvinal2008}
      \bibdef[basic]{Norton2004c}
      \bibdef[basic]{shigley2004}
      \bibdef[compl]{hibbeler2010}

%%%%%%
%%%%%%
%
\classdef[Pro.ContProc]{ENG07042}{4}{MODELAGEM E CONTROLE DE PROCESSOS INDUSTRIAIS}

     \csummary{Introdução à modelagem matemática de processos industriais. Aplicação das leis de conservação em regime estacionário e dinâmico. 
     Equações constitutivas. Simulação estática e dinâmica de processos. Malhas de controle típicas da indústria de processos. Projeto de controladores aplicados na indústria de processos.}

      \bibdef{bequette1998}
      \bibdef{Campos2006c}
      \bibdef{seborg2003}
      \bibdef[basic]{Edgar2001b}
      \bibdef[basic]{Luyben1990b}
      \bibdef[basic]{rice1995}
      \bibdef[compl]{hangos2001}
      \bibdef[compl]{Ogunnaike1994b}

%%%%%%
%%%%%%
%
\classdef[Pro.Control]{ENG10019}{4}{SISTEMAS DE CONTROLE DIGITAIS}

     \csummary{Análise de Sistemas de Controle amostrados através da transformada Z. Digitalização de controladores analógicos. Identificação de sistemas pelo método dos mínimos quadrados. Projeto de controladores digitais para sistemas monovariáveis. Implementação de controladores digitais.}

      \bibdef{bazanella2012}
      \bibdef{astrom1997}
      \bibdef{franklin2006b}
      \bibdef[basic]{aguirre2007}
      \bibdef[basic]{kuo2002}
      \bibdef[basic]{ogata1995}
      \bibdef[compl]{bazanella2005c}
      \bibdef[compl]{houpis1991}
      \bibdef[compl]{hemerly2000}
      \bibdef[compl]{geromel2004}

%%%%%%
%%%%%%
%
\classdef[Pro.Fabricacao]{ENG03387}{4}{SISTEMAS DE FABRICAÇÃO}

     \csummary{Modos de produção e arranjo físico industrial. Modelos e métricas de produção. Análise de sistemas de produção: estações de trabalho operadas manualmente e automatizadas, análise de grupos de máquinas, linhas de montagem. Tecnologia de grupo: sistemas de codificação e 
     classificação, métodos para formação de famílias de peças e de células de manufatura. Manufatura celular. Movimentação interna de materiais e armazenamento. Introdução à fabricação CNC.}

      \bibdef{groover2010}
      \bibdef[basic]{black2000}
      \bibdef[compl]{groover2007}
      \bibdef[compl]{liker2005}
      \bibdef[compl]{lorino1990}

%%%%%%
%%%%%%
%
\classdef[Pro.Fabricacao]{ENG03386}{4}{FABRICAÇÃO AUXILIADA POR COMPUTADOR}

     \csummary{Processos de fabricação por usinagem: torneamento, furação e fresamento. Planejamento de processo e CAPP. Projeto orientado à manufatura e montagem (DFM/A). Fundamentos da usinagem CNC. Arquitetura de sistemas CNC: hardware e software. Linguagem ISSO. Programação 
     manual de centros de usinagem CNC. Programação manual de tornos CNC. CAD/CAM.}

      \bibdef{groover2010b}
      \bibdef[basic]{chang2005}
      \bibdef[basic]{sawhney2007}
      \bibdef[compl]{amorim2012a}
      \bibdef[compl]{amorim2012b}
      \bibdef[compl]{amorim2013}
      \bibdef[compl]{cornelius2003}
      \bibdef[compl]{groover2007b}

%%%%%%
%%%%%%
%
\classdef[Pro.Control]{ENG03046}{4}{CONTROLE DE SISTEMAS FLUÍDO-MECÂNICOS}

     \csummary{Modelagem dinâmica de sistemas hidráulicos, pneumáticos e híbridos. Características não lineares de sistemas hidráulicos e pneumáticos: aspectos construtivos e análise por aproximações lineares. Servoatuadores hidráulicos e pneumáticos: análise, controle e aplicações.}

      \bibdef{manring2005}
      \bibdef{Slotine1991a}
      \bibdef{watton2012b}
      \bibdef[basic]{costa2015}
      \bibdef[basic]{cundiff2013}
      \bibdef[basic]{linsingen2008}
      \bibdef[basic]{merritt1991}
      \bibdef[basic]{watton2012c}
      \bibdef[compl]{akers2006}
      \bibdef[compl]{dorf2008}
      \bibdef[compl]{franklin2013}
      \bibdef[compl]{ogata2010}
      \bibdef[compl]{parr2011}
      \bibdef[compl]{rabie2009b}
      \bibdef[compl]{watton2007}

%%%%%%
%%%%%%
%
\classdef[Pro.Maquinas]{ENG10027}{4}{ELETRÔNICA FUNDAMENTAL II - B}

     \csummary{Amplificador operacional: modelamento e características. Circuitos não-lineares com amplificadores operacionais: conformadores, comparadores, detectores de pico, amostradores, conversores tensão-frequência, amplificadores logarítmicos, mono-estáveis, estáveis. Circuitos integrados especiais e aplicações. Conceitos básicos de comportamento em frequência de amplificadores.}

      \bibdef{sedra2007}
      \bibdef[basic]{graeme1981}
      \bibdef[basic]{razavi2010}
      \bibdef[basic]{franco1998}
      \bibdef[basic]{wait1991}
      \bibdef[basic]{jung2004}
      \bibdef[basic]{wong1972}
      \bibdef[compl]{lancaster1975}
      \bibdef[compl]{ott1988} 