%%%==============================================================================
% WinEdt pragmas
% !Mode:: "TeX:EN"
% Default Compile engines:
% !TEX program = pdflatex
% !PDFTeXify ext =  --enable-etex  --restrict-write18
% !PDFLaTeX ext  =  --enable-etex  --restrict-write18
% !BIB program = biber --validate_datamodel
%%%==============================================================================
%% Copyright 2023-present by Alceu Frigeri
%%
%% This work may be distributed and/or modified under the conditions of
%%
%% * The [LaTeX Project Public License](http://www.latex-project.org/lppl.txt),
%%   version 1.3c (or later), and/or
%% * The [GNU Affero General Public License](https://www.gnu.org/licenses/agpl-3.0.html),
%%   version 3 (or later)
%%
%% This work has the LPPL maintenance status *maintained*.
%%
%% The Current Maintainer of this work is Alceu Frigeri
%%
%% This is version {1.11} {2025/10/26} 
%%
%% The list of files that compose this work can be found in the README.md file at
%% https://ctan.org/pkg/starray
%%
%%%==============================================================================
\NeedsTeXFormat{LaTeX2e}[2023/11/01]
\documentclass[10pt]{article}
\RequirePackage[verbose,a4paper,marginparwidth=27.5mm,top=2.5cm,bottom=1.5cm,hmargin={40mm,20mm},marginparsep=2.5mm,columnsep=10mm,asymmetric]{geometry}
%\RequirePackage[verbose,a4paper,marginparwidth=27.5mm,top=2.5cm,bottom=1.5cm,hmargin={45mm,25mm},marginparsep=2.5mm,columnsep=10mm,asymmetric]{geometry}
\usepackage[T1]{fontenc}
\usepackage[utf8]{inputenc}
\usepackage{lmodern}
\usepackage[infograb]{codedescribe}

\usepackage{stcurrdemo}

\usepackage{tikz}
	\usetikzlibrary{shapes.multipart}

\RequirePackage[inline]{enumitem}
\SetEnumitemKey{miditemsep}{parsep=0ex,itemsep=0.4ex}

\RequirePackage[backend=biber,style=abnt,pretty,extrayear,repeatfields]{biblatex}

\addbibresource{curricula-xg1-part1.tex}
\addbibresource{curricula-xg2-part1.tex}
\addbibresource{curricula-xg3-part1.tex}
\addbibresource{curricula-xg4b-part1.tex}
\addbibresource{curricula-xg5-part1.tex}
\addbibresource{curricula-xg6d-part1.tex}
\addbibresource{curricula-xg7-part1.tex}

\NewDocumentCommand{\defcolorpair}{m}
  {
    \colorlet{light#1}{white!85!#1}
    \colorlet{dark#1}{#1!70!black!100}
    \colorlet{darker#1}{#1!40!black!100}
  }

\defcolorpair{red}
\defcolorpair{green}
\defcolorpair{blue}
\defcolorpair{magenta}
\defcolorpair{teal}
\defcolorpair{yellow}
\defcolorpair{cyan}



\RequirePackage{array}
%\RequirePackage{nicematrix}
%\NiceMatrixOptions{cell-space-limits = 1pt}

\newcolumntype{P}[1]{>{\raggedleft\arraybackslash}p{#1}}
\newcolumntype{B}[1]{>{\raggedleft\arraybackslash\bfseries}p{#1}}
\newcolumntype{C}[1]{>{\centering\arraybackslash}p{#1}}
\newcolumntype{R}[1]{>{\raggedleft\arraybackslash}p{#1}}
\newcolumntype{L}[1]{>{\raggedright\arraybackslash}p{#1}}
\newcolumntype{J}[1]{>{\arraybackslash}p{#1}}
\RequirePackage{longtable}

%\NewDocumentEnvironment{tocless}{}
%   {
%     \setcounter{secnumdepth}{-1}
%     \addtocontents{toc}{\protect\setcounter{tocdepth}{-1}\ignorespaces}
%   }
%   {
%     \setcounter{secnumdepth}{\l__ufrgscca_secdepth_tl}
%     \addtocontents{toc}{\protect\setcounter{tocdepth}{\l__ufrgscca_tocdepth_tl}\ignorespaces}
%   }
%   
%\def\notoc#1#2{\begin{tocless}#1{#2}\end{tocless}}
%\def\notoc\relax


\usepackage{relsize}

\RequirePackage[hidelinks,hypertexnames=false]{hyperref}
\begin{document}

%%%%%%%%%%%%%%%%%
%%%%%%%%%%%%%%%%%
%%%
%%%
%%%
%% %%%%%%%%%%%%%%%
%%%%%%%%%%%%%%%%%

%\colorlet{TcolorCiclobasico}{darkgray}
%\colorlet{TcolorCiclobasicoeng}{darkgray}
%\colorlet{TcolorCicloprof}{darkgray}
%\colorlet{TcolorCicloavan}{darkgray}
%\colorlet{TcolorCiclotransv}{darkgray}
%\colorlet{TcolorOutros}{darkgray}

%\ifshowind{%
%    \colorlet{TcolorCiclobasico}{darkgray}
%    \colorlet{TcolorCiclobasicoeng}{darkgray}
%    \colorlet{TcolorCicloprof}{darkgray}
%    \colorlet{TcolorCicloavan}{darkgray}
%    \colorlet{TcolorCiclotransv}{darkgray}
%    \colorlet{TcolorOutros}{darkgray}
%}


\topicdef[TcolorCiclobasico]{BaseAcolhimento}{Ciclo Básico: Acolhimento}
\topicdef[TcolorCiclobasico]{BaseMatematica}{Ciclo Básico: Matemática, Física e Química}
\topicdef[TcolorCiclobasico]{BaseGraf}{Ciclo Básico: Expressão Gráfica e Programação}
\topicdef[TcolorCiclobasicoeng]{BaseMec}{Ciclo Básico Engenharia: Mecanismos e Dinâmica}
\topicdef[TcolorCiclobasicoeng]{BaseEletro}{Ciclo Básico Engenharia: Eletrônica e Eletrotécnica}
\topicdef[TcolorCiclobasicoeng]{Base.Admin}{Ciclo Profissionalizante: Administração e Gerenciamento}
\topicdef[TcolorCiclobasicoeng]{Base.Mat}{Ciclo Básico Engenharia: Ciência dos Materiais e Mecânica Estrutural}
\topicdef[TcolorCiclobasicoeng]{Base.FenTrans}{Ciclo Básico Engenharia: Fenômenos de Transporte}

%\topicdef[TcolorCicloprof]{Pro.IndTrans}{Ciclo Profissionalizante: Processos da Industria de Transformação}
\topicdef[TcolorCicloprof]{Pro.Fabricacao}{Ciclo Profissionalizante: Fabricação}
\topicdef[TcolorCicloprof]{Pro.Projetos}{Ciclo Profissionalizante: Projetos Integrados}
\topicdef[TcolorCicloprof]{Pro.Maquinas}{Ciclo Profissionalizante: Máquinas e Equipamentos}
\topicdef[TcolorCicloprof]{Pro.Robotica}{Ciclo Profissionalizante: Robótica}
\topicdef[TcolorCicloprof]{Pro.Automacao}{Ciclo Profissionalizante: Automação}
\topicdef[TcolorCicloprof]{Pro.Control}{Ciclo Profissionalizante: Sistemas de Controle}
%\topicdef[TcolorCicloprof]{Pro.DispMec}{Ciclo Profissionalizante: Dispositivos Mecânicos}
\topicdef[TcolorCicloprof]{Pro.ContProc}{Ciclo Profissionalizante: Controle de Processos}



\topicdef[TcolorCiclobasicoeng]{Base.Digit}{Ciclo Básico Engenharia: Sistemas Digitais}
%\topicdef[TcolorCiclobasicoeng]{Base.Instru}{Ciclo Básico Engenharia: Instrumentação}
%\topicdef[TcolorCicloprof]{Pro.ProjSis}{Ciclo Profissionalizante: Projeto de Sistemas}

%\topicdef[TcolorCicloavan]{Avan.info}{Ciclo Profissionalizante: Informática Avançada}
\topicdef[TcolorCicloavan]{Avan.topic}{Ciclo Profissionalizante: Tópicos}
%\topicdef[TcolorCicloavan]{Avan.control}{Ciclo Profissionalizante: Controle Avançado}
%\topicdef[TcolorCicloavan]{Avan.robotica}{Ciclo Profissionalizante: Robótica Avançado}
%\topicdef[TcolorCicloavan]{Avan.SisMec}{Ciclo Profissionalizante: Sistemas Mecânicos}


%\topicdef[darkred]{Social.admin}{Ciclo Agente Social e Empreendedorismo: Administração e Gerenciamento}
\topicdef[TcolorCiclotransv]{Transv.outros}{Temas Transversais: Formação Cidadã}
\topicdef[TcolorCiclotransv]{Transv.integ}{Temas Transversais: Atividades Integradoras}


\topicdef[TcolorOutros]{BaseEng}{Ciclo Básico Engenharia}
\topicdef[TcolorOutros]{Aplic}{Aplicações de Controle e Automação}
\topicdef[TcolorOutros]{Advanced}{Tópicos Avançados em Controle e Automação}
\topicdef[TcolorOutros]{Transv}{Conceitos Transversais}

\defaulttopic{Transv}

%%%%%%%%%%%%%%%%%%%
%%%%%%%%%%%%%%%%%%%
%%%%%
%%%%% Etapa 01
%%%%%
%%%%%%%%%%%%%%%%%%%
%%%%%%%%%%%%%%%%%%%

%%%%%%
%%%%%%
%
\classdef[BaseGraf]{INF01202}{6}{ALGORÍTMOS E PROGRAMAÇÃO - CIC}

     \csummary{Noção de algoritmo, dado, variável, instrução e programa. Construções básicas: atribuição, leitura e escrita. Estruturas de controle: seqüência, seleção e iteração. Tipos de dados escalares: inteiros, reais, caracteres, intervalos e enumerações. Tipos estruturados básicos: vetores, matrizes registros e strings. Subprogramas: funções, procedimentos e recursão. Arquivos.}

     \bibdef{damas2007}
     \bibdef{edelweiss2014}
     \bibdef{salvetti1998}
     \bibdef[basic]{deitel2007}
     \bibdef[basic]{goodrich2004}
     \bibdef[basic]{harbison2002}
     \bibdef[basic]{kernighan1988}
     \bibdef[basic]{orth2001}
     \bibdef[basic]{senne2009}
     \bibdef[basic]{ziviani2004}

%%%%%%
%%%%%%
%
\classdef[BaseMatematica]{MAT01353}{6}{CÁLCULO E GEOMETRIA ANALÍTICA I - A}

     \csummary{Estudo da reta e de curvas planas. Cálculo diferencial de uma variável real. Cálculo integral das funções de uma variável real.}

     \bibdef{anton2014}
     \bibdef[basic]{rogawski2009}
     \bibdef[compl]{avila2003}
     \bibdef[compl]{hugheshallet1997}
     \bibdef[compl]{larson1998}
     \bibdef[compl]{shenk1984}
     \bibdef[compl]{simmons1987}
     \bibdef[compl]{stewart2006}
     \bibdef[compl]{strang1991}

%%%%%%
%%%%%%
%
\classdef[BaseMatematica]{FIS01181}{6}{FÍSICA I-C}

     \csummary{Medidas físicas. Cinemática, estática e dinâmica do ponto e do corpo rígido. Gravitação.}

     \bibdef{halliday2016}
     \bibdef{tipler2009}
     \bibdef{young2016}
     \bibdef[basic]{chaves2007}
     \bibdef[basic]{nussenzveig2002}
     \bibdef[basic]{resnick2003}
     \bibdef[basic]{serway2004}
     \bibdef[compl]{corradi2010}

%%%%%%
%%%%%%
%
\classdef[BaseGraf]{ARQ03317}{2}{GEOMETRIA DESCRITIVA II-A}

     \csummary{Fundamentos da expressão gráfica. Métodos atuais de representação. Representação da forma e posição. Deslocamentos. Vistas auxiliares. Seções.}

     \bibdef{borges1998}

%%%%%%
%%%%%%
%
\classdef[Transv.integ]{CCA99001}{4}{INTRODUÇÃO À ENGENHARIA DE CONTROLE E AUTOMAÇÃO}

     \csummary{Descrição da área de Engenharia de Controle e Automação e do Perfil dos profissionais atuantes na área. Metodologia Científica aplicada à Engenharia de Controle e Automação. Organização do curso e compreensão das atividades de ensino, pesquisa e extensão desenvolvidos nos Departamentos e Laboratórios ligados ao curso. Figura do Engenheiro Cidadão na Sociedade moderna, questões étnico-sociais históricas, acessibilidade e segurança. Inserção da abordagem científica e soluções de engenharia na resolução de problemas, de forma segura, no contexto étnico-social atual.}

     \bibdef{bazzo2008}
     \bibdef{severino}
     \bibdef[basic]{creswell}
     \bibdef[compl]{moraes2001}

%%%%%%
%%%%%%
%
\classdef[BaseMatematica]{QUI01009}{4}{QUIMICA FUNDAMENTAL A}

     \csummary{Estequiometria. Soluções. Cinética Química. Equilíbrio químico e iônico. Colóides. Estrutura atômica. Propriedades periódicas. Ligação química: covalente, iônica e metálica.}

     \bibdef{brown2009}
     \bibdef{brown2016}
     \bibdef{kotz2010}
     \bibdef[basic]{atkins2012}
     \bibdef[basic]{chang2010}
     \bibdef[basic]{chang2013}
     \bibdef[basic]{russell1994}
     \bibdef[basic]{tro2017}
     \bibdef[compl]{brady2009}
     \bibdef[compl]{brady1990}
     \bibdef[compl]{ebbing1988}
     \bibdef[compl]{mahan1995}
     \bibdef[compl]{masterton1990}

%%%%%%%%%%%%%%%%%%%
%%%%%%%%%%%%%%%%%%%
%%%%%
%%%%% Etapa 02
%%%%%
%%%%%%%%%%%%%%%%%%%
%%%%%%%%%%%%%%%%%%%

%%%%%%
%%%%%%
%
\classdef[BaseMatematica]{MAT01355}{4}{ÁLGEBRA LINEAR I - A}

     \csummary{Sistema de equações lineares. Matrizes. Fatoração LU. Vetores. Espaços vetoriais. Ortogonalidade. Valores próprios. Aplicações.}

     \bibdef{lay2018}
     \bibdef[basic]{anton2001}
     \bibdef[basic]{strang2013}
     \bibdef[basic]{nicholson2006}
     \bibdef[compl]{boldrini1986}
     \bibdef[compl]{lima2006}
     \bibdef[compl]{lipschutz1994}

%%%%%%
%%%%%%
%
\classdef[BaseMatematica]{MAT01354}{6}{CÁLCULO E GEOMETRIA ANALÍTICA II - A}

     \csummary{Geometria analítica espacial. Derivadas parciais. Integrais múltiplas. Séries.}

     \bibdef{anton2014b}
     \bibdef[basic]{avila2003b}
     \bibdef[basic]{rogawski2018}
     \bibdef[basic]{simmons1987b}
     \bibdef[basic]{stewart2017}
     \bibdef[compl]{anton2007}
     \bibdef[compl]{rogawski2009b}

%%%%%%
%%%%%%
%
\classdef[BaseGraf]{ARQ03319}{4}{DESENHO TÉCNICO II-A}

     \csummary{Extensão do processo de representação em vistas ortogonais. Vistas auxiliares primárias e secundárias. Cortes e secções. Dimensionamento dos desenhos. Desenho convencional. Aplicação da normalização.}

     \bibdef{abnt1990}
     \bibdef{giesecke2009}
     \bibdef[basic]{abnt1987}
     \bibdef[basic]{abnt1989}
     \bibdef[basic]{abnt1995}
     \bibdef[basic]{abnt2020}
     \bibdef[basic]{abnt1994}
     \bibdef[basic]{cunha2004}
     \bibdef[basic]{french1969}
     \bibdef[compl]{abnt1999}
     \bibdef[compl]{bachmann1969}
     \bibdef[compl]{giesecke2002}

%%%%%%
%%%%%%
%
\classdef[BaseMatematica]{FIS01182}{6}{FÍSICA GERAL - ELETROMAGNETISMO}

     \csummary{Eletrostática. Eletrodinâmica. Magnetismo. Eletromagnetismo.}

     \bibdef{chabay2018}
     \bibdef{halliday2009}
     \bibdef{serway2004b}
     \bibdef[compl]{tipler2000}

%%%%%%
%%%%%%
%
\classdef[BaseMec]{ENG03041}{4}{MECÂNICA APLICADA I}

     \csummary{Estática de pontos materiais. Sistemas equivalentes de forças. Equilíbrio de corpos rígidos. Forças distribuídas, centróides e baricentros. Treliças. Estruturas. Esforços internos em vigas. Atrito. Momentos de inércia de área e de volume.}

     \bibdef[basic]{beer2019}
     \bibdef[basic]{hibbeler2011}
     \bibdef[basic]{plesha2014}
     \bibdef[basic]{shames2002}

%%%%%%
%%%%%%
%
\classdef[BaseGraf]{INF01057}{4}{PROGRAMAÇÃO ORIENTADA A OBJETO}

     \csummary{Abstração e encapsulamento de dados. Conceitos de orientação a objeto: classes, instância, herança, polimorfismo. Ferramentas de desenvolvimento e modelagem, usando orientação a objetos. Aplicação dos conceitos e ferramentas a partir da utilização de uma linguagem de programação específica.}

     \bibdef[basic]{santos2003}
     \bibdef[compl]{arnold2007} 
%%%%%%%%%%%%%%%%%
%%%%%%%%%%%%%%%%%
%%%
%%%  Etapa 03
%%%
%%%%%%%%%%%%%%%%%
%%%%%%%%%%%%%%%%%

%%%%%%
%%%%%%
%
\classdef[BaseEletro]{ENG10001}{4}{CIRCUITOS ELÉTRICOS I - C}

     \csummary{Análise de circuitos resistivos. Quadripolos resistivos. Análise de circuitos de primeira e segunda ordem de domínio do tempo.}

     \bibdef{alexander_fundamentos}
     \bibdef{nilsson_circuitos}
     \bibdef[basic]{irwin_analise}
     \bibdef[basic]{scott_elements}
     \bibdef[compl]{desoer_teoria}
     \bibdef[compl]{dorf_introducao}
     \bibdef[compl]{foerster_circuitos}

%%%%%%
%%%%%%
%
\classdef[BaseMatematica]{MAT01167}{6}{EQUAÇÕES DIFERENCIAIS II}

     \csummary{Equações diferenciais ordinárias e lineares. Elementos de séries de Fourier, polinômios de Legendre e funções de Bessel. Equações diferenciais lineares a derivadas parciais (problemas de contorno: equações da Física Clássica).}

     \bibdef{edwards_equacoes}
     \bibdef{boyce2015equacoes}
     \bibdef{zill2003equacoes}
     \bibdef[basic]{brietzke_notas}
     \bibdef[compl]{asmar2005partial}
     \bibdef[compl]{boyce2006equacoes}
     \bibdef[compl]{churchill1978series}
     \bibdef[compl]{figueiredo2003analise}
     \bibdef[compl]{kreyszig2006advanced}
     \bibdef[compl]{simmons1972differential}
     \bibdef[compl]{solow1998differential}
     \bibdef[compl]{spiegel1976analise}
     \bibdef[compl]{tenenbaum1963ordinary}
     \bibdef[compl]{zill2001equacoes}
     \bibdef[compl]{zill2001equacoesb}

%%%%%%
%%%%%%
%
\classdef[BaseMatematica]{FIS01183}{6}{FÍSICA III-C}

     \csummary{Temperatura. Calor. Teoria cinética dos gases. Termodinâmica. Física ondulatória: ondas mecânicas e eletro-magnéticas. Reflexão e refração.}

     \bibdef{edwards_equacoesb}
     \bibdef{boyce2015equacoesb}
     \bibdef{zill2003equacoesb}
     \bibdef[basic]{alonso1999fisica}
     \bibdef[basic]{alonso1967fisica}
     \bibdef[basic]{halliday2006fundamentos}
     \bibdef[basic]{mckelvey1979fisica}
     \bibdef[basic]{nussenzveig2002curso}
     \bibdef[basic]{sears1983fisica}
     \bibdef[basic]{tipler2009fisica}

%%%%%%
%%%%%%
%
\classdef[Base.Mat]{ENG03043}{4}{MATERIAIS PARA ENGENHARIA A}

     \csummary{Materiais e aplicações principais em engenharia. Correlação entre estrutura e propriedades dos materiais. Microestrutura e suas relações com o comportamento mecânico. Materiais metálicos: classificação e aplicações específicas, metalografia, tratamentos térmicos e termoquímicos. Influência da microestrutura no comportamento mecânico. Processamento, microestrutura e comportamento mecânico dos materiais cerâmicos, poliméricos e conjugados.}

     \bibdef{callister2020ciencia}
     \bibdef[basic]{askeland2019ciencia}
     \bibdef[compl]{shackelford2008ciencia}

%%%%%%
%%%%%%
%
\classdef[BaseMec]{ENG03042}{4}{MECÂNICA APLICADA II}

     \csummary{Cinemática do ponto material. 2ª. Lei de Newton. Energia e quantidade de movimento. Sistemas de pontos materiais. Cinemática de corpos rígidos. Princípios de conservação de energia e quantidade de movimento. Movimento de corpos rígidos em duas e três dimensões.}

     \bibdef{beer2019mecanica}
     \bibdef{hibbeler2017dinamica}
     \bibdef{meriam2016mecanica}
     \bibdef[basic]{gray2014mecanica}
     \bibdef[basic]{nelson2013engenharia}
     \bibdef[basic]{rade2017cinematica}
     \bibdef[basic]{tenenbaum2016dinamica}
     \bibdef[basic]{tongue2007dinamica}

%%%%%%
%%%%%%
%
\classdef[BaseMatematica]{MAT02219}{4}{PROBABILIDADE E ESTATÍSTICA}

     \csummary{Probabilidade: definições e axiomas. Probabilidade condicional e independência. Variáveis aleatórias e funções de distribuição. Esperança matemática. Distribuições discretas e contínuas. Distribuições conjunta e marginal. Estimação pontual. Intervalos de confiança. Testes de hipóteses. Regressão linear simples e múltipla. Planejamento de experimentos. Análise de variância. Controle estatístico de processos.}

     \bibdef{barbetta2008estatistica}
     \bibdef{devore_probabilidade}
     \bibdef{montgomery2009estatistica}
     \bibdef[basic]{costaneto2002estatistica}
     \bibdef[basic]{fonseca1996curso}
     \bibdef[basic]{magalhaes2005noc}
     \bibdef[basic]{meyer2000probabilidade}
     \bibdef[basic]{morettin2009estatistica}
     \bibdef[basic]{spiegel2004probabilidade}

%%%%%%
%%%%%%
%
\classdef[Base.Digit]{ENG10042}{4}{SISTEMAS DIGITAIS}

     \csummary{Conceitos básicos de sistemas digitais. Álgebra Booleana e portas lógicas. Sistemas combinacionais. Sistemas sequenciais. Memórias. Síntese de circuitos digitais: circuitos aritméticos, contadores, registradores e máquinas de estados. Ferramentas computacionais de projeto e simulação. Circuitos integrados Digitais. Arranjos lógicos programáveis.}

     \bibdef{brown2008fundamentals}
     \bibdef{floyd_sistemas}
     \bibdef{wakerly2005digital}
     \bibdef[basic]{mano_logic}
     \bibdef[basic]{tocci2011sistemas}
     \bibdef[basic]{tokheim_fundamentos}
     \bibdef[basic]{vahid_digital}
     \bibdef[compl]{fletcher_an}
     \bibdef[compl]{karris_digital}
     \bibdef[compl]{sandige_fundamentals}
     \bibdef[compl]{wagner_fundamentos}

%%%%%%%%%%%%%%%%%
%%%%%%%%%%%%%%%%%
%%%
%%%  Etapa 04
%%%
%%%%%%%%%%%%%%%%%
%%%%%%%%%%%%%%%%%

%%%%%%
%%%%%%
%
\classdef[BaseMatematica]{MAT01169}{6}{CÁLCULO NUMÉRICO}

     \csummary{Erros numéricos. Resolução de equações não lineares. Sistemas de equações lineares. Aproximação de funções. Derivação e integração numérica. Solução numérica de equações diferenciais ordinárias. Aplicações. Implementação computacional de métodos numéricos.}

     \bibdef{borche2008metodos}
     \bibdef{burden2003analise}
     \bibdef[basic]{bortoli2001introducao}
     \bibdef[basic]{ruggiero1996calculo}
     \bibdef[compl]{barroso1987calculo}
     \bibdef[compl]{conte1965elementos}
     \bibdef[compl]{burden2005numerical}
     \bibdef[compl]{roque2000introducao}
     \bibdef[compl]{sperandio2003calculo}

%%%%%%
%%%%%%
%
\classdef[BaseEletro]{ENG10002}{4}{CIRCUITOS ELÉTRICOS II - C}

     \csummary{Análise sinusoidal de circuitos. Análise de circuitos no domínio da frequência. Potência em circuitos AC. Redes trifásicas. Redes magneticamente acopladas. Transformadores. Quadripolos. Análise de circuitos usando transformada de Laplace.}

     \bibdef{alexander2003fundamentos}
     \bibdef{irwin2003analise}
     \bibdef{nilsson2003circuitos}
     \bibdef[basic]{bird2009circuitos}
     \bibdef[compl]{desoer1979teoria}
     \bibdef[compl]{dorf2003introducao}
     \bibdef[compl]{hayt2008analise}

%%%%%%
%%%%%%
%
\classdef[Base.Digit]{ENG10043}{2}{LABORATÓRIO DE SISTEMAS DIGITIAIS}

     \csummary{Análise e projeto de circuitos lógicos combinacionais e sequenciais. Utilização de circuitos integrados lógicos de pequena, média e grande escala de integração. Utilização de simuladores e ferramentas de apoio ao projeto de circuitos digitais. Implementação de circuitos lógicos em FPGA.}

     \bibdef{brown_fundamentals2}
     \bibdef{floyd_sistemas2}
     \bibdef{wakerly_digital}
     \bibdef[basic]{mano_logic2}
     \bibdef[basic]{tocci_sistemas}
     \bibdef[basic]{vahid_digital2}
     \bibdef[compl]{karris_digital2}

%%%%%%
%%%%%%
%
\classdef[BaseMatematica]{MAT01168}{6}{MATEMÁTICA APLICADA II}

     \csummary{Séries de Fourier. Integral de Fourier. Transformadas de Fourier e de Laplace. Análise vetorial.}

     \bibdef{anton2007calculo}
     \bibdef{hsu2011sinais}
     \bibdef[basic]{hsu1973analise}
     \bibdef[basic]{strauch_notas}
     \bibdef[basic]{kreyszig1983matematica}
     \bibdef[basic]{spiegel1972analise}
     \bibdef[basic]{spiegel_schaum}
     \bibdef[basic]{zill2003equacoes3}
     \bibdef[compl]{asmar2005partial2}
     \bibdef[compl]{oneil2003advanced}
     \bibdef[compl]{spiegel1978transformadas}
     \bibdef[compl]{strang1991calculus2}
     \bibdef[compl]{stroud2003advanced}
     \bibdef[compl]{zill2001equacoes3}

%%%%%%
%%%%%%
%
\classdef[Base.Mat]{ENG03092}{4}{MECÂNICA DOS SÓLIDOS I-A}

     \csummary{Introdução à Mecânica dos Sólidos. Solicitações internas. Tensões e deformações. Esforço axial. Torção. Flexão simples. Cisalhamento em vigas. Solicitações compostas. Análise e transformação de tensões. Análise e transformação de deformações. Critérios de falha. Noções de coeficiente de segurança.}

     \bibdef{beer2013estatica}
     \bibdef{popov1978introducao}
     \bibdef{hibbeler2010resistencia}
     \bibdef[basic]{gere_mecanica}
     \bibdef[basic]{reddy_principles}
     \bibdef[basic]{mendonca2019metodo}
     \bibdef[compl]{gordon_structures}
     \bibdef[compl]{pilkey_modern}
     \bibdef[compl]{shames_introducao}

%%%%%%
%%%%%%
%
\classdef[BaseMec]{ENG03316}{4}{MECANISMOS I}

     \csummary{Introdução a análise de mecanismos: Conceito e classificação de mecanismos. Cadeias Cinemáticas. Análise cinemática dos mecanismos. Cames. Teoria das engrenagens. Forças de inércia em máquinas. Balanceamento estático e dinâmico. Aplicações Industriais ou em Equipamentos.}

     \bibdef[basic]{norton_cinematica}
     \bibdef[basic]{uicker2011theory}
     \bibdef[compl]{mabie1987mechanisms}
     \bibdef[compl]{norton2007design}

%%%%%%
%%%%%%
%
\classdef[BaseMec]{ENG03044}{4}{MODELAGEM DE SISTEMAS MECÂNICOS}

     \csummary{Modelagem e modelos. Tipos de modelos. Modelagem em computador. Estimativas e aproximações. Modelagem sistemática de sistemas mecânicos, elétricos, fluídicos e térmicos. Analogias elétricas. Sistemas híbridos. Técnicas de representação de modelos matemáticos. Respostas transitória e permanente de sistemas dinâmicos. Análise no domínio frequência. Simulação de resposta de sistemas dinâmicos a excitações típicas.}

     \bibdef{close2001modeling}
     \bibdef{kluever2015dynamic}
     \bibdef{palm2013system}
     \bibdef[basic]{kulakowski2012dynamic}
     \bibdef[basic]{lu2014modeling}
     \bibdef[basic]{palm1999modeling}
     \bibdef[basic]{rao2008vibracoes}
     \bibdef[compl]{das2009mechatronic}
     \bibdef[compl]{franklin2013sistemas}
     \bibdef[compl]{golnaraghi_sistemas}
     \bibdef[compl]{ogata_engenharia} 
%%%%%%%%%%%%%%%%%
%%%%%%%%%%%%%%%%%
%%%
%%%  Etapa 05
%%%
%%%%%%%%%%%%%%%%%
%%%%%%%%%%%%%%%%%



%%%%%%
%%%%%%
%
\classdef[BaseEletro]{ENG10044}{4}{ELETRÔNICA FUNDAMENTAL I-B}

     \csummary{Amplificadores operacionais, diodos, circuitos conformadores, transistores de junção e efeito de campo: características, polarização, estabilidade térmica e resposta em frequência. Amplificadores de um ou mais estágios, realimentação e o teorema do elemento extra: princípios de análise de estabilidade e resposta em frequência.}

      \bibdef{sedra2007microeletronica}

      \bibdef[basic]{millman2010integrated}
      \bibdef[basic]{schilling_circuitos}

      \bibdef[compl]{razavi_fundamentos}
      \bibdef[compl]{boylestad_dispositivos}

%%%%%%
%%%%%%
%
\classdef[BaseEletro]{ENG10003}{2}{LABORATÓRIO DE CIRCUITOS ELÉTRICOS}

     \csummary{Instrumentos de medida e conceitos fundamentais de medição. Ferramentas Computacionais de análise, e simulação. Aplicação de análise de circuitos resistivos. Aplicação de análise de circuitos de 1ª e 2ª ordem. Resposta em frequência de circuitos. Análise fasorial. Medição de potência em Circuitos Trifásicos. Transformadores.}

      \bibdef{alexander2008circuitos}
      \bibdef{nilsson1999circuitos}

      \bibdef{irwin2003analise}

      \bibdef[compl]{desoer1979teoria}
      \bibdef[compl]{bird2009circuitos}
      \bibdef[compl]{hayt2008analise}
      \bibdef[compl]{tsividis2001lab}

%%%%%%
%%%%%%
%
\classdef[Base.Mat]{ENG03004}{4}{MECÂNICA DOS SÓLIDOS II}

     \csummary{Análise de tensões. Teorias estruturais. Análise de flexão de vigas. Métodos clássicos de análise de vigas. Métodos de solução de problemas estaticamente indeterminados. Introdução à análise limite em vigas. Princípios energéticos. Flambagem de colunas. Introdução à elasticidade.}

      \bibdef{hibbeler2017analise}

      \bibdef[basic]{beer2006resistencia}
      \bibdef[basic]{salvadori1987estructuras}
      \bibdef[basic]{steffen1982pratica}
      \bibdef[basic]{sussekind1994curso}

%%%%%%
%%%%%%
%
\classdef[Pro.Control]{ENG10017}{6}{SISTEMAS E SINAIS}

     \csummary{Técnicas de modelagem e análise de sistemas lineares e sistemas amostrados. Introdução a sistemas não lineares.}

      \bibdef{haykin2001sinais}

      \bibdef[basic]{franklin2006feedback}
      \bibdef[basic]{lathi2007sinais}
      \bibdef[basic]{oppenheim1997signals}

      \bibdef[compl]{geromel2004analise}
      \bibdef[compl]{hsu2004sinais}
      \bibdef[compl]{olivier2019linear}

%%%%%%
%%%%%%
%
\classdef[Base.FenTrans]{ENG07086}{5}{TERMODINÂMICA E TRANSFERÊNCIA DE CALOR}

     \csummary{Propriedades termodinâmicas de substâncias puras e misturas. Energia, trabalho e calor e as Leis da Termodinâmica. Termodinâmica dos sistemas abertos. Eficiência de processos térmicos. Mecanismos de transferência de calor. Condução de calor em regime estacionário e transiente.}

      \bibdef{borgnakke2009termodinamica}
      \bibdef{smith2005termodinamica}

      \bibdef[basic]{koretsky2017termodinamica}

%%%%%%
%%%%%%
%
\classdef[Pro.Robotica]{ENG10026}{4}{ROBÓTICA-A}

     \csummary{Estrutura de robô: características, acionamento, controle, manipuladores e sensores. Capacidade do robô. Aplicações do robô. Noções de cinemática e dinâmica. Programação do robô. Sistemas de programação. Sistema controlador - periféricos-robô.}

      \bibdef[basic]{craig2005robotics}
      \bibdef[basic]{fu1987robotics}

      \bibdef[compl]{asada1986robot}
      \bibdef[compl]{goebel2015ros}
      \bibdef[compl]{martinez2015ros}
      \bibdef[compl]{OKane2013a}
      \bibdef[compl]{romano2002}
      
%      \bibdef[compl]{spong2005robot}
%      \bibdef[compl]{groover_industrial}
%%TODO: those are wrong... missing some
%%TODO: MISSED ENG03386

\classdef[Pro.Robotica]{ENG03380}{4}{ROBÓTICA}

     \csummary{Configurações físicas de robôs, movimentos básicos, características técnicas, programação elementar, tipos de linguagens, efetuadores finais, controle da célula de trabalho. Aplicação, dados de projeto.}


      \bibdef[basic]{craig2013}
      \bibdef[basic]{niku2013}
      \bibdef[compl]{siciliano_robotics}
      \bibdef[compl]{pazos2002}
      \bibdef[compl]{rosario2005}
      \bibdef[compl]{spong2005robot}
      \bibdef[compl]{groover_industrial}


%%%%%%%%%%%%%%%%%
%%%%%%%%%%%%%%%%%
%%%
%%%  Etapa 06
%%%
%%%%%%%%%%%%%%%%%
%%%%%%%%%%%%%%%%%




%%%%%%
%%%%%%
%
\classdef[Pro.Maquinas]{ENG10047}{4}{FUNDAMENTOS DE MÁQUINAS ELÉTRICAS}

     \csummary{Princípios de conversão eletromecânica de energia. Dispositivos eletromagnéticos. Máquinas de corrente contínua. Máquinas de corrente alternada. Modelos de dispositivos em regime permanente. Características operacionais em regime permanente.}

      \bibdef{fitzgerald_maquinas}
      \bibdef{white_electromechanical}
      \bibdef{bim_maquinas}

      \bibdef[basic]{ivanovsmolensky1980machines}
      \bibdef[basic]{kostenko1979maquinas}
      \bibdef[basic]{krause_analysis}
      \bibdef[basic]{nasar_electric}

      \bibdef[compl]{gross_electric}
      \bibdef[compl]{hameyer_numerical}
      \bibdef[compl]{chapman_electric}

%%%%%%
%%%%%%
%
\classdef[Pro.Maquinas]{ENG10022}{4}{INSTRUMENTAÇÃO FUNDAMENTAL PARA CONTROLE E AUTOMAÇÃO}

     \csummary{Medidas em processos industriais. Precisão, erros e sua propagação. Transdutores para medição de grandezas físicas. Condicionamento de sinais e interfaceamento. Métodos indiretos de medida.}

      \bibdef{balbinot2011instrumentacao}
      \bibdef{balbinot_instrumentacao}

      \bibdef[basic]{doebelin_measurement}
      \bibdef[basic]{fraden2010sensors}
      \bibdef[basic]{pallasareny_sensors}

      \bibdef[compl]{considine_process}
      \bibdef[compl]{holman_experimental}

%%%%%%
%%%%%%
%
\classdef[BaseEletro]{ENG10045}{2}{LABORATÓRIO DE ELETRÔNICA}

     \csummary{Instrumentos de medida e conceitos fundamentais de medição. Ferramentas computacionais de análise e simulação de circuitos não-lineares: diodos, transistores de junção e efeito de campo. Resposta em frequência de circuitos ativos. Circuitos conformadores, amplificadores de um e de diversos estágios realimentados. Amplificadores operacionais.}

      \bibdef{sedra_microeletronica}
      \bibdef{silva2008circuitos}

      \bibdef[basic]{desoer2006circuit}

      \bibdef[compl]{cordell_amplifiers}

%%%%%%
%%%%%%
%
\classdef[Base.FenTrans]{ENG07069}{2}{PRINCÍPIOS DA MECÂNICA DE FLUIDOS}

     \csummary{Princípios de transferência de quantidade de movimento. Equações de conservação nas formas integral e diferencial. Estática dos fluidos. Camada limite. Equações de projeto para sistemas de transporte de fluidos.}

      \bibdef{potter2013mecanica}
      \bibdef{fox2017mecanica}
      \bibdef{welty2017fundamentos}

      \bibdef[basic]{bird2004fenomenos}

%%%%%%
%%%%%%
%
\classdef[Pro.Automacao]{ENG10023}{4}{SISTEMAS DE AUTOMAÇÃO}

     \csummary{Sistemas de automação industrial e de controle de processos. Técnicas de Modelagem e Metodologia de Desenvolvimento de Sistemas de Automação Industrial (Clássica e Orientada a objetos), Sistemas de Tempo Real (Linguagens de Programação, Sistemas Operacionais).}

      \bibdef{ward_structured}
      \bibdef{selic1994realtime}

      \bibdef[compl]{awad1996object}

%%%%%%
%%%%%%
%
\classdef[Pro.Control]{ENG10004}{4}{SISTEMAS DE CONTROLE I - B}

     \csummary{Modelagem e identificação de sistemas dinâmicos. Conceitos básicos e problemas fundamentais em sistemas de controle. Controladores PID: Teoria e ajuste. Projeto de controladores para sistemas monovariáveis via método do lugar das raízes. Aspectos não-lineares em sistemas de controle.}

      \bibdef{bazanella2005controle}

      \bibdef[basic]{astrom1995pid}
      \bibdef[basic]{franklin_feedback}
      \bibdef[basic]{ogata_controle}

      \bibdef[compl]{dorf_controle}
      \bibdef[compl]{kuo_controle} 
%%%%%%%%%%%%%%%%%
%%%%%%%%%%%%%%%%%
%%%
%%%  Etapa 07
%%%
%%%%%%%%%%%%%%%%%
%%%%%%%%%%%%%%%%%

%%%%%%
%%%%%%
%
\classdef[Pro.Maquinas]{ENG10049}{4}{ACIONAMENTO DE MÁQUINAS ELÉTRICAS}

     \csummary{Seleção de motores elétricos. Comportamento e modelos dinâmicos de máquinas elétricas. Controle de velocidade e torque. Princípios de eletrônica potência, operação e componentes básicos de conversores estáticos. Acionamento de máquinas com conversores estáticos.}

      \bibdef{bose1997}
      \bibdef{bin2009}
      \bibdef{krause2013}
      \bibdef[basic]{hughes2006}
      \bibdef[basic]{boldea1999}
      \bibdef[basic]{pillai1989}
      \bibdef[basic]{leonhard1990}
      \bibdef[compl]{white1959}
      \bibdef[compl]{slemon1992}
      \bibdef[compl]{cathey2000}
      \bibdef[compl]{murphy1988}

%%%%%%
%%%%%%
%
\classdef[Pro.Control]{ENG10005}{2}{LABORATÓRIO DE CONTROLE}

     \csummary{Métodos experimentais para ajuste de controladores PID (Ziegler-Nichols e similares). Verificação experimental de desempenho de malhas de controle. Projeto de controladores por métodos de controle clássico. Projeto prático de controladores para: processo térmico, processo mecânico, controle de velocidade de motor elétrico, controle de posição de motor elétrico.}

      \bibdef{bazanella2005}
      \bibdef[basic]{ogata2003}
      \bibdef[compl]{astrom1995}
      \bibdef[compl]{franklin2006}

%%%%%%
%%%%%%
%
\classdef[Pro.Automacao]{ENG04475}{5}{MICROPROCESSADORES I}

     \csummary{Arquitetura de microprocessadores. Endereçamento e conjunto de instruções. Memória e adaptadores de interface de entrada e saída. Projeto lógico e elétrico de sistemas microprocessados. Sistemas supervisores. Programação e algoritmos.}

      \bibdef{cady2010}
      \bibdef{pont2002}
      \bibdef{susnea2005}
      \bibdef[basic]{balch2003}
      \bibdef[basic]{mcfarland2006}
      \bibdef[basic]{nicolosi2013}
      \bibdef[basic]{nicolosi2014}
      \bibdef[basic]{sengupta2010}
      \bibdef[basic]{silvajr2004}
      \bibdef[basic]{silvajr1990}
      \bibdef[compl]{hennessy2013a}
      \bibdef[compl]{hennessy2012}
      \bibdef[compl]{patterson2013}
      \bibdef[compl]{patterson2013b}

%%%%%%
%%%%%%
%
\classdef[Pro.Fabricacao]{ENG03021}{4}{PROCESSOS DISCRETOS DE PRODUÇÃO}

     \csummary{Introdução aos Materiais Empregados em Engenharia e Seleção de materiais. Fundição: princípios; principais tipos de moldes utilizados e processos de fabricação empregados, tais como fornos elétricos, por indução, etc. Conformação Mecânica: princípios; principais processos 
     empregados, tais como laminação, forjamento, etc., `a quente' e `a frio'. Usinagem: princípios; principais métodos empregados, tais como torno, retífica, etc. Soldagem e Técnicas Conexas: princípios; principais processos empregados, tais como eletrodo revestido, MIG/MAG, etc. Introdução 
     ao Planejamento das Operações de Manufatura, considerações econômicas e comparações de custos entre os processos descritos.}

      \bibdef{groover2012}
      \bibdef[basic]{diniz2006}
      \bibdef[basic]{ferreira1999}
      \bibdef[basic]{cetlin2005}
      \bibdef[basic]{marques2009}
      \bibdef[compl]{lefteri2009}
      \bibdef[compl]{dieter1981}

%%%%%%
%%%%%%
%
\classdef[Pro.Automacao]{ENG10048}{4}{PROTOCOLOS DE COMUNICAÇÃO}

     \csummary{Conceitos básicos de redes de computadores. Definição de sistemas abertos (modelo ISO/OSI). Nível físico, enlace de dados; algoritmos de detecção e correção de erro, redes, transporte e aplicação. Barramentos industriais para automação e instrumentação: IEEE448, Profibus, Fieldbus, CAN-BUS e outros protocolos de chão-de-fábrica.}

      \bibdef{tanenbaum2003}
      \bibdef[basic]{soares1996}

%%%%%%
%%%%%%
%
\classdef[Pro.Control]{ENG10018}{4}{SISTEMAS DE CONTROLE II}

     \csummary{Análise e projeto de sistemas de controle por métodos freqüenciais. Sensibilidade e robustez de sistemas de controle. Análise de ciclo-limite em sistemas não-lineares. Modelagem, análise e projeto de sistemas de controle por variáveis de estado.}

      \bibdef{bazanella2005b}
      \bibdef{ogata2003b}
      \bibdef[basic]{franklin2002}
      \bibdef[compl]{boldrini1986}
      \bibdef[compl]{chen1998}
      \bibdef[compl]{campestrini2006}
      \bibdef[compl]{zanchin2003}

%%%%%%
%%%%%%
%
\classdef[Pro.Maquinas]{ENG03027}{4}{SISTEMAS HIDRÁULICOS E PNEUMÁTICOS}

     \csummary{Introdução à hidráulica e pneumática industrial. descrição de componentes para circuitos de comando e controle: atuadores, válvulas, cilindros, bombas e compressores. Características e propriedades dos fluidos hidráulicos. Elementos de mecatrônica.}

      \bibdef{prudente2013}
      \bibdef{rabie2009}
      \bibdef{watton2012}
      \bibdef[basic]{capuamo2009}
      \bibdef[basic]{cundiff2011}
      \bibdef[basic]{linsingen2001}
      \bibdef[compl]{bollmann1996}
      \bibdef[compl]{bolton1996}
      \bibdef[compl]{martin1990}
      \bibdef[compl]{merritt1967}
      \bibdef[compl]{parr2007}
      \bibdef[compl]{yeaple1996}

%%%%%%%%%%%%%%%%%
%%%%%%%%%%%%%%%%%
%%%
%%%  Etapa 08
%%%
%%%%%%%%%%%%%%%%%
%%%%%%%%%%%%%%%%%

%%%%%%
%%%%%%
%
\classdef[Pro.Fabricacao]{ENG03045}{4}{ELEMENTOS DE MÁQUINAS}

     \csummary{Noções básicas sobre projeto mecânico. Fadiga dos materiais. Eixos de transmissão. Dimensionamento, seleção e aplicação de molas, 
     rolamentos, mancais de escorregamento, engrenagens, freios e embreagens, elementos flexíveis, parafusos de fixação, acoplamentos elásticos, elementos de transmissão de movimento.}

      \bibdef{juvinal2008}
      \bibdef[basic]{Norton2004c}
      \bibdef[basic]{shigley2004}
      \bibdef[compl]{hibbeler2010}

%%%%%%
%%%%%%
%
\classdef[Pro.ContProc]{ENG07042}{4}{MODELAGEM E CONTROLE DE PROCESSOS INDUSTRIAIS}

     \csummary{Introdução à modelagem matemática de processos industriais. Aplicação das leis de conservação em regime estacionário e dinâmico. 
     Equações constitutivas. Simulação estática e dinâmica de processos. Malhas de controle típicas da indústria de processos. Projeto de controladores aplicados na indústria de processos.}

      \bibdef{bequette1998}
      \bibdef{Campos2006c}
      \bibdef{seborg2003}
      \bibdef[basic]{Edgar2001b}
      \bibdef[basic]{Luyben1990b}
      \bibdef[basic]{rice1995}
      \bibdef[compl]{hangos2001}
      \bibdef[compl]{Ogunnaike1994b}

%%%%%%
%%%%%%
%
\classdef[Pro.Control]{ENG10019}{4}{SISTEMAS DE CONTROLE DIGITAIS}

     \csummary{Análise de Sistemas de Controle amostrados através da transformada Z. Digitalização de controladores analógicos. Identificação de sistemas pelo método dos mínimos quadrados. Projeto de controladores digitais para sistemas monovariáveis. Implementação de controladores digitais.}

      \bibdef{bazanella2012}
      \bibdef{astrom1997}
      \bibdef{franklin2006b}
      \bibdef[basic]{aguirre2007}
      \bibdef[basic]{kuo2002}
      \bibdef[basic]{ogata1995}
      \bibdef[compl]{bazanella2005c}
      \bibdef[compl]{houpis1991}
      \bibdef[compl]{hemerly2000}
      \bibdef[compl]{geromel2004}

%%%%%%
%%%%%%
%
\classdef[Pro.Fabricacao]{ENG03387}{4}{SISTEMAS DE FABRICAÇÃO}

     \csummary{Modos de produção e arranjo físico industrial. Modelos e métricas de produção. Análise de sistemas de produção: estações de trabalho operadas manualmente e automatizadas, análise de grupos de máquinas, linhas de montagem. Tecnologia de grupo: sistemas de codificação e 
     classificação, métodos para formação de famílias de peças e de células de manufatura. Manufatura celular. Movimentação interna de materiais e armazenamento. Introdução à fabricação CNC.}

      \bibdef{groover2010}
      \bibdef[basic]{black2000}
      \bibdef[compl]{groover2007}
      \bibdef[compl]{liker2005}
      \bibdef[compl]{lorino1990}

%%%%%%
%%%%%%
%
\classdef[Pro.Fabricacao]{ENG03386}{4}{FABRICAÇÃO AUXILIADA POR COMPUTADOR}

     \csummary{Processos de fabricação por usinagem: torneamento, furação e fresamento. Planejamento de processo e CAPP. Projeto orientado à manufatura e montagem (DFM/A). Fundamentos da usinagem CNC. Arquitetura de sistemas CNC: hardware e software. Linguagem ISSO. Programação 
     manual de centros de usinagem CNC. Programação manual de tornos CNC. CAD/CAM.}

      \bibdef{groover2010b}
      \bibdef[basic]{chang2005}
      \bibdef[basic]{sawhney2007}
      \bibdef[compl]{amorim2012a}
      \bibdef[compl]{amorim2012b}
      \bibdef[compl]{amorim2013}
      \bibdef[compl]{cornelius2003}
      \bibdef[compl]{groover2007b}

%%%%%%
%%%%%%
%
\classdef[Pro.Control]{ENG03046}{4}{CONTROLE DE SISTEMAS FLUÍDO-MECÂNICOS}

     \csummary{Modelagem dinâmica de sistemas hidráulicos, pneumáticos e híbridos. Características não lineares de sistemas hidráulicos e pneumáticos: aspectos construtivos e análise por aproximações lineares. Servoatuadores hidráulicos e pneumáticos: análise, controle e aplicações.}

      \bibdef{manring2005}
      \bibdef{Slotine1991a}
      \bibdef{watton2012b}
      \bibdef[basic]{costa2015}
      \bibdef[basic]{cundiff2013}
      \bibdef[basic]{linsingen2008}
      \bibdef[basic]{merritt1991}
      \bibdef[basic]{watton2012c}
      \bibdef[compl]{akers2006}
      \bibdef[compl]{dorf2008}
      \bibdef[compl]{franklin2013}
      \bibdef[compl]{ogata2010}
      \bibdef[compl]{parr2011}
      \bibdef[compl]{rabie2009b}
      \bibdef[compl]{watton2007}

%%%%%%
%%%%%%
%
\classdef[Pro.Maquinas]{ENG10027}{4}{ELETRÔNICA FUNDAMENTAL II - B}

     \csummary{Amplificador operacional: modelamento e características. Circuitos não-lineares com amplificadores operacionais: conformadores, comparadores, detectores de pico, amostradores, conversores tensão-frequência, amplificadores logarítmicos, mono-estáveis, estáveis. Circuitos integrados especiais e aplicações. Conceitos básicos de comportamento em frequência de amplificadores.}

      \bibdef{sedra2007}
      \bibdef[basic]{graeme1981}
      \bibdef[basic]{razavi2010}
      \bibdef[basic]{franco1998}
      \bibdef[basic]{wait1991}
      \bibdef[basic]{jung2004}
      \bibdef[basic]{wong1972}
      \bibdef[compl]{lancaster1975}
      \bibdef[compl]{ott1988} 
%%%%%%%%%%%%%%%%%
%%%%%%%%%%%%%%%%%
%%%
%%%  Etapa 09
%%%
%%%%%%%%%%%%%%%%%
%%%%%%%%%%%%%%%%%



%%%%%%
%%%%%%
%
\classdef[Pro.ContProc]{ENG07087}{3}{CONTROLE AVANÇADO DE PROCESSOS}

     \csummary{Introdução à análise e controle de sistemas MIMO. Técnicas de controle avançado de processos: Controle preditivo baseado em modelo (MPC). Estimadores de estados: Filtro de Kalman e Filtro de Kalman Estendido.}

%%%%%%
%%%%%%
%
\classdef[Transv.integ]{TCC/CCA - I}{2}{TRABALHO DE CONCLUSÃO DE CURSO / CCA - I}

     \csummary{Tema de livre escolha do aluno dentro do ramo da Engenharia de Controle e Automação. Cada aluno terá um professor orientador e o trabalho final será examinado por professores que atuam na parte profissionalizante e específica do curso. O trabalho consistirá de uma monografia preliminar, propondo, contextualizando e delineando um plano de solução para um problema de Engenharia de Controle e Automação, a ser completado até o final da atividade de TCC/CCA - II, e deverá consistir minimamente da pesquisa bibliográfica e estado da arte do problema proposto, bem como a análise de viabilidade técnica/econômica da solução pretendida.}

%%%%%%   
%%%%%%
%
\classdef[Pro.Automacao]{ENG10021}{4}{SISTEMAS A EVENTOS DISCRETOS}

     \csummary{Sistemas a eventos discretos: conceituação, propriedades. Redes de Petri: conceitos básicos e aplicações na modelagem e controle de sistemas a eventos discretos. Teoria de autômatos: modelos de autômatos e aplicações ao controle de sistemas a eventos discretos. Sistemas de supervisão: conceituação aplicações em sistemas de automação.}

      \bibdef{cardoso1990}
      \bibdef{cassandras2010}
      \bibdef{miyagi1996}
      \bibdef[basic]{cassandras1993}
      \bibdef[basic]{aguirre2007b}
      \bibdef[basic]{reisig2013}
      \bibdef[basic]{seatzu2013}
      \bibdef[basic]{zhou2007}
      \bibdef[compl]{hein2010}
      \bibdef[compl]{banks2005}
      \bibdef[compl]{jensen2009}
      \bibdef[compl]{popova2013}

%%%%%%  
%%%%%%
%
\classdef[Pro.Maquinas]{ENG03047}{4}{PROJETOS DE SISTEMAS MECÂNICOS}

     \csummary{Princípios de projeto de um sistema mecânico. Estudo de problemas do projeto mecânico em geral. Aplicações em diversas áreas com ênfase em: controle e supressão de vibrações; avaliação e seleção de atuadores e sensoriamento para robôs; cinemática e controle de trajetória de robôs.}

      \bibdef{siciliano2009}
      \bibdef{pahl2007}
      \bibdef[basic]{nwokah2002}
      \bibdef[compl]{dorf1994}
      \bibdef[compl]{rao2008}
      \bibdef[compl]{spong2005}

%%%%%% 
%%%%%%
%
\classdef[Pro.Maquinas]{ENG10050}{2}{LABORATÓRIO DE MÁQUINAS E ACIONAMENTOS}

     \csummary{Ferramentas de análise de campos elétricos e magnéticos. Características operacionais da máquina CC. Características operacionais de máquina CA. Controle de velocidade de máquinas elétricas. Acionamentos usando conversores estáticos.}

      \bibdef{fitzgerald2014}
      \bibdef{bim2014}
      \bibdef{kostenko1979}
      \bibdef[basic]{rezek2011}
      \bibdef[basic]{white1959}
      \bibdef[basic]{mohan2015}

%%%%%% 
%%%%%%
%
\classdef[Pro.Maquinas]{ENG10046}{2}{PRINCÍPIOS DE ELETRÔNICA DE POTÊNCIA}

     \csummary{Princípios de operação e componentes básicos de conversores estáticos. Conversores CA-CC (retificadores controlados e não-controlados), CA-CA, CC-CC, CC-CA (inversores, tipos de modulação). Aplicação de conversores para acionamento de máquinas elétricas.}

      \bibdef{vithayathil1995}
      \bibdef{rashid2015}
      \bibdef{mohan2003}
      \bibdef[basic]{bose1997}
      \bibdef[basic]{aburub2014}
      \bibdef[basic]{murphy1988}
      \bibdef[basic]{martins2000}
      \bibdef[basic]{martins2005}
      \bibdef[basic]{sen1989}
      \bibdef[compl]{barbi2006}
      \bibdef[compl]{elhawary2002}
      \bibdef[compl]{bollen2011}

%%%%%%%%%%%%%%%%%
%%%%%%%%%%%%%%%%%
%%%
%%%  Etapa 10
%%%
%%%%%%%%%%%%%%%%%
%%%%%%%%%%%%%%%%%



%%%%%%
%%%%%%

\classdef[Transv.outros]{ENG03010}{3}{CIÊNCIA, TECNOLOGIA E AMBIENTE}

     \csummary{Ecologia: conceitos básicos. A biosfera e seu equilíbrio, desenvolvimento sustentável. Ciência e tecnologia: conceitos básicos, efeitos da tecnologia sobre o equilíbrio ambiental, tecnologia e desenvolvimento sócio-econômico. O ambiente industrial, legislação ambiental brasileira, a preservação dos recursos naturais, aspectos internos e externos do ambiente industrial, geração e o impacto de resíduos (sólidos, líquidos e pastosos) industriais, o tratamento e disposição final dos resíduos industriais, planejamento ambiental da atividade industrial.}

      \bibdef{macedo2000}
      \bibdef{vesilind1991}
      \bibdef[basic]{ipt2000}
      \bibdef[basic]{aisse1982}
      \bibdef[basic]{azevedo1977}
      \bibdef[basic]{lima1991}
      \bibdef[basic]{menegat2006}
      \bibdef[basic]{valle1996}
      \bibdef[compl]{campbell1995}
      \bibdef[compl]{ely1990}
      \bibdef[compl]{franciss1980}
      \bibdef[compl]{jacobi1989}
      \bibdef[compl]{knijnik1994}
      \bibdef[compl]{mandelli1991}
      \bibdef[compl]{mota1981}

%%%%%%
%%%%%%
%
\classdef[Transv.outros]{ENG03048}{4}{GERÊNCIA E ADMINISTRAÇÃO DE PROJETOS}

     \csummary{Idéias, técnicas e metodologias avançadas para o planejamento, controle e desenvolvimento de projetos de sistemas. Apresentação de um processo disciplinado e estruturado de administração de projetos de sistemas, segundo uma visão de negócio, de forma a cumprir prazos, orçamentos e requisitos. Exploração dos principais componentes do processo de gerenciamento de projetos nas organizações, fornecendo ferramental para projetar, avaliar e medir a efetividade e os fatores de risco da implementação de projetos. Técnicas de elaboração de estimativas de custos, prazos e recursos nos projetos de desenvolvimento de sistemas. Controle e garantia da qualidade no desenvolvimento de sistemas.}

      \bibdef{carvalho2008}
      \bibdef{pmi2000}
      \bibdef[basic]{kerzner2006}
      \bibdef[basic]{carvalho2006}
      \bibdef[basic]{rabechini2009}

%%%%%%
%%%%%%
%
\classdef[Transv.integ]{TCC/CCA - II}{2}{TRABALHO DE CONCLUSÃO DE CURSO / CCA - II}

     \csummary{Tema de livre escolha do aluno dentro do ramo da Engenharia de Controle e Automação, continuação do trabalho iniciado em TCC/CCA - I. Cada aluno, sob orientação de um professor, deverá concluir a análise iniciada em TCC/CCA - I, desenvolvendo e implementando a solução do problema proposto. A solução será documentada sob a forma de monografia, a ser apresentada perante uma banca de professores que atuam na parte profissionalizante e específica do curso.} 
%%%%I t%%%%%%%%%%%%%
%%%%%%%%%%%%%%%%%
%%%
%%%  Etapa Eletivas
%%%
%%%%%%%%%%%%%%%%%
%%%%%%%%%%%%%%%%%

%%%%%%
%%%%%%
%
\classdef[Transv.integ]{CCA99008}{9}{PROJETO INTEGRADO I}

     \csummary{Atuação em equipes para analisar, propor e desenvolver soluções para problemas de Engenharia de interesse da sociedade, contemplando seus aspectos técnicos, 
     econômicos, socioambientais, de acessibilidade e de prevenção de desastres, entre outros. Os problemas serão colhidos 
     junto à sociedade via ação de extensão vinculada, sendo de natureza inerentemente aberta, prática e integradora, oportunizando que os 
     estudantes trabalhem simultaneamente os conteúdos aprendidos múltiplas disciplinas diferentes e em contextos realistas. Desenvolvimento de 
     habilidades de trabalho autônomo, comunicação, convívio social e respeito à diversidade através da atuação em grupos e do contato com questões
      e/ou indivíduos externos à Universidade. Dada a natureza integradora desta atividade, espera-se que os alunos de Projeto Integrado I, 
      interajam com os alunos de Projeto Integrado II e III no processo, permitindo a troca de experiência de alunos em diversos momentos na parte 
      profissionalizante do curso.}

      \bibdef{Brockman2009}
      \bibdef{Silva2006}
      \bibdef{Junival2008}
      \bibdef[basic]{Shigley2005}
      \bibdef[basic]{Bishop2002}
      \bibdef[compl]{Silva2008}
      \bibdef[compl]{Onwubolu2005}
      \bibdef[compl]{Bishop2008}
      \bibdef[compl]{Norton2004}
%%%%%%
%%%%%%
%
\classdef[Transv.integ]{CCA99009}{9}{PROJETO INTEGRADO II}

     \csummary{Atuação em equipes para analisar, propor e desenvolver soluções para problemas de Engenharia de interesse da sociedade, contemplando seus aspectos técnicos, 
     econômicos, socioambientais, de acessibilidade e de prevenção de desastres, entre outros. Os problemas serão colhidos 
     junto à sociedade via ação de extensão vinculada, sendo de natureza inerentemente aberta, prática e integradora, oportunizando que os 
     estudantes trabalhem simultaneamente os conteúdos aprendidos múltiplas disciplinas diferentes e em contextos realistas. Desenvolvimento de 
     habilidades de trabalho autônomo, comunicação, convívio social e respeito à diversidade através da atuação em grupos e do contato com questões 
     e/ou indivíduos externos à Universidade.  Dada a natureza integradora desta atividade, espera-se que os alunos de Projeto Integrado II, 
     interajam com os alunos de Projeto Integrado I e III no processo, permitindo a troca de experiência de alunos em diversos momentos na parte profissionalizante do curso.}

      \bibdef{Juvinall2008}
      \bibdef{Silva2006b}
      \bibdef[basic]{Bishop2002b}
      \bibdef[basic]{Shigley2005b}
      \bibdef[compl]{Bishop2008b}
      \bibdef[compl]{Norton2004b}
      \bibdef[compl]{Onwubolu2005b}
      \bibdef[compl]{Silva2008b}
%%%%%%
%%%%%%
%
\classdef[Transv.integ]{CCA99010}{9}{PROJETO INTEGRADO III}

     \csummary{Atuação em equipes para analisar, propor e desenvolver soluções para problemas de Engenharia de interesse da sociedade, contemplando 
     seus aspectos técnicos, econômicos, socioambientais, de acessibilidade e de prevenção de desastres, entre outros. Os problemas serão colhidos 
     junto à sociedade via ação de extensão vinculada, sendo de natureza inerentemente aberta, prática e integradora, oportunizando que os 
     estudantes trabalhem simultaneamente os conteúdos aprendidos múltiplas disciplinas diferentes e em contextos realistas. Desenvolvimento de 
     habilidades de trabalho autônomo, comunicação, convívio social e respeito à diversidade através da atuação em grupos e do contato com questões
      e/ou indivíduos externos à Universidade.  Dada a natureza integradora desta atividade, espera-se que os alunos de Projeto Integrado III, 
      interajam com os alunos de Projeto Integrado I e II no processo, permitindo a troca de experiência de alunos em diversos momentos na parte 
      profissionalizante do curso.}

      \bibdef{Cetinkunt2015}
      \bibdef{Horowitz2015}
      \bibdef{Russell2015}
      \bibdef[basic]{Silva2006c}
      \bibdef[basic]{Junival2008b}
      \bibdef[basic]{Bishop2002c}
      \bibdef[compl]{Silva2008c}
      \bibdef[compl]{Onwubolu2005c}
      \bibdef[compl]{Bishop2008c}
%%%%%%
%%%%%%
%
\classdef[Pro.Automacao]{INF1017}{4}{APRENDIZADO DE MÁQUINA}

     \csummary{Fundamentos da área de aprendizado de máquina e algoritmos baseados em redes neurais e em abordagens estatísticas. Aplicações para a resolução de problemas de aprendizado supervisionado, não-supervisionado, e por reforço.}

      \bibdef{Faceli2011}
      \bibdef{Russell2010}
      \bibdef{Sutton1999}
      \bibdef[basic]{Haykin2001b}
      \bibdef[compl]{Geron2022}
      \bibdef[compl]{Haykin2009}
%%%%%%
%%%%%%
%
\classdef[Pro.Automacao]{INF01037}{4}{COMPUTAÇÃO EVOLUTIVA}

     \csummary{Conceitos básicos sobre Vida Artificial. Introdução ao Paradigma de Algoritmos Genéticos. Aplicações de Algoritmos Genéticos. Programação evolutiva.}

      \bibdef{Eiben2003}
      \bibdef{Michalewicz1999}
      \bibdef{Peitgen1992}
      \bibdef[basic]{Barone2003}
      \bibdef[basic]{Fogel2000}
      \bibdef[basic]{Jones2008}
      \bibdef[basic]{Mitchell1996}
%%%%%%
%%%%%%
%
\classdef[Transv.outros]{ADM01135}{2}{ENGENHARIA ECONÔMICA E AVALIAÇÕES}

     \csummary{Introdução à engenharia econômica. Engenharia de avaliações. Projetos econômicos.}

      \bibdef{Casarotto2007}
      \bibdef[basic]{Blank2008}
      \bibdef[basic]{DalZot2008}
      \bibdef[basic]{Gitman2004}
      \bibdef[basic]{Hirschfeld1998}
      \bibdef[basic]{Ross2002}
%%%%%%
%%%%%%
%
\classdef[Pro.ContProc]{ENG07043}{2}{INSTRUMENTAÇÃO DE PROCESSOS INDUSTRIAIS}

     \csummary{Fluxograma de engenharia, normas para descrever estratégias de controle de processos industriais. Principais estratégias de controle 
     utilizadas para controlar colunas de destilação, reatores químicos, trocadores de calor, fornos, biorreatores e demais processos usados nas 
     indústrias de processos. Utilização industrial de malhas de controle feedback, cascata e feedforward. Dimensionamento de válvulas de controle e 
     atuadores. Apresentação dos principais instrumentos de medição utilizados no cenário industrial. Medidores de temperatura, pressão, vazão, 
     nível e composição/analisadores. Descrição e quantificação dos erros de medição. Desenvolvimento de inferidores para acompanhar variáveis de 
     difícil medição.}

      \bibdef{Campos2006}
      \bibdef[basic]{Liptak1995a}
      \bibdef[basic]{Liptak1995b}
%%%%%%
%%%%%%
%
\classdef[Pro.ContProc]{ENG07012}{3}{LABORATÓRIO DE CONTROLE E OPERAÇÃO DE PROCESSOS}

     \csummary{Aulas práticas de laboratório, contemplando experimentos, coleta de dados e interpretação de resultados, em assuntos abordados nas disciplinas instrumentação da indústria química, cálculo de reatores e controle de processos.}

      \bibdef{Ogunnaike1994}
      \bibdef{Campos2006b}
      \bibdef{Seborg2004}
      \bibdef[basic]{Luyben1990}
      \bibdef[basic]{Levine1996}
      \bibdef[compl]{Foust1980}
      \bibdef[compl]{Macintyre1997}
      \bibdef[compl]{Mccabe2005}
      \bibdef[compl]{Perry2007}
      \bibdef[compl]{Welty2007}
%%%%%%
%%%%%%
%
\classdef[Transv.outros]{EDU03071}{2}{LÍNGUA BRASILEIRA DE SINAIS (LIBRAS)}

     \csummary{Aspectos linguísticos da Língua Brasileira de Sinais (LIBRAS). História das comunidades surdas, da cultura e das identidades surdas. Ensino básico da LIBRAS. Políticas linguísticas e educacionais para surdos.}

      \bibdef{Campello2014}
      \bibdef{Gesser2009}
      \bibdef{Quadros2004}
      \bibdef[basic]{Brasil2005}
      \bibdef[basic]{Brasil2010}
      \bibdef[basic]{Brasil2002}
      \bibdef[basic]{Brasil2015}
      \bibdef[basic]{Karnopp2014}
      \bibdef[basic]{Pontin2014a}
      \bibdef[basic]{Pontin2014b}
      \bibdef[compl]{Gomes2015}
      \bibdef[compl]{Karnopp2010}
      \bibdef[compl]{Thoma2010}
      \bibdef[compl]{Thoma2004}
%%%%%%
%%%%%%
%
\classdef[Pro.Control]{ENG07062}{3}{OTIMIZAÇÃO APLICADA}

     \csummary{Os métodos de programação matemática (métodos de otimização) são apresentados aplicados à solução de diferentes classes de problemas, 
     tais como: síntese de processos, programação de produção e logística, estimação de parâmetros, otimização em tempo real, controle preditivo, 
     entre outras aplicações encontradas comumente na engenharia. O curso inicia com a revisão de conceitos básicos de otimização, tais como: 
     critérios de optimalidade, convexidade, linearidade, continuidade, etc. A seguir as diversas técnicas empregadas para resolver as diferentes 
     formulações de problemas de otimização são apresentadas, segundo a seguinte classificação comumente adotada: a) programação não linear (NLP) 
     com e sem restrições; b) programação linear (LP); c) programação quadrática (QP); d) Programação inteira mista linear (MILP); e) programação 
     inteira mista não linear (MINLP); f) programação dinâmica e g) otimização global. Cada uma dessas técnicas é apresentada tendo como ponto de partida uma aplicação real.}

      \bibdef{Chong2001}
      \bibdef{Bartholomew2008}
      \bibdef{Nocedal2006}
      \bibdef[basic]{Antoniou2007}
      \bibdef[basic]{Edgar2001}
      \bibdef[basic]{Hendrix2010}
      \bibdef[basic]{Biegler2010}
      \bibdef[compl]{Neumann2010}
      \bibdef[compl]{Schaffler2012}


%%%%%%
%%%%%%
%
\classdef[Transv.outros]{ENG09023}{2}{PLANEJAMENTO ESTRATÉGICO DA PRODUÇÃO}

     \csummary{Administrar estrategicamente é um processo contínuo e interativo que visa manter a organização como um conjunto apropriadamente 
     integrado ao seu ambiente. Não é mais suficiente gerenciar a organização como um objeto específico; é preciso gerenciar o negócio da 
     organização, envolvendo fatores, influências, recursos e variáveis externas e internas, buscando competitividade. Com o planejamento 
     estratégico, não se pretende adivinhar o futuro. O intuito é traçar objetivos futuros viáveis e propor ações para alcançá-los. Na disciplina 
     são discutidos Negócio, Missão e Princípios organizacionais, Análise do Ambiente e identificação de oportunidades e ameaças, Definição de Visão 
     e objetivos a serem alcançados, além da Definição de Estratégias para atingir os objetivos, com ênfase na discussão de estratégias de produção.}

      \bibdef{Muller2003}
      \bibdef[compl]{Ansoff1990}
      \bibdef[compl]{Ansoff1993}
      \bibdef[compl]{Campos2004}
      \bibdef[compl]{Certo2005}
      \bibdef[compl]{Kaplan1997}
      \bibdef[compl]{Kaplan2001}
      \bibdef[compl]{Mintzberg2000}
      \bibdef[compl]{Naisbitt1997}
      \bibdef[compl]{Oliveira1999}
      \bibdef[compl]{Paiva2009}
      \bibdef[compl]{Popcorn1997}
      \bibdef[compl]{Porter1989}
      \bibdef[compl]{Porter2004}
      \bibdef[compl]{Prahalad2005}
      \bibdef[compl]{Scott1998}
      \bibdef[compl]{Slack2002}
      \bibdef[compl]{Thompson2008}
      \bibdef[compl]{Valadares2002}
      \bibdef[compl]{Vasconcellos2001}
%%%%%%
%%%%%%
%
\classdef[Avan.topic]{CCA99005}{2}{TÓPICOS ESPECIAIS EM ENGENHARIA DE CONTROLE E AUTOMAÇÃO I}

     \csummary{Disciplina com tema variado sempre dentro da área de Engenharia de Controle a Automação.}

%%%%%%
%%%%%%
%
\classdef[Avan.topic]{CCA99006}{4}{TÓPICOS ESPECIAIS EM ENGENHARIA DE CONTROLE E AUTOMAÇÃO II}

     \csummary{Disciplina com tema variado sempre dentro da área de Engenharia de Controle e Automação.}

%%%%%%
%%%%%%
%
\classdef[Avan.topic]{CCA99007}{6}{TÓPICOS ESPECIAIS EM ENGENHARIA DE CONTROLE E AUTOMAÇÃO III}

     \csummary{Disciplina com tema variado sempre dentro da área de Engenharia de Controle e Automação.}

%%%%%%
%%%%%%
%
\classdef[Pro.Robotica]{ENG10051}{4}{DINÂMICA E CONTROLE DE ROBÔS}

     \csummary{Modelagem dinâmica de robôs: modelos de Lagrange e Newton-Euter. Controle Independente por juntas. Controle de robôs: por toque calculado, no espaço cartesiano, por realimentação variante no tempo, por realimentação não-suave. Aspectos de implementação.}

      \bibdef{Martinez2013}
      \bibdef{Fu1987}
      \bibdef[basic]{Koubaa2016}
      \bibdef[basic]{OKane2013}
      \bibdef[basic]{Goebel2013}
%%%%%%
%%%%%%
%
\classdef[Pro.Robotica]{ENG10052}{4}{LABORATÓRIO DE ROBÓTICA}

     \csummary{Ambientes de programação e simulação de robôs. Projeto e concepção de células robotizadas. Integração software-hardware do sistema robótico. Protocolos de comunicação como o robô. Normas de segurança.}

      \bibdef{Craig2012}
      \bibdef{Nof1999}
      \bibdef[basic]{Sciavicco2000}
      \bibdef[compl]{Fu1987b}
      \bibdef[compl]{Groover1986}
%%%%%%
%%%%%%
%
\classdef[Pro.Robotica]{ENG10xxa}{4}{ROBÓTICA MÓVEL}

     \csummary{Comportamento não-holonômico, modelagem cinemática e dinâmica de robôs móveis, controle de robôs móveis, localização, mapeamento de ambiente, localização e mapeamento simultâneos, aspectos de implementação, plataformas para robótica móvel.}

      \bibdef{Latombe1991}
      \bibdef{Newman2018}
      \bibdef{Thrun2005}
      \bibdef[basic]{Fu1987c}
      \bibdef[basic]{Murray1994}
      \bibdef[compl]{Brown1983}
      \bibdef[compl]{Everett1995}
%%%%%%
%%%%%%
%
\classdef[Pro.Robotica]{ENG10xxB}{4}{SISTEMAS DE TEMPO REAL}

     \csummary{Caracterização de sistemas tempo-real. Sistemas operacionais tempo-real: métodos de escalonamento. Linguagens de programação para sistemas tempo-real.}

      \bibdef{Burns2001}
      \bibdef[basic]{Tanenbaum2001}
      \bibdef[basic]{Stevens2005}
      \bibdef[compl]{Stroustrup1997}
      \bibdef[compl]{Stroustrup1993}
      \bibdef[compl]{BenAri2006}
%%%%%%
%%%%%%
%
\classdef[Pro.Control]{ELE223}{4}{Sistemas Lineares – A}

     \csummary{Fundamentos de álgebra linear. Sistemas dinâmicos lineares de tempo contínuo e de tempo discreto: definições e propriedades. 
     Representações entrada-saída: equações diferenciais, convolução, função de transferência. Representação por variáveis de estado. Realizações. 
     Análise de sistemas lineares e invariantes no tempo: estabilidade, controlabilidade e observalidade. Realimentação de estados. Observadores de 
     estado.}

      \bibdef[basic]{Chen1999}
      \bibdef[basic]{Lipschutz1978}
      \bibdef[basic]{Strang1988}
      \bibdef[compl]{Zadeh1979}
      \bibdef[compl]{Kailath1980}
      \bibdef[compl]{Haykin2001}
      \bibdef[compl]{Boldrini1984}
      \bibdef[compl]{Steinbruch1987}
%%%%%%
%%%%%%
%
\classdef[Pro.Control]{ELE312}{4}{Sistemas Não-Lineares – B}

     \csummary{Equilíbrios, ciclos-limite e atratores; definições de estabilidade; caracterização de domínios de atração. Método indireto de 
     Liapunov. Método direto de Liapunov em sistemas autônomos: funções de Liapunov, princípio de invariância, estimação de domínios de atração. 
     Estabilidade absoluta: critério do círculo, critério de Popov. Passividade; o lema positivo real.}

      \bibdef[basic]{Khalil1996}
      \bibdef[basic]{Sastry1999}
      \bibdef[compl]{Sepulchre1997}
      \bibdef[compl]{Seydel1994}
      \bibdef[compl]{Slotine1991}
      \bibdef[compl]{Davison1971}
      \bibdef[compl]{Genesio1985}
      \bibdef[compl]{Chiang1987}
      \bibdef[compl]{Bazanella2004}
      \bibdef[compl]{Coutinho2004}
      \bibdef[compl]{Bazanella1997}
      \bibdef[compl]{Bazanella1999}
      \bibdef[compl]{Bazanella2000}
      \bibdef[compl]{GomesDaSilva2002}
      \bibdef[compl]{GomesDaSilva2014}
      \bibdef[compl]{Larios2003}
      \bibdef[compl]{Reginatto2001}
%%%%%%
%%%%%%
%
\classdef[Pro.Control]{ELE216}{4}{ELE216 – Controle Multivariável}

     \csummary{Realimentação de estados. Realimentação estática e dinâmica de saída. Problema de seguimento de referência e rejeição de perturbações. Controle linear quadrático. Introdução ao controle robusto.}

      \bibdef[basic]{Chen1984}
      \bibdef[basic]{Kailath1980b}
      \bibdef[basic]{Maciejowski1990}
      \bibdef[basic]{Kwakernaak1972}
      \bibdef[basic]{Wonham1979}
      \bibdef[basic]{Dorato1995}
      \bibdef[basic]{Zhou1996}
%%%%%%
%%%%%%
%
\classdef[Pro.Control]{ELExxx}{4}{Processos Estocásticos}

     \csummary{to be written} 

%%%%%%%%%%%%%%%%%%%
%%%%%%%%%%%%%%%%%%%
%%%%%
%%%%%  Curricula
%%%%%
%%%%%%%%%%%%%%%%%%%
%%%%%%%%%%%%%%%%%%%

\currdef{CCA-FEO}{FEO CCA}{Formação Essencial Obrigatória - CCA}


%%%%%%%%%%%%%%%%%%%
%%%%%%%%%%%%%%%%%%%
%%%%%
%%%%%  Etapa 01
%%%%%
%%%%%%%%%%%%%%%%%%%
%%%%%%%%%%%%%%%%%%%
\semdef{Etp.01}{Etapa 01}{1}

  \addclass{-2}{INF01202}{ob}
  \addclass{-3}{MAT01353}{ob}
  \addclass{-10}{FIS01181}{ob}
  \addclass{-1}{ARQ03317}{ob}
  \addclass{-8}{CCA99001}{ob}
  \addclass{-5}{QUI01009}{ob}

%%%%%%%%%%%%%%%%%%%
%%%%%%%%%%%%%%%%%%%
%%%%%
%%%%% Etapa 02
%%%%%
%%%%%%%%%%%%%%%%%%%
%%%%%%%%%%%%%%%%%%%
\semdef{Etp.02}{Etapa 02}{2}

  \addclass{-4}{MAT01355}{ob}
    \depdef{MAT01353}
  \addclass{-5}{MAT01354}{ob}
    \depdef{MAT01353}
  \addclass{-1}{ARQ03319}{ob}
    \depdef{ARQ03317}
  \addclass{-10}{FIS01182}{ob}
    \depdef{FIS01181}
  \addclass{-8}{ENG03041}{ob}
    \depdef[A]{MAT01353}
    \depdef{CCA99001}
    \depdef[-A]{FIS01181}
  \addclass{-2}{INF01057}{ob}
    \depdef{INF01202}

%%%%%%%%%%%%%%%%%%%
%%%%%%%%%%%%%%%%%%%
%%%%%
%%%%% Etapa 03
%%%%%
%%%%%%%%%%%%%%%%%%%
%%%%%%%%%%%%%%%%%%%
\semdef{Etp.03}{Etapa 03}{3}

  \addclass{-11}{ENG10001}{ob}
    \depdef{FIS01182}
    \depdef{MAT01354}
  \addclass{-5}{MAT01167}{ob}
    \depdef{MAT01355}
    \depdef{MAT01354}
  \addclass{-10}{FIS01183}{ob}
    \depdef{FIS01182}
  \addclass{-7}{ENG03043}{ob}
    \depdef{QUI01009}
  \addclass{-8}{ENG03042}{ob}
    \depdef{ENG03041}
  \addclass{-3}{MAT02219}{ob}
    \depdef{MAT01353}
  \addclass{-12}{ENG10042}{ob}
    \depdef*{44 cred.ob.}

%%%%%%%%%%%%%%%%%%%
%%%%%%%%%%%%%%%%%%%
%%%%%
%%%%% Etapa 04
%%%%%
%%%%%%%%%%%%%%%%%%%
%%%%%%%%%%%%%%%%%%%
\semdef{Etp.04}{Etapa 04}{4}

  \addclass{-4.4}{MAT01169}{ob}
    \depdef[-30:0]{INF01202}
    \depdef{MAT01167}
  \addclass{-11}{ENG10002}{ob}
   \depdef{ENG10001}
   \depdef{MAT01167}
  \addclass{-12}{ENG10043}{ob}
    \depdef{ENG10042}
  \addclass{-5.4}{MAT01168}{ob}
    \depdef[-A]{MAT01167}
  \addclass{-7}{ENG03092}{ob}
    \depdef[A]{ENG03043}
    \depdef[B]{ENG03041}
  \addclass{-8}{ENG03316}{ob}
    \depdef{ENG03042}
  \addclass{-9}{ENG03044}{ob}
    \depdef{MAT01167}
    \depdef{FIS01183}
    \depdef{ENG10001}

%%%%%%%%%%%%%%%%%%%
%%%%%%%%%%%%%%%%%%%
%%%%%
%%%%% Etapa 05
%%%%%
%%%%%%%%%%%%%%%%%%%
%%%%%%%%%%%%%%%%%%%
\semdef{Etp.05}{Etapa 05}{5}

  \addclass{-12}{ENG10044}{ob}
    \depdef{ENG10002}
  \addclass{-11}{ENG10003}{ob}
    \depdef{ENG10002}
  \addclass{-5}{ENG03004}{ob}
    \depdef{ENG03092}
    \depdef{MAT01169}
  \addclass{-4}{ENG10017}{ob}
    \depdef{MAT01168}
  \addclass{-10}{ENG07086}{ob}
    \depdef{FIS01183}
  \addclass{-6.4}{ENG10026}{alt}
    \depdef{ENG03316}
  \addclass{-9}{ENG03380}{alt}
    \depdef{ENG03316}


%%%%%%%%%%%%%%%%%%%
%%%%%%%%%%%%%%%%%%%
%%%%%
%%%%% Etapa 06
%%%%%
%%%%%%%%%%%%%%%%%%%
%%%%%%%%%%%%%%%%%%%
\semdef{Etp.06}{Etapa 06}{6}

  \addclass{-11}{ENG10047}{ob}
    \depdef{ENG10002}
  \addclass{-10}{ENG10022}{ob}
    \depdef[A]{MAT02219}
    \depdef{ENG10044}
  \addclass{-13}{ENG10045}{ob}
    \depdef[-A]{ENG10044}
    \depdef[-A]{ENG10003}
  \addclass{-8}{ENG07069}{ob}
    \depdef{ENG07086}
  \addclass{-12}{ENG10023}{ob}
    \depdef[B]{INF01057}
    \depdef[-A]{ENG10043}
  \addclass{-4}{ENG10004}{ob}
    \depdef{ENG10017}
    \depdef[30:-70]{ENG03044}

%%%%%%%%%%%%%%%%%%%
%%%%%%%%%%%%%%%%%%%
%%%%%
%%%%% Etapa 07
%%%%%
%%%%%%%%%%%%%%%%%%%
%%%%%%%%%%%%%%%%%%%
\semdef{Etp.07}{Etapa 07}{7}

  \addclass{-13}{ENG10049}{ob}
    \depdef{ENG10047}
    \depdef[-A]{ENG10045}
  \addclass{-2}{ENG10005}{ob}
    \depdef{ENG10004}
  \addclass{-14}{ENG04475}{ob}
    \depdef[-A]{ENG10043}
    \depdef[-A]{ENG10044}
  \addclass{-8}{ENG03021}{ob}
    \depdef[A]{ENG03043}
  \addclass{-12}{ENG10048}{ob}
    \depdef{ENG10023}
  \addclass{-3}{ENG10018}{ob}
    \depdef{ENG10004}
  \addclass{-11}{ENG03027}{ob}
   \depdef{ENG07069}

%%%%%%%%%%%%%%%%%%%
%%%%%%%%%%%%%%%%%%%
%%%%%
%%%%% Etapa 08
%%%%%
%%%%%%%%%%%%%%%%%%%
%%%%%%%%%%%%%%%%%%%
\semdef{Etp.08}{Etapa 08}{8}

  \addclass{-7}{ENG03045}{ob}
    \depdef{ENG03004}
    \depdef[-A]{ENG03021}
  \addclass{-4}{ENG07042}{ob}
    \depdef{ENG07069}
    \depdef[-A]{ENG10018}
  \addclass{-5}{ENG10019}{ob}
    \depdef{ENG10004}
    \depdef[-A]{ENG04475}
  \addclass{-6}{ENG03387}{ob}
    \depdef{ENG03021}
  \addclass[darkred]{-8}{ENG03386}{el}[Obrigatória na Grade Sistemas Discretos CAD/CAM, eletiva nas demais]
    \depdef{ENG03021}
  \addclass[darkgreen]{-1}{ENG03046}{el}[Obrigatória na Grade Controle Avançado de Sistemas Mecânicos, eletiva nas demais]
    \depdef[B]{ENG03044}
    \depdef[A]{ENG10018}
    \depdef{ENG03027}
  \addclass[darkcyan]{-12}{ENG10027}{el}[Obrigatória na Grade Eletrônica de Potência e Acionamento, eletiva nas demais]
    \depdef{ENG10002}
    \depdef[-A]{ENG10044}

%%%%%%%%%%%%%%%%%%%
%%%%%%%%%%%%%%%%%%%
%%%%%
%%%%% Etapa 09
%%%%%
%%%%%%%%%%%%%%%%%%%
%%%%%%%%%%%%%%%%%%%
\semdef{Etp.09}{Etapa 09}{9}

  \addclass{-4}{ENG07087}{ob}
    \depdef{ENG07042}
  \addclass[darkred]{-10}{ENG10021}{el}[Obrigatória na Grade Sistemas Discretos CAD/CAM, eletiva nas demais]
    \depdef{ENG10023}
  \addclass[darkgreen]{-9}{ENG03047}{el}[Obrigatória na Grade Controle Avançado de Sistemas Mecânicos, eletiva nas demais]
    \depdef{ENG03044}
  \addclass[darkcyan]{-13}{ENG10050}{el}[Obrigatória na Grade Eletrônica de Potência e Acionamento, eletiva nas demais]
    \depdef{ENG10049}
  \addclass[darkcyan]{-14}{ENG10046}{el}[Obrigatória na Grade Eletrônica de Potência e Acionamento, eletiva nas demais]
    \depdef{ENG10044}
  \addclass{-8}{TCC/CCA - I}{ob}
    \depdef{ENG03027}
    \depdef{ENG10005}
    \depdef{ENG10048}
    \depdef{ENG10049}

%%%%%%%%%%%%%%%%%%%
%%%%%%%%%%%%%%%%%%%
%%%%%
%%%%% Etapa 10
%%%%%
%%%%%%%%%%%%%%%%%%%
%%%%%%%%%%%%%%%%%%%
\semdef{Etp.10}{Etapa 10}{10}

  %\addclass{-2}{ECO02063}{ob}
  \addclass{-2}{ENG03048}{ob}
    \depdef*{180 cred.ob.}  
  \addclass{-3}{ENG03010}{ob}
    \depdef*{180 cred.ob.}  
  \addclass{-5}{TCC/CCA - II}{ob}
    \depdef[-A]{TCC/CCA - I}
    \depdef[-A]{ENG03045}
    \depdef[-A]{ENG07042}
    \depdef[-A]{ENG10019}
      %\depdef[-A]{ENG03046}
      %\depdef[-A]{ENG03047}
			%\altdep
      %\depdef[-A]{TCC/CCA - I}
      %\depdef[-A]{ENG10027}
      %\depdef[-A]{ENG10050}
      %\depdef[-A]{ENG10046}

%%%%%%%%%%%%%%%%%%%
%%%%%%%%%%%%%%%%%%%
%%%%%
%%%%% Etapa Eletivas
%%%%%
%%%%%%%%%%%%%%%%%%%
%%%%%%%%%%%%%%%%%%%
\semdef{Adicionais}{Adicionais}{11}

  \addclass{-1}{CCA99008}{ad}[Disciplina com 100 Horas CHE]
    \depdef{ENG03316}
    \depdef{ENG03044}
    \depdef{ENG10003}
    \depdef{ENG10042}
    \depdef{ENG07086}
  \addclass{-10}{CCA99009}{ad}[Disciplina com 100 Horas CHE]
    \depdef{CCA99008}
    \depdef[-A]{ENG10004}
    \depdef[-A]{ENG10023}
    \depdef[-A]{ENG04475}
    \depdef[-A]{ENG10022}
    \depdef[-A]{ENG03021}
    \depdef[-A]{ENG07069}
  \addclass{-10}{CCA99010}{ad}[Disciplina com 100 Horas CHE]
    \depdef[A]{CCA99009}
    
    
\semdef{Eletivas}{Eletivas}{12}
  \addclass{-10}{INF1017}{el}
    \depdef{INF01057}
    \depdef*{100 cred.ob.}
  \addclass{-2}{INF01037}{el}
    \depdef{INF01057}
    \depdef*{100 cred.ob.}
  \addclass{-3}{ADM01135}{el}
    \depdef{MAT02219}
  \addclass{-4}{ENG07043}{el}
    %\depdef{ENG07031-}
    \depdef{ENG07069}
    \depdef{ENG10018}
  \addclass{-5}{ENG07012}{el}
    \depdef{ENG07042}
  \addclass{-6}{EDU03071}{el}
  \addclass{-7}{ENG07062}{el}
    \depdef{ENG07042}
  \addclass{-8}{ENG09023}{el}
    \depdef{ENG03021}
  \addclass{-11}{CCA99005}{el}
  \addclass{-12}{CCA99006}{el}
  \addclass{-13}{CCA99007}{el}
  \addclass{-15}{ENG10051}{el}
    \depdef{ENG10004}
    \depdef{ENG10026}
  	\altdep
    \depdef{ENG10004}
    \depdef{ENG03380}
  \addclass{-16}{ENG10052}{el}
    \depdef{ENG10026}
  	\altdep
    \depdef{ENG03380}
  \addclass{-17}{ENG10xxa}{el}
    \depdef{INF01057}
    \depdef{MAT02219}
    \depdef{ENG10017}
  \addclass{-17}{ENG10xxB}{el}
    \depdef{ENG04475}
  \addclass{-18}{ELE223}{el}
    \depdef{ENG10017}
  \addclass{}{ELE312}{el}
    \depdef{ELE223}
  \addclass{}{ELE216}{el}
    \depdef{ENG10018}
	\altdep
    \depdef{ELE223}
  \addclass{}{ELExxx}{el}
    \depdef{ENG10017}
    \depdef{MAT02219}




\tstitle{
  author={Alceu Frigeri\footnote{\tsverb{https://github.com/alceu-frigeri/starray/tree/main/demo}}},
  date={\tsdate},
  title={The stcurrdemo Package\break for starray version \PkgInfo{starray}{version}}
  }
 



\begin{tsabstract}
  As an example of how a package writer could use the \tsobj[pkg]{starray} package, this documents a demo package, \tsobj[pkg]{stcurrdemo}, which defines a set of commands aiming at describing a course curricula.
\end{tsabstract}

\tableofcontents

\section{Introduction}
The purpose of this is to create an example of how to use a \tsobj[pkg]{starray} in a complete setup. That for, this demo has a few parts:
\begin{enumerate*}
  \item A companion package \tsobj[pkg]{stcurrdemo.sty},
  \item this document which documents not just the user level functions/commands, but also the companion package, with  a small example (at the end) of how to use it.
\end{enumerate*}
\begin{tsremark}
  About the version number, since this is ``part'' of \tsobj[pkg]{starray}, and to keep tracking simple, the same version number (from \tsobj[pkg]{starray}) will be used.
\end{tsremark}

\setnewcodekey{stdemo}{codeprefix={},resultprefix={},letter={@,_},
  keywd2={
    starray_new,starray_def_from_keyval,
    starray_new_term,starray_get_unique_id,starray_gset_prop,starray_set_prop,
    starray_get_cnt,starray_term_parser,starray_set_iter,
    starray_set_iter_from_hash,starray_get_prop,starray_gset_from_keyval,
    starray_iterate_over,starray_parsed_get_prop,starray_parsed_get_cnt
    },
  texcs2={
    NewActivity,ActivitySet,ActivitySetCoord,ActivitySetCoordTitle,
    ActivitySelect,Activity,ActivityCoord,
    ActivityCalendarIterate,
    ActivitySetNewEvent,ActivitySetEventDay,
    student,advisor,coadvisor,examiner,reviewer,advisorinfo,coadvisorinfo,examinerinfo,
    examinergrades,studentgrade,studentselect,studentReviewerSelect,
    studentAdvCase,studentCoadvCase,studentReviewerSetCase,studentCase,
    studentiterate,studentadvisoriterate,
    emptytermifnone, emptyfields,studentaddtolist,studentlistsort,
    listemptytermsifnone,studentlistiterate,checklist
    },
  emph={
    __stcurrdemo_set_prof_info,__stcurrdemo_set_prof,__stcurrdemo_emptyterm_if_none,__stcurrdemo_student_emptyfields_if_none,
    __stcurrdemo_seq_sort,__stcurrdemo_emptyfields,
  },
  emph3={starray_term_syntax},
  }

%%%%%%%%%%%%%%%%
%%%%%%%%%%%%%%%%
%%%%%%%%%%%%%%%%
%%%%%%%%%%%%%%%%


\begin{codestore}[datamodel]
nothing
\end{codestore}

\begin{codestore}[datamodel]
\starray_new:n {topics}
\starray_def_from_keyval:nn {topics}
  {
    topichash  = ,
    name = ,
    color = ,
    class . struct = {
      classcode = ,
    }
  }
\end{codestore}

\begin{codestore}[datamodel]
\starray_new:n {classes}
\starray_def_from_keyval:nn {classes}
  {
   cred = ,
   classcode = ,
   name = ,
   summary = ,
   topic = ,
   remark = ,
   uniqueID = , 
   ref . struct = {
     currhash = ,
     semhash = ,
     kind = ,
   } ,
  }
\end{codestore}

\begin{codestore}[datamodel]
\starray_new:n {curricula}
\starray_def_from_keyval:nn {curricula}
  {
    currhash = ,
    name = ,
    text = ,
    sem.struct = {
      pos = ,
      semhash = ,
      name = ,
      class . struct = {
        classcode = ,
        kind = ,
        obs = ,
        pos = ,
        color = ,
        prereqset . struct = {
          prereq . struct = {
            starred = ,
            classcode = ,
            ang = ,
          }
        } ,
      } ,
    } ,
  }
\end{codestore}



\begin{codestore}[topicdef]
\NewDocumentCommand\topicdef{O{}mm}
  {
    \starray_new_term:nn {topics}{#2}
    \starray_set_from_keyval:nn {topics}
      {
        topichash = {#2} ,
        name = {#3} ,
        color = {#1} ,
      }
    \starray_get_unique_id:nNTF {topics}\l__stcurrdemo_uniqueID_tmpa_tl
      {}
      {\tl_set:Nn \l__stcurrdemo_uniqueID_tmpa_tl {}}
    \starray_set_prop:nnV {topics}{uniqueID} \l__stcurrdemo_uniqueID_tmpa_tl
    \seq_new:c {l__stcurrdemo_curr_ \l__stcurrdemo_uniqueID_tmpa_tl _seq}
    
  }
\end{codestore}

\begin{codestore}[topicdef]
\tl_new:N \l__stcurrdemo_default_topic_tl
\NewDocumentCommand\defaulttopic{m}
  { \tl_set:Ne \l__stcurrdemo_default_topic_tl{#1} }
\end{codestore}




\begin{codestore}[classdef]
\NewDocumentCommand{\classdef}{O{\l__stcurrdemo_default_topic_tl}mmm}
  {
    \starray_new_term:nn {classes}{#2}
    \starray_set_from_keyval:nn {classes}
      {
        classcode = {#2} ,
        cred = {#3} ,
        name = {#4} ,
        topichash = {#1} ,
      }
    \starray_get_unique_id:nNTF {classes}\l__stcurrdemo_uniqueID_tmpb_tl
      {}
      {\tl_set:Nn \l__stcurrdemo_uniqueID_tmpb_tl {}}
    \starray_set_prop:nnV {classes}{uniqueID} \l__stcurrdemo_uniqueID_tmpb_tl
    \seq_new:c {l__stcurrdemo_curr_ \l__stcurrdemo_uniqueID_tmpb_tl _bib _seq}
    \seq_new:c {l__stcurrdemo_curr_ \l__stcurrdemo_uniqueID_tmpb_tl _bib basic _seq}
    \seq_new:c {l__stcurrdemo_curr_ \l__stcurrdemo_uniqueID_tmpb_tl _bib compl _seq}
    

    \starray_set_iter_from_hash:nn {topics}{#1}
    \starray_get_prop:nnN {topics}{uniqueID} \l__stcurrdemo_uniqueID_tmpa_tl
    \seq_put_right:cn {l__stcurrdemo_curr_ \l__stcurrdemo_uniqueID_tmpa_tl _seq} {#2}    
  }
\end{codestore}
  
\begin{codestore}[classdef]
\NewDocumentCommand{\csummary}{m}
  {
    \starray_set_prop:nnn {classes}{summary}{#1}
  }
\end{codestore}
  
\begin{codestore}[classdef]
\NewDocumentCommand{\classremark}{m}
  {
    \starray_set_prop:nnn {classes}{remark}{#1}
  }
\end{codestore}
  
\begin{codestore}[classdef]
\NewDocumentCommand{\Orgbibdef}{O{main}m}
  {
    \starray_get_prop:nnN {classes}{uniqueID} \l__stcurrdemo_uniqueID_tmpb_tl
    
    \str_case:nnF {#1}
      {
        {main}
          {
            \seq_gput_right:cn {l__stcurrdemo_curr_ \l__stcurrdemo_uniqueID_tmpb_tl _bib _seq} {#2}
          }
        {basic}
          {
            \seq_gput_right:cn {l__stcurrdemo_curr_ \l__stcurrdemo_uniqueID_tmpb_tl _bib basic _seq} {#2}
          }
        {compl}
          {
            \seq_gput_right:cn {l__stcurrdemo_curr_ \l__stcurrdemo_uniqueID_tmpb_tl _bib compl _seq} {#2}
          }
      }
      {
            \seq_put_right:cn {l__stcurrdemo_curr_ \l__stcurrdemo_uniqueID_tmpb_tl _bib _seq} {#2}
      }
  }
\end{codestore}
  
\begin{codestore}[classdef]
\NewDocumentCommand{\classset}{m}
  {
    \starray_set_iter_from_hash:nn {classes}{#1}
  }
\let\setclass\classset
\end{codestore}



\begin{codestore}[currdef]
\NewDocumentCommand{\currdef}{mmm}
  {
    \starray_new_term:nn {curricula}{#1}
    \starray_set_from_keyval:nn {curricula}
      {
        currhash = {#1} ,
        name = {#2} ,
        text = {#3} ,
      }
  }
\end{codestore}

\begin{codestore}[currdef]
\NewDocumentCommand{\semdef}{mmm}
  {
    \starray_new_term:nn {curricula.sem}{#1}
    \starray_set_from_keyval:nn {curricula.sem}
      {
        semhash       = {#1} ,
        name       = {#2} ,
        pos        = {#3} ,
      }
  }
\end{codestore}

\begin{codestore}[currdef]
\NewDocumentCommand{\addclass}{O{}D<>{}mmO{}}
  {
    \starray_new_term:nn {curricula.sem.class}{#3}
    \starray_set_from_keyval:nn {curricula.sem.class}
      {
        color = {#1} ,
        pos   = {#2} ,
        classcode  = {#3} ,
        kind  = {#4} ,
        obs   = {#5} ,
      }
    \starray_new_term:nn {curricula.sem.class.prereqset}{}
    \starray_set_iter_from_hash:nn {classes}{#3}

    \starray_new_term:nn {classes.ref}{}
    \starray_get_prop:nnN {curricula}{currhash} \l_tmpa_tl
    \starray_set_prop:nnV {classes.ref}{currhash} \l_tmpa_tl
    \starray_get_prop:nnN {curricula.sem}{semhash} \l_tmpa_tl
    \starray_set_prop:nnV {classes.ref}{semhash} \l_tmpa_tl
    \starray_set_prop:nnn {classes.ref}{kind}{#4}
  }
\end{codestore}

\begin{codestore}[currdef]
\NewDocumentCommand{\depdef}{sD<>{}m}
  {
    \starray_new_term:nn {curricula.sem.class.prereqset.prereq}{#3}
    \starray_set_from_keyval:nn {curricula.sem.class.prereqset.prereq}
      {
        starred = {#1} ,
        name = {#3} ,
        ang  = {#2}
      }
  }
\end{codestore}

\begin{codestore}[currdef]
\NewDocumentCommand{\altdep}{}
  {
    \starray_new_term:nn {curricula.sem.class.prereqset}{}
  }
\end{codestore}

%%%%%%%%%%%%%%%%
%%%%%%%%%%%%%%%%
%%%%%%%%%%%%%%%%
%%%%%%%%%%%%%%%%


\section{Data Model}\label{DataModel}
In the following example, a ``curricula'' is defined as a set of ``terms'', each ``term'' composed of a set of ``classes'', each ``class'' having it's own set of pre-requisites. Moreover, each ``class'' belongs to a given ``topic''. To lessen redundancy, topics, classes and curricula are separated structures, pointing to each other as needed.

\begin{tsremark}
As in any ``procedural language'', it is advisable to  carefully design the data model, since this will shape the functions which will set and use said data.
\end{tsremark}

\begin{tsremark}
  Pay attention to the use of the tildes,  \~{} , since those definitions will be made, most likely, in an \tsobj[pkg]{expl3} code régime, remember that spaces are ignored, therefore, if needed, use a tilde instead of a space.
\end{tsremark}

\begin{enumerate}
  \item ``topics'' with associated:
\begin{enumerate}
  \item ID (hash), name, color
\end{enumerate}
\tscode*[stdemo]{datamodel}[2]
  
  \item ``classes'' with associated:
\begin{enumerate}
  \item name, acronym (hash)
  \item summary, kind/type
  \item related topic (see below)
  \item list of curricula (see below) it's part of
\end{enumerate}
\tscode*[stdemo]{datamodel}[3]

  \item ``curricula'' with associated:
\begin{enumerate}
  \item name, (short) description
  \item list of semesters each with
  \begin{enumerate}
    \item name, position
    \item list of classes, each with a list of requisites (ref. to other classes)
  \end{enumerate}
\end{enumerate}
\tscode*[stdemo]{datamodel}[4]
  
\end{enumerate}



\subsection{Topic Set}\label{topic}
Just a pair of commands, to define topics and select one as default.
\begin{codedescribe}{\topicdef,\defaulttopic}
\begin{codesyntax}
  \tsmacro{\topicdef}[color]{topic-ID,name}
  \tsmacro{\defaulttopic}{topic-ID}
\end{codesyntax}
  \tsobj{\topicdef} defines a given topic. \tsobj[marg]{name} is it long name. \tsobj[marg]{topic-ID} is just an identifier to reference it (it is used as the \tsobj[pkg]{starray} hash of that topic). \tsobj[oarg]{color} associates a color with it.
  
  \tsobj{\defaulttopic} sets the default topic hash (stored in a token list variable).
\end{codedescribe}

\tscode*[stdemo]{topicdef}[1]

\tscode*[stdemo]{topicdef}[2]


\subsection{Class Set}\label{classdef}

%%\tscode*[stcurrdemo]{activity-def}


\tscode*[stdemo]{classdef}[1]

\tscode*[stdemo]{classdef}[2]

\tscode*[stdemo]{classdef}[3]

\tscode*[stdemo]{classdef}[4]

\tscode*[stdemo]{classdef}[5]

\subsection{Curricula Set}\label{curriculadef}


\tscode*[stdemo]{currdef}[1]

\tscode*[stdemo]{currdef}[2]

\tscode*[stdemo]{currdef}[3]

\tscode*[stdemo]{currdef}[4]

\tscode*[stdemo]{currdef}[5]

Whereas, the ``coord'' sub-structure is for the activity's coordinator, whilst ``calendar'' shall (for instance) contains a list of calendar events, and, finally, the many ``chk* '' will be used for a ``check list''.

\begin{tsremark}
The ``chkID'' (and checklists). In many cases it's handy to have an unique identifier for a given structure. That can be obtained with \tsobj{\starray_get_unique_ID:nN}, and to avoid calling this function time and time again, one can just store that ID as a field for later use.
\end{tsremark}
\begin{tsremark}
  Could the Coordinator's name and title be a direct property (avoiding the ``coord'' sub-structure)? of course, that's a matter of choice on how to model it.
\end{tsremark}





\subsection{Student Set}\label{student.DataModel}
A student's structure might contain, besides student's name, work title, some flags, an advisor (and co-advisor, if needed), reviewer's list (with a provision for reviewer's grade, if needed).

Of course, one doesn't need to define a \tsobj[pkg]{starray} structure using \tsobj{\starray_def_from_keyval:nn}, but, as in this,  if the set of properties is known, it always makes for a cleaner definition.

\begin{tsremark}
  The fields/properties defaults can be anything, including usual \LaTeXe\  commands, like a \tsobj{\rule} which is handy, for instance, when generating forms, e.g., if the fields are all set, a form can be created with the proper values, otherwise, it will be  created with ``rules'' in place (no need to test if the properties were set).
\end{tsremark}

%\tscode*[stcurrdemo]{student-def}








\section{Auxiliary Functions}
Once the data layout is set (see \ref{DataModel}) the next step is to define a set of (document level) functions, so the data can be initialized and used by the end user.

\subsection{Generic Recovery Functions}\label{generic:datafield}

\begin{codedescribe}{\DataField,\DataGet}
  \begin{codesyntax}
    \tsmacro{\DataField}{starray,item}
    \tsmacro{\DataGet}{starray,item,store-var}
  \end{codesyntax}
\tsobj{\DataField} will recovery an item from any \tsobj[pkg]{starray}, for instance, \tsobj[marg]{starray} might be \tsobj[meta,sep=or]{activity,activity.coord,activity.calendar,student,student.advisor} etc. whilst \tsobj[marg]{item} might be any corresponding field. The \tsobj{\DataGet} will store said value in an auxiliary \tsobj[marg]{store-var}.
\begin{tsremark}[\color{red}Note:]
  None of those commands are expandable. In general it should be enough (for the end user) to just use \tsobj{\DataField}, but it might be necessary to use a temporary variable, \tsobj[marg]{store-var}, allowing  its use in an expandable context.
\end{tsremark}
\end{codedescribe}


%\tscode*[stcurrdemo]{DataRecovery}[3]
%\tscode*[stcurrdemo]{DataRecovery}[4]

\subsection{Activity's Functions}

One could define a single function to initialize all fields (using a key=val interface), but, in a more traditional approach  one can set two functions to start the initialization process \tsobj{\NewActivity,\ActivitySet}. 


\subsubsection{Creating and Setting an Activity's Data}

\begin{codedescribe}{\NewActivity}
\begin{codesyntax}
  \tsmacro{\NewActiviy}{act-ID,acronym,name}
\end{codesyntax}
\tsobj{\NewActivity} will create a new activity term, \tsobj[marg]{act-ID} will be it's identifier (hash).

\end{codedescribe}
\begin{tsremark}
  Every time a \tsobj[pkg]{starray} is instantiated, up to two hashes are created: a numerical one (starting at one) and an ``user defined one''. In the \tsobj{\NewActivity} function above, \tsobj[marg]{act-ID} is that hash, so that instance can be later referenced by it. Of course, it must be an unique ID/hash.
\end{tsremark}
\begin{tsremark}
  One thing to be noticed about \tsobj[pkg]{starray}: every structure has an associated internal index (iterator). When you create a new instance, this iterator always points to the newly created one, therefore, sparing the use of an explicit index in the subsequent commands.
\end{tsremark}

%\tscode*[stcurrdemo]{ActCmd.New}[1]
%%\tscode*[stcurrdemo]{ActCmd.New}[2]

here

~

Similarly, one can define some functions to set the activity's coordinator. Of course, it's up to the package programmer to choose if one, two (or more) functions for this.

\begin{codedescribe}{\ActivitySetCoord}
\begin{codesyntax}
  \tsmacro{\ActivitySetCoord}[act-ID]{name,title}
\end{codesyntax}
The optional argument \tsobj[oarg]{act-ID} should refer to an already create activity, and, if not given, will use the current one.
\end{codedescribe}

%\tscode*[stcurrdemo]{ActCmd.SetCoord}[1]
%%\tscode*[stcurrdemo]{ActCmd.SetCoord}[2]


And the associated ``Calendar Events'', assuming there will be a fixed set of events (each semester/year), leaving the date to be set later on. 

\begin{codedescribe}{\ActivitySetNewEvent,\ActivitySetEventDay}
\begin{codesyntax}
  \tsmacro{\ActivitySetNewEvent}[act-ID]{event-ID,description}
  \tsmacro{\ActivitySetEventDay}[act-ID]{event-ID,date,week}
\end{codesyntax}
The optional argument \tsobj[oarg]{act-ID} refers to an already create activity, and, if not given, the current one will used. \tsobj[marg]{event-ID} can be any identifier. That way, the user can first define a set of events, and later on, set the associated dates.
\end{codedescribe}


%\tscode*[stcurrdemo]{ActCmd.SetEvent}[1]



%\tscode*[stcurrdemo]{ActCmd.SetEvent}[2]



\subsubsection{Check Lists}\label{Activity-checklist}

It's often desirable to have a ``check list''. What such list could entice is always up to debate, the idea behind the few next functions is to allow the end user to define which items such a list (as a matrix) might have.

\begin{codedescribe}{\checkdef}
\begin{codesyntax}
  \tsmacro{\checkdef}{chkID,chkPos,chktext}
\end{codesyntax}
  \tsobj[marg]{chkID} is just an ID to reference the check list item. \tsobj[marg]{chkPos} will relate the item to a position in a matrix (tabular environment, see \tsobj{\StudentCheckListTable}) and \tsobj[marg]{chktext} is the (assumed) short text.
  The command \tsobj{\checkdef} defines/create a new check item.
\end{codedescribe}
\begin{tsremark}
  In the implementation below, three property lists are created (based on the activity unique ID). Two of them with the \tsobj[marg]{chktext} marked and unmarked, and a third relating the \tsobj[marg]{chkID} a given position \tsobj[marg]{chkPos}.
\end{tsremark}



%\tscode*[stcurrdemo]{ActCmd.CheckList}[1]

\begin{codedescribe}{\checklist}
\begin{codesyntax}
  \tsmacro{\checklist}[act-ID]{chkID-list}
\end{codesyntax}
  This sets a list of \tsobj[marg]{chkID}s associated with the current student. Note that, since the checklist is based on an activity, the optional parameter allows to explicitly select one.
\end{codedescribe}
\begin{tsremark}
  In the implementation below, the property list, associated with the student unique ID, will use as keys the \tsobj[marg]{chkPost} from \tsobj{\checkdef} and as associated value the ones from the activity's marked property list.
\end{tsremark}

%\tscode*[stcurrdemo]{ActCmd.CheckList}[2]


\begin{codedescribe}{\StudentCheckListTable}
\begin{codesyntax}
  \tsmacro{\StudentCheckListTable}{L-list,C-list}
\end{codesyntax}
  This will produces a checklist matrix. Both \tsobj[marg]{L-list,C-list} shall be a comma list of ``lines'' and ``columns''. More specifically, each \tsobj[marg]{L-list} element will be compose with each \tsobj[marg]{C-list} element like ``$L_iC_j$'' which shall correspond to one of the \tsobj[marg]{chkPos} elements defined with \tsobj{\checkdef}.
\end{codedescribe}
\begin{tsremark}
  Better said, this will produces the inner part of a table, sans the table begin/end. Also note, in the code example below, that each table line is finished with a \tsobj[verb]{\\*}.
\end{tsremark}

%\tscode*[stcurrdemo]{ActCmd.CheckList}[3]



%\tsmergedcode*[stcurrdemo]{{ActCmd.CheckList}[1-2]}

~

\subsubsection{Selecting an Activity}

\begin{codedescribe}{\ActivitySelect}
  \begin{codesyntax}
    \tsmacro{\ActivitySelect}{act-ID}
  \end{codesyntax}
  This will just select an activity, identified by \tsobj[marg]{act-ID} as the current one. So that, in the following commands, one can avoid the first, optional, argument.
\end{codedescribe}


%\tscode*[stcurrdemo]{ActCmd.Select}



\subsubsection{Iterating over the Calendar Data}

\begin{codedescribe}{\ActivityCalendarIterate}
  \begin{codesyntax}
    \tsmacro{\ActivityCalendarIterate}{code}
  \end{codesyntax}
  This is a helper function, based on \tsobj{\starray_iterate_over:nn}, so that the end user is free to construct an ``Event Calendar'' with the (activity's) stored data. The suggested pattern is: 
  \begin{enumerate*}
    \item Select an activity with \tsobj{\ActivitySelect}, then
    \item execute the code for each item stored in the activity's calendar list.
  \end{enumerate*}. The user is supposed to use (in \tsobj[marg]{code}) \tsobj[code,sep=or]{\DataField,\DataGet} to retrieve and use the calendar's data.
\end{codedescribe}

%\tscode*[stcurrdemo]{ActCmd.CalIterate}


\subsection{Student's Functions}

\subsubsection{Creating and Setting Student's Data}\label{student.new}

\begin{codedescribe}{\student,\studentremark,\worktitle}
  \begin{codesyntax}
    \tsmacro{\student}[student-hash]{last,first,ID,email}
    \tsmacro{\studentremark}{remark}
    \tsmacro{\worktitle}{work-title}
  \end{codesyntax}
  As always, there are many ways to achieve this. The \tsobj{\student} ``creates'' a new student entry, the (not so) optional parameter \tsobj[oarg]{student-hash} associates it with the given hash (otherwise one would have to keep track 'which index' correspond to a student). An auxiliary property list, for a checklist (see \ref{Activity-checklist} and \tsobj{\StudentCheckListTable} ), is created using  the student unique identifier, \tsobj{\starray_get_unique_id}.
\end{codedescribe}

%\tscode*[stcurrdemo]{student.New}[1]
%\tscode*[stcurrdemo]{student.New}[2]
%\tscode*[stcurrdemo]{student.New}[3]

\begin{codedescribe}{\advisor,\coadvisor,\examiner}
  \begin{codesyntax}
    \tsmacro{\advisor}[pre-nom]{last,first}
    \tsmacro{\coadvisor}[pre-nom]{last,first}
    \tsmacro{\examiner}[pre-nom]{last,first}
  \end{codesyntax}
  Those are just some auxiliary commands to set the advisor's/coadvisor's/examiner's name. The first optional parameter \tsobj[oarg]{pre-nom} can be used as a title. With each call of those, a new \tsobj[pkg]{starry} term is created for the current student.
\end{codedescribe}

%\tscode*[stcurrdemo]{student.NewAdv}[1]
%\tscode*[stcurrdemo]{student.NewAdv}[2]
%\tscode*[stcurrdemo]{student.NewAdv}[3]
%\tscode*[stcurrdemo]{student.NewAdv}[4]

\begin{codedescribe}{\advisorinfo,\coadvisorinfo,\examinerinfo}
  \begin{codesyntax}
    \tsmacro{\advisorinfo}{institute,title,email}
    \tsmacro{\coadvisorinfo}{institute,title,email}
    \tsmacro{\examinerinfo}{institute,title,email}
  \end{codesyntax}
  Some extra commands to set the advisor's/coadvisor's/examiner's other data. Note, though, it is assumed that these commands will be called after the respective \tsobj[code,sep=or]{\advisor,\coadvisor,\examiner} command call.
\end{codedescribe}

%\tscode*[stcurrdemo]{student.NewAdv}[5]
%\tscode*[stcurrdemo]{student.NewAdv}[6]
%\tscode*[stcurrdemo]{student.NewAdv}[7]
%\tscode*[stcurrdemo]{student.NewAdv}[8]


\begin{codedescribe}{\examinergrades}
\begin{codesyntax}
  \tsmacro{\examinergrades}[case]{gradeA,gradeB,gradeC,gradeD}
\end{codesyntax}  
Nothing much to be said, this allows to set 4 grades, per examiner. \tsobj[oarg]{case} sets how the average shall be calculated. Note that, this is supposed to be used immediately  after the respective command \tsobj{\examiner}, better said, before another \tsobj{\examiner}, or after having selected a specific reviewer with \tsobj{\studentreviewerselect} (see \ref{student.data-recovery}).
\begin{tsremark}
  The \tsobj{\starray_gset_prop:nne} is used to assure the stored grade will be a floating point, and not an expression to be evaluated later on.
\end{tsremark}
\end{codedescribe}

%\tscode*[stcurrdemo]{student.grades}[1]



\begin{codedescribe}{\studentgrade}
\begin{codesyntax}
  \tsmacro{\studentgrade}[case]{min}
\end{codesyntax}  
Similarly, this will set/calculate the student's final grade based on the individual reviewers' grades, and the flag \tsobj[key]{flag-approved} (based on \tsobj[marg]{min}).  The optional parameter \tsobj[oarg]{case} selects how the final grade will be calculated.
\begin{tsremark}
  The \tsobj{\starray_gset_prop:nne} is used to assure the stored grade will be a floating point, and not an expression to be evaluated later on.
\end{tsremark}
\begin{tsremark}
  To simplify the logic, it's assumed there are 3 reviewers (even though in some cases only two are needed). Therefore the command \tsobj{\emptytermifnone} (see \ref{student.auxcmds});
\end{tsremark}
\end{codedescribe}

%\tscode*[stcurrdemo]{student.grades}[2]


\subsubsection{Selecting Student's Data}\label{student.data-recovery}

\begin{codedescribe}{\studentselect,\studentReviewerSelect}
\begin{codesyntax}
  \tsmacro{\studentselect}{student-hash}
  \tsmacro{\studentReviewerSelect}{rev-index}
\end{codesyntax}
 \tsobj{\studentselect} allows to select a given student, given it's hash (note, the student index could also be used). Since the \tsobj{\examiner} command (see \ref{student.new}) doesn't associate a hash with each examiner, one can only select one with it's index number: 1, 2 ...
\end{codedescribe}

%\tscode*[stcurrdemo]{student.select}[1]
%\tscode*[stcurrdemo]{student.select}[2]

\begin{codedescribe}{\studentcase,\studentadvcase,\studentcoadvcase,\studentreviewersetcase}
\begin{codesyntax}
  \tsmacro{\studentCase}{if-approved,if-not}
  \tsmacro{\studentAdvCase}{if-one,if-many}
  \tsmacro{\studentCoadvCase}{if-set,if-not}
  \tsmacro{\studentReviewerSetCase}{if-set,if-not}  
\end{codesyntax}
 Those are auxiliary conditionals. \tsobj{\studentAdvCase} tests if the student has one or more than one advisors assigned. \tsobj{\studentCoadvCase} tests if there is a co-advisor. \tsobj{\studentReviewerSetCase} tests if the current reviewer (better said, student.reviewer) is set (not the default -empty- set). And, finally, \tsobj{\studentCase} verifies the state of the flag flag-approved.
\end{codedescribe}
\begin{tsremark}
  In the code below, one could have used \tsobj{\starray_get_prop:} instead. The construct is meant as an example of how to use one of the few \tsobj{\starray_parsed_} commands, which can be used in an expandable context.
\end{tsremark}

%\tscode*[stcurrdemo]{student.cases}[1]
%\tscode*[stcurrdemo]{student.cases}[2]
%\tscode*[stcurrdemo]{student.cases}[3]
%\tscode*[stcurrdemo]{student.cases}[4]


\subsubsection{Iterating over Students}

\begin{codedescribe}{\studentiterate,\studentadvisoriterate}
  \begin{codesyntax}
    \tsmacro{\studentiterate}{code}
    \tsmacro{\studentadvisoriterate}{code}
  \end{codesyntax}
  As their name implies, they are auxiliary commands to iterate over all students, \tsobj{\studentiterate}, or all advisors of the current student, \tsobj{\studentadvisoriterate}.
\end{codedescribe}
\begin{tsremark}
  To retrieve the student's/advisor's data, the end user is supposed to use \tsobj[code,sep=or]{\DataField,\DataGet} (see \ref{generic:datafield}), like \tsverb[verb]{\DataField{student}{name}} (do not use any index/hash) or \tsobj[verb]{\DataField{student.advisor}{name}}.
\end{tsremark}
%\tscode*[stcurrdemo]{student.iter}[1]
%\tscode*[stcurrdemo]{student.iter}[2]


\subsubsection{Auxiliary Commands}\label{student.auxcmds}
\begin{codedescribe}{\emptytermifnone,\emptyfields}
  \begin{codesyntax}
    \tsmacro{\emptytermifnone}[count]{struct}\tsargs[oarg]{code}
    \tsmacro{\emptyfields}{}
  \end{codesyntax}
  \tsobj{\emptyfields} creates an ``empty'' student (better said, with a hash: ``empty'') all fields remain at their default value (see \ref{student.DataModel}). \tsobj{\emptytermifnone} assures that there is(are) at least \tsobj[oarg]{count} (defaults to 1) (sub)structures \tsobj[marg]{struct}. Optionally, \tsobj[oarg]{code} will be executer after each instantiation (if needed) of a term defined by \tsobj[marg]{struct}.
\end{codedescribe}

%\tscode*[stcurrdemo]{student.emptyterm}[1]
%\tscode*[stcurrdemo]{student.emptyterm}[2]



\subsubsection{Student's Lists}
\begin{codedescribe}{\studentaddtolist,\studentlistsort}
\begin{codesyntax}
  \tsmacro{\studentaddtolist}{list}
  \tsmacro{\studentlistsort}[field]{list}
\end{codesyntax}
A \tsobj[pkg]{starray} has an implicit order: it's instantiation sequence, which is ok, but not always. To be able the access/list the terms of it, an option is to have one (or more) associated lists. Note that those commands will just create a sequence associated with \tsobj[marg]{list}

\tsobj{\studentaddtolist} will insert a student (better said, the current student's hash) to a \tsobj[marg]{list}, if the \tsobj[marg]{list} isn't already defined, a new one will be created. \tsobj{\studentlistsort} will sort the elements of \tsobj[marg]{list} based on the value of the field \tsobj[oarg]{field} (defaults to ``name'').
\end{codedescribe}
\begin{tsremark}
  Since \tsobj[marg]{list} will store student's hash, the sort has to first retrieve the associated \tsobj[pkg]{starray} term and the ``sorting field'' which can be, in fact, anything associated with the student. In the example below, only the immediate fields can be used, like name, email, worktitle, but one can easily modify the code below to retried, for instance, the advisor's name.
\end{tsremark}

%\tscode*[stcurrdemo]{student.list}[1]

%\tscode*[stcurrdemo]{student.list}[2]

\begin{codedescribe}{\listemptytermsifnone}
\begin{codesyntax}
  \tsmacro{\listemptytermsifnone}{list}
\end{codesyntax}
This will iterate over all students in \tsobj[marg]{list} assuring that each student has at leas 1 advisor and 3 examiners (reviewers). Note that, in the case of the examiners the \tsobj{\examiner} command (see \ref{student.new}) sets the ``flag-set'' associated with said examiner to true. This, however, keeps the it's default value: false.
\end{codedescribe}

%\tscode*[stcurrdemo]{student.list}[3]


\begin{codedescribe}{\studentlistiterate}
\begin{codesyntax}
  \tsmacro{\studentlistiterate}{list,code}
\end{codesyntax}
As the name implies, it will iterate over all students in \tsobj[marg]{list} executing \tsobj[marg]{code} for each of them. Note that, before starting the iteration, \tsobj{\studentlistiterate} verifies if the list is already sorted (if not, \tsobj{\studentlistsort} will be called, with it's default) and it will make sure all students have an advisor term (and examiners) by calling \tsobj{\listemptytermsifnone}.
\end{codedescribe}

%\tscode*[stcurrdemo]{student.list}[4]

\newpage
\section{Example of Use}
\subsection{Setting Things up}

\begin{codestore}[demo.datasetting]
  \NewActivity{FW I}{fw0501}{Final Work I}
  \ActivitySetCoord{Prof. Willian S.}{Final Work Coordinator}

  \NewActivity{FW II}{fw0502}{Final Work II}
  \ActivitySetCoord[FW II]{Prof. Karen S.}{Final Work Coordinator}
  
  \NewActivity{TN A}{tn0101}{Trainees}
  \ActivitySetCoord{Prof. Samantha S.}{Trainee Program Coordinator}
\end{codestore}

%\tscode*[stcurrdemo]{demo.datasetting}

%%\tsexec{demo.datasetting}


\subsection{Defining per Activity Check List}


\begin{codestore}[activity.checklist]
\ActivitySelect{FW I}

\checkdef{L1C1}{docs}{Documentation OK}
\checkdef{L2C1}{prop}{Proposal OK}
\checkdef{L3C1}{advisor}{Advisor assig.}

\checkdef{L1C2}{middle}{middle term}
\checkdef{L2C2}{examiners}{Examiners assig.}

\checkdef{L1C3}{final}{Final Text}
\checkdef{L2C3}{tutorok}{Tutor approval}

\checkdef{L1C4}{text}{text approved}
\checkdef{L2C4}{graded}{examiners grade}

\checkdef{L3C5}{library}{Text Catalogued}
\end{codestore}


\begin{codestore}[activity.checklist]
\ActivitySelect{FW II}

\checkdef{L1C1}{docs}{Documentation OK}
\checkdef{L2C1}{prop}{Proposal OK}
\checkdef{L3C1}{advisor}{Advisor assig.}

\checkdef{L1C2}{middle}{middle term}
\checkdef{L2C2}{examiners}{Examiners assig.}

\checkdef{L1C3}{final}{Final Text}
\checkdef{L2C3}{tutorok}{Tutor approval}

\checkdef{L1C4}{text}{text approved}
\checkdef{L2C4}{graded}{examiners grade}

\checkdef{L3C5}{library}{Text Catalogued}
\end{codestore}


\begin{codestore}[activity.checklist]
\ActivitySelect{TN A}
\checkdef{L1C1}{docs}{Documentation OK}

\checkdef{L2C1}{tutor}{tutor assigned}

\checkdef{L1C2}{middle}{middle term report}

\checkdef{L1C3}{final}{Final Report}
\checkdef{L2C3}{tutorok}{tutor approval}

\checkdef{L1C4}{text}{text approved}

\checkdef{L3C5}{library}{Report Catalogued}
\end{codestore}

%\tscode*[stcurrdemo]{activity.checklist}[1]
%\tscode*[stcurrdemo]{activity.checklist}[2]
%\tscode*[stcurrdemo]{activity.checklist}[3]
%%\tsexec{activity.checklist}[1]
%%\tsexec{activity.checklist}[2]
%%\tsexec{activity.checklist}[3]

\subsection{Defining per Activity Calendar}
Setting a set of events related to an activity, and then the relevant dates.


\begin{codestore}[activity.calendar]
\ActivitySelect{FW II}

%%%%%%%%
\ActivitySetNewEvent{opening}
    {First class. Activity goals.}

\ActivitySetNewEvent{proposals}
    {Deadline for proposals submission.}

\ActivitySetNewEvent{middle term}
    {Students middle term presentation and follow-up.  }

\ActivitySetNewEvent{submission}
    {Deadline for student's work submission.}

\ActivitySetNewEvent{review}
    {Work review by examiners.}

\ActivitySetNewEvent{feedback}
    {Feedback on the work done, per student (private)}

\ActivitySetNewEvent{exam}
    {Final Exam}
\end{codestore}

\begin{codestore}[activity.calendar]
\ActivitySelect{FW II}

%%%%%%%%
\ActivitySetEventDay {opening}{01/09}{1st week}
\ActivitySetEventDay {proposals}{15/09}{3rd week}
\ActivitySetEventDay {middle term}{07/10}{6th week}
\ActivitySetEventDay {submission}{21/10}{8th week}
\ActivitySetEventDay {review}{28/10-03/11}{9th week}
\ActivitySetEventDay {feedback}{06-10/11}{10th week}
\ActivitySetEventDay {exam}{15/11}{last week}
\end{codestore}


%\tscode*[stcurrdemo]{activity.calendar}[1]

Once the events are defined, the associated dates can be set, later on, at any time. There is no need, per-se, to set the dates right away.

%\tscode*[stcurrdemo]{activity.calendar}[2]
%%\tsexec{activity.calendar}[1]
%%\tsexec{activity.calendar}[2]


\subsection{Constructing an Activity Calendar}
Once all set, it is just a matter of using the \tsobj{\ActivityCalendarIterate} to go over all the events. Note that, the sequence is exactly the original one (defined by \tsobj{\ActivitySetNewEvent}).

\begin{codestore}[activity.calendar]
\ActivitySelect{FW II}
\begin{center}
  \sc Calendar for \DataField{activity}{name}
\end{center}
  \begin{tabular}{lcl}
    \ActivityCalendarIterate{ \DataField{activity.calendar}{date} & \DataField{activity.calendar}{week} & \DataField{activity.calendar}{event} \\ }
  \end{tabular} 
\end{codestore}

%%\tsdemo*[stcurrdemo]{activity.calendar}[3]

\subsection{Students}

\begin{codestore}[demo.student.data]
  \student[James S.]{Smith}{James}{ID001}{smith.james@uni.gov}
  \studentremark{2nd time}
  \worktitle{Some Useful System}

  \advisor{T.}{Jonathan}
  \advisorinfo{University of Z}{Prof.}{jon.t@uni.gov}

  \examiner{T.}{William}
  \examinergrades{10}{9}{8}{7}
  \examinerinfo{University of Z}{Prof.}{william.t@uni.gov}

  \examiner{T.}{Jame}
  \examinerinfo{University of Z}{Prof.}{jame@uni.gov}
  \examinergrades{10}{9}{8}{7} 

  \examiner{T.}{Thomaz}
  \examinerinfo{University of Z}{Prof.}{thomaz.t@uni.gov}
  \examinergrades{10}{9}{8}{7}
  
  \ActivitySelect{FW I}
  \checklist{docs,prop,advisor,middle,examiners,text,graded}
  \studentaddtolist{FW-I}
  
\end{codestore}

\begin{codestore}[demo.student.data]
  \student[Sarah S.]{Barnes}{James}{ID003}{sarah.james@uni.gov}
  \worktitle{Some Useful System}

  \advisor{T.}{Jonathan}
  \advisorinfo{University of Z}{Prof.}{jon.t@uni.gov}

  \examiner{T.}{William}
  \examinergrades{10}{9}{8}{7}
  \examinerinfo{University of Z}{Prof.}{william.t@uni.gov}

  \examiner{T.}{Jame}
  \examinerinfo{University of Z}{Prof.}{jame@uni.gov}
  \examinergrades{10}{9}{8}{7} 

  \examiner{T.}{Thomaz}
  \examinerinfo{University of Z}{Prof.}{thomaz.t@uni.gov}
  \examinergrades{10}{9}{8}{7}
  
  \ActivitySelect{FW II}
  \checklist{docs,prop,advisor,middle,examiners,text,graded,library}
  \studentaddtolist{FW-II}

\end{codestore}


\begin{codestore}[demo.student.data]
  \student[Barney]{Smith}{Barney}{ID002}{loren.s@uni.gov}
  \worktitle{Some Useful System}

  \advisor{T.}{J.J.}
  \advisorinfo{University of Z}{Prof.}{jon.t@uni.gov}

  \examiner{T.}{Ceasare}
  \examinergrades{10}{9}{8}{7}
  \examinerinfo{University of Z}{Prof.}{william.t@uni.gov}

  \examiner{T.}{Marcus}
  \examinerinfo{University of Z}{Prof.}{jame@uni.gov}
  \examinergrades{10}{9}{8}{7} 

  \examiner{T.}{Brutus}
  \examinerinfo{University of Z}{Prof.}{thomaz.t@uni.gov}
  \examinergrades{10}{9}{8}{7}

  \ActivitySelect{FW II}
  \checklist{docs,prop}
  \studentaddtolist{FW-II}

  \student[Kate]{Smithson}{Kate}{ID004}{loren.s@uni.gov}
  \worktitle{Some Useful System}

  \advisor{T.}{Jonathan}
  \advisorinfo{University of Z}{Prof.}{jon.t@uni.gov}

  \advisor{T.}{William}
  \advisorinfo{University of Z}{Prof.}{jon.t@uni.gov}

  \examiner{Aurelius}{T.}
  \examinergrades{10}{9}{8}{7}
  \examinerinfo{Uni. Z}{Prof.}{aurelius@uni.gov}

  \examiner{T.}{Cicero}
  \examinerinfo{Uni. Z}{Prof.}{cicero@uni.gov}
  \examinergrades{10}{9}{8}{7} 

%  \examiner{T.}{Plutarco}
%  \examinerinfo{Uni. Z}{Prof.}{plutarco.t@uni.gov}
%  \examinergrades{10}{9}{8}{7}

  \ActivitySelect{FW II}
  \checklist{docs,prop}
  \studentaddtolist{FW-II}

\end{codestore}
Setting student's data (some others, not shown, likely defined).


%\tscode*[stcurrdemo]{demo.student.data}[1]
%\tscode*[stcurrdemo]{demo.student.data}[2]
%%\tsexec{demo.student.data}[1]
%%\tsexec{demo.student.data}[2]
%%\tsexec{demo.student.data}[3]


\subsection{Accessing a single student data}
Just an example of how to select/access the data of a single student, and it's corresponding check list (associated with a given activity).

\begin{codestore}[demo.student.single-use]
\ActivitySelect{FW I}
\studentselect{James S.}
Student's Name: \DataField{student}{name}

Activity: \DataField{activity}{name}

\begingroup
  \scriptsize
  \begin{tabular}{lllll}
  \StudentCheckListTable{L1,L2,L3}{C1,C2,C3,C4,C5}
  \end{tabular}
\endgroup
\end{codestore}

%%\tsdemo*[stcurrdemo]{demo.student.single-use}

\subsection{Iterating over a list of students}
Given a student list, one can just iterate over it. Note that no explicit index/hash has been used. Also note that, since the \tsobj{\studentlistiterate} executes \tsobj{\listemptytermsifnone}, there is no explicit  need to test if there are 3 examiners.

\begin{codestore}[demo.student.list]
\studentlistsort[last]{FW-II} % sorting the students by their last name.
\studentlistiterate{FW-II}{
    \studentgrade[A]{6} %using the A (formula) and setting the minimum at 6.
    \par
    Student: \DataField{student}{name}\par
    \studentAdvCase
      { %more than one
        Advisors:
        \studentadvisoriterate
          {
            \DataField{student.advisor}{name}, ~
          }
      }
      { %just one
        Advisor: \DataField{student.advisor}{name}
      }
    \par
    
    \begin{tabular}{@{\hspace{10mm}}lll}
    \studentReviewerSelect{1} 
    1st examiner: \DataField{student.reviewer}{name} & grade: \DataField{student.reviewer}{grade} \\
    \studentReviewerSelect{2} 
    2nd examiner: \DataField{student.reviewer}{name} & grade: \DataField{student.reviewer}{grade} \\
    \studentReviewerSelect{3} 
    3rd examiner: \DataField{student.reviewer}{name} & grade: \DataField{student.reviewer}{grade} \\
    Average: \DataField{student}{grade} - \studentCase{Approved}{Failed}
    \end{tabular}

    \begingroup
      \scriptsize
      \begin{tabular}{lllll}
      \StudentCheckListTable{L1,L2,L3}{C1,C2,C3,C4,C5}
      \end{tabular}
    \endgroup
    \\[5mm]
  }
\end{codestore}

%%\tsdemo*[stcurrdemo]{demo.student.list}





\newpage
\TabTopic{BaseMatematica}

\TabEtp[\subsubsection]{Etp.01}

\TabEtp[\subsubsection]{Etp.02}

\TabEtp[\subsubsection]{Etp.03}

\TabEtp[\subsubsection]{Etp.04}

\TabEtp[\subsubsection]{Etp.05}

\TabEtp[\subsubsection]{Etp.06}

\TabEtp[\subsubsection]{Etp.07}

\TabEtp[\subsubsection]{Etp.08}

\TabEtp[\subsubsection]{Etp.09}


\TabEtp[\subsubsection]{Etp.10}

\TabEtp[\subsubsection]{Eletivas}


\noindent\hspace*{-7mm}\resizebox{150mm}{115mm}{
  \begin{tikzGraphSem}[graph title={Dependency's Graph Example},default color=gray,colors=both,deltaX=6.3,deltaY=2.5,maxX=10,maxY=-16]
   \GraphEtp{Etp.01}

 \GraphEtp{Etp.02}

 \GraphEtp{Etp.03}

 \GraphEtp{Etp.04}

 \GraphEtp{Etp.05}

 \GraphEtp{Etp.06}

 \GraphEtp{Etp.07}

 \GraphEtp{Etp.08}

 \GraphEtp{Etp.09}

 \GraphEtp{Etp.10}

  \end{tikzGraphSem}
  }

\end{document} 