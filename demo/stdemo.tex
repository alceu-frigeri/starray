%%%==============================================================================
%% Copyright 2023-present by Alceu Frigeri
%%
%% This work may be distributed and/or modified under the conditions of
%%
%% * The [LaTeX Project Public License](http://www.latex-project.org/lppl.txt),
%%   version 1.3c (or later), and/or
%% * The [GNU Affero General Public License](https://www.gnu.org/licenses/agpl-3.0.html),
%%   version 3 (or later)
%%
%% This work has the LPPL maintenance status *maintained*.
%%
%% The Current Maintainer of this work is Alceu Frigeri
%%
%% This is version {1.9b} {2025/02/14} 
%%
%% The list of files that compose this work can be found in the README.md file at
%% https://ctan.org/pkg/starray
%%
%%%==============================================================================
\NeedsTeXFormat{LaTeX2e}[2023/11/01]
\documentclass[10pt]{article}
\RequirePackage[verbose,a4paper,marginparwidth=27.5mm,top=2.5cm,bottom=1.5cm,hmargin={40mm,20mm},marginparsep=2.5mm,columnsep=10mm,asymmetric]{geometry}
%\RequirePackage[verbose,a4paper,marginparwidth=27.5mm,top=2.5cm,bottom=1.5cm,hmargin={45mm,25mm},marginparsep=2.5mm,columnsep=10mm,asymmetric]{geometry}
\usepackage{codedescribe}
\usepackage{stdemopack}
\RequirePackage[inline]{enumitem}
\SetEnumitemKey{miditemsep}{parsep=0ex,itemsep=0.4ex}

\RequirePackage[hidelinks,hypertexnames=false]{hyperref}
\begin{document}

\setnewcodekey{stdemo}{codeprefix={},resultprefix={},letter={@,_},texcs2={starray_new,starray_def_from_keyval}}

\section{Data Model}
As an example, let's define two structures, one to describe/list ``Activities'' (like a term project, course project, etc.)  and a second one to describe/list the enrolled students (assuming that each enrolled student has one, or more, advisors and a set of reviewers.
\begin{tsremark}
As in any ``procedural language'', one is advised to pay special attention and carefully design the data model, since this will shape the functions which will set and use said data.
\end{tsremark}

\subsection{Activity Set}
For the activities one could set an ``starray'' as follow:


\begin{codestore}[activity-def]
\starray_new:n {activity}
\starray_def_from_keyval:nn {activity} {
    name = Activity's~ name ,
    acronym = ACRO ,
    coord . struct =  {
        name = Coordinator's~ name,
        title = Coordinator's~ title ,
      } ,
    calendar . struct = {
        date = {-day-} ,
        week = {-week-} ,
        event = {-event-} ,
      } ,
    chkID = ,        %%% 'unique ID' for checklists
    chkmarked = ,    %%% This shall be a prop list of   marked itens
    chkunmarked = ,  %%% This shall be a prop list of unmarked itens
    chkref = ,       %%% This shall be a prop list of ref      itens
  }
\end{codestore}


\begin{codestore}[student-def]
\starray_new:n {student}
\starray_def_from_keyval:nn {student} {
  self = , %% this shall be self hash (if any)
  first = ,
  last = ,
  name = \rule{\l__stdemo_name_rule_dim}{.1pt} ,
  Nproc = \rule{\l__stdemo_ID_rule_dim}{.1pt} ,
  ID    = \rule{\l__stdemo_ID_rule_dim}{.1pt} , 
  email = \rule{\l__stdemo_email_rule_dim}{.1pt} ,
  worktitle = \rule{\l__stdemo_worktitle_rule_dim}{.1pt} ,
  remarks = ,
  board-local = {local} ,
  board-date   = {dia} ,
  board-time  = {hora} ,
  gradeavrg = 0,
  grade = ,
  flag-null = \c_false_bool , %% IF no grade was given
  flag-graded = \c_false_bool , %%% IF gradeavrg AND finalgrade already calculated (or defined)
  flag-approved = \c_false_bool ,
  flag-coadvisor = \c_false_bool ,
  advisor . struct = {
    first = ,
    last =  ,
    name = \rule{\l__stdemo_name_rule_dim}{.1pt},
    institution = \rule{\l__stdemo_name_rule_dim}{.1pt},
    title = \rule{\l__stdemo_title_rule_dim}{.1pt} ,
    email = \rule{\l__stdemo_email_rule_dim}{.1pt} ,
  } ,
  coadvisor . struct = {
    first = ,
    last =  ,
    name = \rule{\l__stdemo_name_rule_dim}{.1pt},
    institution = \rule{\l__stdemo_name_rule_dim}{.1pt},
    title = \rule{\l__stdemo_title_rule_dim}{.1pt} ,
    email = \rule{\l__stdemo_email_rule_dim}{.1pt} ,
  } ,
  reviewer . struct = {
    first = ,
    last =  ,
    name = \rule{\l__stdemo_name_rule_dim}{.1pt},
    institution = \rule{\l__stdemo_name_rule_dim}{.1pt},
    title = \rule{\l__stdemo_title_rule_dim}{.1pt} ,
    email = \rule{\l__stdemo_email_rule_dim}{.1pt} ,
    pointA = ,
    pointB = ,
    pointC = ,
    pointD = ,
    grade = 0 ,
    flag-set = \c_false_bool , 
  } ,
 }
\end{codestore}





\tscode*[stdemo]{activity-def}


Whereas, the ``coord'' sub-structured is for the activity's coordinator, whilst ``calendar'' shall (for instance) contains a list of calendar events, and, finally, the many ``chk* '' will be used for a ``check list''.

\begin{tsremark}
The ``chkID'' (and checklists). In many cases it's handy to have an unique identifier for a given structure. That can be obtained with \tsobj{\starray_get_unique_ID:nN}, and to avoid having to call this function time and time again, one can just store that ID as a field for later use. (as it will be done in this example).
\end{tsremark}
\begin{tsremark}
  Could the Coordinator's name and title be a direct property (dismissing the ``coord'' sub-structure) ? of course, that's a matter of taste/choice, on how to model it.
\end{tsremark}

\subsection{Student Set}
Similarly, a student's structure might contain, besides student's name, work title, some flags, an advisor (and co-advisor, if needed), reviewer's list (with a provision for reviewer's grade, if needed).

Of course, one doesn't need to define a \tsobj[pkg]{starray} structure using \tsobj{\starray_def_from_keyval:nn}, but, as in this,  if the set of properties is known, it always makes for a cleaner definition.

\begin{tsremark}
  The fields/properties defaults can be anything, including usual \LaTeXe\  commands, like a \tsobj{\rule} which is handy, for instance, when generating forms, e.g., if the fields are all set, a form can be created with the proper values, otherwise, it will be  created with ``rules'' in place (no need to test if the properties were set).
\end{tsremark}

\tscode*[stdemo]{student-def}



\end{document} 