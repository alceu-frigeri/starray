%%%==============================================================================
%% Copyright 2023-present by Alceu Frigeri
%%
%% This work may be distributed and/or modified under the conditions of
%%
%% * The [LaTeX Project Public License](http://www.latex-project.org/lppl.txt),
%%   version 1.3c (or later), and/or
%% * The [GNU Affero General Public License](https://www.gnu.org/licenses/agpl-3.0.html),
%%   version 3 (or later)
%%
%% This work has the LPPL maintenance status *maintained*.
%%
%% The Current Maintainer of this work is Alceu Frigeri
%%
%% This is version {1.9b} {2025/02/14} 
%%
%% The list of files that compose this work can be found in the README.md file at
%% https://ctan.org/pkg/starray
%%
%%%==============================================================================
\NeedsTeXFormat{LaTeX2e}[2023/11/01]
\documentclass[10pt]{article}
\RequirePackage[verbose,a4paper,marginparwidth=27.5mm,top=2.5cm,bottom=1.5cm,hmargin={40mm,20mm},marginparsep=2.5mm,columnsep=10mm,asymmetric]{geometry}
%\RequirePackage[verbose,a4paper,marginparwidth=27.5mm,top=2.5cm,bottom=1.5cm,hmargin={45mm,25mm},marginparsep=2.5mm,columnsep=10mm,asymmetric]{geometry}
\usepackage{codedescribe}
\usepackage{stdemo}
\RequirePackage[inline]{enumitem}
\SetEnumitemKey{miditemsep}{parsep=0ex,itemsep=0.4ex}

\RequirePackage[hidelinks,hypertexnames=false]{hyperref}
\begin{document}

\tstitle{
  author={Alceu Frigeri\footnote{\tsverb{https://github.com/alceu-frigeri/starray/tree/main/demo}}},
  date={\tsdate},
  title={The stdemo Package\break for starray version \PkgInfo{starray}{version}}
  }
 
 
\begin{tsabstract}
  As an example of how a package writer could use the \tsobj[pkg]{starray} package, this documents a demo package, \tsobj[pkg]{stdemo}, which defines a set of commands aiming to describe a set of activities and students' work associated with them. A companion document, \tsobj[pkg]{stdemodoc}, shows how to use these commands.
\end{tsabstract}

\tableofcontents

\section{Introduction}
The purpose of this is to create an example of how to use a \tsobj[pkg]{starray} in a complete setup. That for, this demo has a few parts:
\begin{enumerate*}
  \item A companion package \tsobj[pkg]{stdemo.sty},
  \item this document which documents not just the user level functions/commands, but also how the the companion package was created, and
  \item a document using the demo package.
\end{enumerate*}
\begin{tsremark}
  About the version number, since this is ``part'' of \tsobj[pkg]{starray}, and to keep tracking simple, the same version number (from \tsobj[pkg]{starray}) will be used.
\end{tsremark}

\setnewcodekey{stdemo}{codeprefix={},resultprefix={},letter={@,_},
  texcs2={
    starray_new,starray_def_from_keyval,
    starray_new_term,starray_get_unique_id,starray_gset_prop,starray_set_prop,
    starray_set_iter_from_hash,starray_get_prop,starray_gset_from_keyval,
    starray_iterate_over
    },
  keywd2={
    NewActivity,ActivitySet,ActivitySetCoord,ActivitySetCoordTitle,
    ActivitySelect,Activity,ActivityCoord,
    ActivityCalendarIterate,
    ActivitySetNewEvent,ActivitySetEventDay
    }
  }

% this shall be <activity-def> [1]
\begin{codestore}[activity-def]
\starray_new:n {activity}
\starray_def_from_keyval:nn {activity} {
    name = Activity's~ name ,
    acronym = ACRO ,
    coord . struct =  {
        name = Coordinator's~ name,
        title = Coordinator's~ title ,
      } ,
    calendar . struct = {
        date = {-day-} ,
        week = {-week-} ,
        event = {-event-} ,
      } ,
    chkID = ,        %%% 'unique ID' for checklists
    chkmarked = ,    %%% This shall be a prop list of   marked itens
    chkunmarked = ,  %%% This shall be a prop list of unmarked itens
    chkref = ,       %%% This shall be a prop list of ref      itens
  }
\end{codestore}


% this shall be <student-def> [1]
\begin{codestore}[student-def]
\starray_new:n {student}
\starray_def_from_keyval:nn {student} {
  self = , %% this shall be self hash (if any)
  first = ,
  last = ,
  name = \rule{\l__stdemo_name_rule_dim}{.1pt} ,
  ID    = \rule{\l__stdemo_ID_rule_dim}{.1pt} , 
  email = \rule{\l__stdemo_email_rule_dim}{.1pt} ,
  worktitle = \rule{\l__stdemo_worktitle_rule_dim}{.1pt} ,
  remarks = ,
  board-local = {~local/place~} ,
  board-date   = {~date~} ,
  board-time  = {~time~} ,
  gradeavrg = 0,
  grade = ,
  flag-null = \c_false_bool , %% IF no grade was given
  flag-graded = \c_false_bool , %%% IF gradeavrg AND finalgrade already calculated (or defined)
  flag-approved = \c_false_bool ,
  flag-coadvisor = \c_false_bool ,
  advisor . struct = {
    first = ,
    last =  ,
    name = \rule{\l__stdemo_name_rule_dim}{.1pt},
    institution = \rule{\l__stdemo_name_rule_dim}{.1pt},
    title = \rule{\l__stdemo_title_rule_dim}{.1pt} ,
    email = \rule{\l__stdemo_email_rule_dim}{.1pt} ,
  } ,
  coadvisor . struct = {
    first = ,
    last =  ,
    name = \rule{\l__stdemo_name_rule_dim}{.1pt},
    institution = \rule{\l__stdemo_name_rule_dim}{.1pt},
    title = \rule{\l__stdemo_title_rule_dim}{.1pt} ,
    email = \rule{\l__stdemo_email_rule_dim}{.1pt} ,
  } ,
  reviewer . struct = {
    first = ,
    last =  ,
    name = \rule{\l__stdemo_name_rule_dim}{.1pt},
    institution = \rule{\l__stdemo_name_rule_dim}{.1pt},
    title = \rule{\l__stdemo_title_rule_dim}{.1pt} ,
    email = \rule{\l__stdemo_email_rule_dim}{.1pt} ,
    pointA = ,
    pointB = ,
    pointC = ,
    pointD = ,
    grade = 0 ,
    flag-set = \c_false_bool , 
  } ,
 }
\end{codestore}



\begin{codestore}[DataRecovery]
\NewDocumentCommand{\eDataSet}{m}{
  \starray_term_syntax:n{#1}
}  

\cs_new:Npn \eDataField #1 
  { \starray_parsed_get_prop:n{#1} }
\end{codestore}


\begin{codestore}[DataRecovery]
\NewDocumentCommand{\DataField }{mm}{
  \starray_get_prop:nn{#1}{#2}
}  

\NewDocumentCommand{\DataGet}{mmm}{
  \starray_get_prop:nnN{#1}{#2}{#3}
}  
\end{codestore}

% this shall be <ActivityFunctions> [1]
\begin{codestore}[ActivityFunctions]
\NewDocumentCommand{\NewActivity}{m} {
    \starray_new_term:nn {activity}{#1}
    \starray_new_term:nn {activity.coord}{}
    \starray_get_unique_id:nNTF {activity} \l__stdemo_tmpID_tl
      {}
      {}
    \starray_gset_prop:nnV {activity}{chkID} \l__stdemo_tmpID_tl
    \prop_new_linked:c {l__stdemo_ \l__stdemo_tmpID_tl _chkmarked_prop}
    \prop_new_linked:c {l__stdemo_ \l__stdemo_tmpID_tl _chkunmarked_prop}
    \prop_new_linked:c {l__stdemo_ \l__stdemo_tmpID_tl _chkref_prop}
}

\NewDocumentCommand{\ActivitySet}{O{}mm} {
  \tl_if_blank:nTF {#1}
    {
      \starray_set_prop:nnn {activity}{name}{#3}
      \starray_set_prop:nnn {activity}{acronym}{#2}
    }
    {
      \starray_set_prop:nnn {activity[#1]}{name}{#3}
      \starray_set_prop:nnn {activity[#1]}{acronym}{#2}
    }
}
\end{codestore}

% this shall be <ActivityFunctions> [2]
\begin{codestore}[ActivityFunctions]
\NewDocumentCommand{\ActivitySetCoord}{O{}mO{}}{
  \tl_if_blank:nTF {#1}
    {
      \starray_gset_prop:nnn {activity.coord}{name}{#2}
    }
    {
      \starray_gset_prop:nnn {activity[#1].coord}{name}{#2}
    }
}

\NewDocumentCommand{\ActivitySetCoordTitle}{O{}m} {
  \tl_if_blank:nTF {#1}
    { \starray_set_prop:nnn {activity.coord}{title}{#2} }
    { \starray_set_prop:nnn {activity[#1].coord}{title}{#2} }
}
\end{codestore}



% this shall be <ActivityFunctions> [5]
\begin{codestore}[ActivityFunctions] 
\NewDocumentCommand{\ActivitySetNewEvent}{O{}mm}{
  \tl_if_blank:nTF {#1}
    {
      \starray_new_term:nn {activity.calendar}{#2}
      \starray_gset_prop:nnn {activity.calendar}{event}{#3}
    }
    {
      \starray_new_term:nn {activity[#1].calendar}{#2}
      \starray_gset_prop:nnn {activity[#1].calendar}{event}{#3}
    }
}
\end{codestore}

% this shall be <ActivityFunctions> [6]
\begin{codestore}[ActivityFunctions] 
\NewDocumentCommand{\ActivitySetEventDay}{O{}mmm}{
  \tl_if_blank:nTF {#1}
    {
      \starray_gset_from_keyval:nn {activity.calendar[#2]}
        {
            date = {#3} ,
            week = {#4} ,
        }
    }
    {
      \starray_gset_from_keyval:nn {activity[#1].calendar[#2]}
        {
            date = {#3} ,
            week = {#4} ,
        }
    }
}
\end{codestore}

% this shall be <ActivityFunctions> [7]
\begin{codestore}[ActivityFunctions]
\NewDocumentCommand{\checkdef}{mmm}{
    \starray_get_prop:nnN {activity}{chkID}\l__stdemo_chkID_tl
    \prop_gput:cnn {l__stdemo_ \l__stdemo_chkID_tl _chkmarked_prop} {#1}{\__stdemo_checkedbox:~\ #3}
    \prop_gput:cnn {l__stdemo_ \l__stdemo_chkID_tl _chkunmarked_prop} {#1}{\__stdemo_uncheckedbox:~\ #3}
    \prop_gput:cnn {l__stdemo_ \l__stdemo_chkID_tl _chkref_prop} {#2}{#1}
  }

\NewDocumentCommand{\checklist}{O{}m}{
    \tl_if_blank:nF {#1}
      {  \starray_set_iter_from_hash:nn {activity}{#1}  }
  
    \starray_get_prop:nnN {student}  {chkID} \l__stdemo_chkIDa_tl
    \starray_get_prop:nnN {activity} {chkID} \l__stdemo_chkIDb_tl
  
    \clist_map_inline:nn {#2}
      {
        \prop_get:cnNT {l__stdemo_ \l__stdemo_chkIDb_tl _chkref_prop} {##1} \l__stdemo_checkref_tl
          {
            \prop_get:ceN {l__stdemo_ \l__stdemo_chkIDb_tl _chkmarked_prop} {\l__stdemo_checkref_tl} \l__stdemo_checkB_tl
            \prop_gput:cee {l__stdemo_ \l__stdemo_chkIDa_tl _checklist_prop} {\l__stdemo_checkref_tl} {\l__stdemo_checkB_tl}
          }
      }
  }
\end{codestore}


% this shall be <ActivityFunctions> [3]
\begin{codestore}[ActivityFunctions]
\NewDocumentCommand{\ActivitySelect}{m}
  { 
    \starray_set_iter_from_hash:nn {activity}{#1} 
  }
\end{codestore}

% this shall be <ActivityFunctions> [4]
\begin{codestore}[ActivityFunctions]
\NewDocumentCommand{\Activity}{O{}m}{
  \tl_if_blank:nTF {#1}
    { \starray_get_prop:nn {activity}{#2} }
    { \starray_get_prop:nn {activity[#1]}{#2} }
}

\NewDocumentCommand{\ActivityCoord}{O{}m}{
  \tl_if_blank:nTF {#1}
    { \starray_get_prop:nn {activity.coord}{#2} }
    { \starray_get_prop:nn {activity[#1].coord}{#2} }
}
\end{codestore}

% this shall be <ActivityFunctions> [5]
\begin{codestore}[ActivityFunctions]
\NewDocumentCommand{\ActivityCalendarIterate}{m}{
    \starray_iterate_over:nn{activity.calendar}{#1}
}
\end{codestore}



%% this shall be <ActivityFunctions> [8]
\begin{codestore}[ActivityFunctions]
\NewDocumentCommand{\ActivitySelect}{m}{ \starray_set_iter_from_hash:nn {activity}{#1} }

\NewDocumentCommand{\Activity}{O{}m}{
  \tl_if_blank:nTF {#1}
    { \starray_get_prop:nn {activity}{#2} }
    { \starray_get_prop:nn {activity[#1]}{#2} }
}

\NewDocumentCommand{\ActivityCoord}{O{}m}{
  \tl_if_blank:nTF {#1}
    { \starray_get_prop:nn {activity.coord}{#2} }
    { \starray_get_prop:nn {activity[#1].coord}{#2} }
}
\end{codestore}

\begin{codestore}[ActivityUseDemo]
  \NewActivity{FinalWork I}
  \ActivitySet{Final Work I}{FW001}
  
  \NewActivity{FinalWork II}
  
  \NewActivity{InternShip B}
  \ActivitySet{Final Intership}{IN099}
  
  \ActivitySet[FinalWork II]{Final Work II}{FW002}
\end{codestore}


\begin{codestore}[ActivityUseDemo]
  \ActivitySetCoordTitle[FinalWork I]{Prof. }
  \ActivitySetCoord[FinalWork I]{Willian S.}
   
  \ActivitySetCoordTitle[FinalWork II]{Prof. }
  \ActivitySetCoord[FinalWork II]{James S.}

  \ActivitySetCoordTitle[InternShip B]{Dr. }
  \ActivitySetCoord[InternShip B]{Samuel J. }
  
\end{codestore}






\section{Data Model}\label{DataModel}
As an example, let's define two structures:
\begin{enumerate}
  \item ``Activities'' (like a term project, course project, etc.)  with associated 
elements: 
\begin{enumerate}
  \item name, acronym
  \item coordinator
  \item calendar events (presentation dates, exams...)
  \item a check list.
\end{enumerate}

  \item Students, with associated date:
\begin{enumerate}
  \item name, IDs, etc.
  \item advisor and (perhaps) co-advisor.
  \item reviewer(s).
  \item flags, etc.
\end{enumerate}
  
  
\end{enumerate}

\begin{tsremark}
As in any ``procedural language'', one is advised to  carefully design the data model, since this will shape the functions which will set and use said data.
\end{tsremark}

\begin{tsremark}
  Pay attention to the use of the tildes,  \~{} , since those definitions will be made, most likely, in an \tsobj[pkg]{expl3} code régime, one has to remember that spaces are ignored, therefore, if needed, one has to explicitly use a tilde instead of a space.
\end{tsremark}


\subsection{Student Set}
A student's structure might contain, besides student's name, work title, some flags, an advisor (and co-advisor, if needed), reviewer's list (with a provision for reviewer's grade, if needed).

Of course, one doesn't need to define a \tsobj[pkg]{starray} structure using \tsobj{\starray_def_from_keyval:nn}, but, as in this,  if the set of properties is known, it always makes for a cleaner definition.

\begin{tsremark}
  The fields/properties defaults can be anything, including usual \LaTeXe\  commands, like a \tsobj{\rule} which is handy, for instance, when generating forms, e.g., if the fields are all set, a form can be created with the proper values, otherwise, it will be  created with ``rules'' in place (no need to test if the properties were set).
\end{tsremark}

\tscode*[stdemo]{student-def}



\subsection{Activity Set}\label{Activity-set}
For the activities one could set an ``starray'' as follow:







\tscode*[stdemo]{activity-def}


Whereas, the ``coord'' sub-structure is for the activity's coordinator, whilst ``calendar'' shall (for instance) contains a list of calendar events, and, finally, the many ``chk* '' will be used for a ``check list''.

\begin{tsremark}
The ``chkID'' (and checklists). In many cases it's handy to have an unique identifier for a given structure. That can be obtained with \tsobj{\starray_get_unique_ID:nN}, and to avoid calling this function time and time again, one can just store that ID as a field for later use.
\end{tsremark}
\begin{tsremark}
  Could the Coordinator's name and title be a direct property (dismissing the ``coord'' sub-structure)? of course, that's a matter of taste/choice, on how to model it.
\end{tsremark}



\section{Auxiliary Functions}
Once the data layout is set (see \ref{DataModel}) the next step is to define a set of (document level) functions, so the data can be initialized and used by the end user.

\subsection{Generic Recovery Functions}

\begin{codedescribe}{\DataField,\DataGet}
  \begin{codesyntax}
    \tsmacro{\DataField}{starray,item}
    \tsmacro{\DataGet}{starray,item,store-var}
  \end{codesyntax}
\tsobj{\DataField} will recovery an item from any \tsobj[pkg]{starray}, for instance, \tsobj[marg]{starray} might be \tsobj[meta,sep=or]{activity,activity.coord,activity.calendar,student,student.advisor} etc. whilst \tsobj[marg]{item} might be any corresponding field. The \tsobj{\DataGet} will store said value in an auxiliary \tsobj[marg]{store-var}.
\end{codedescribe}


\tscode*[stdemo]{DataRecovery}[2]

\subsection{Activity's Functions}

One could define a single function to initialize all fields (using a key=val interface), but, in a more traditional approach  one can set two functions to start the initialization process \tsobj{\NewActivity,\ActivitySet}. 


~

\begin{codedescribe}{\NewActivity,\ActivitySet}
\begin{codesyntax}
  \tsmacro{\NewActiviy}{act-ID}
  \tsmacro{\ActivitySet}[act-ID]{name,acronym}
\end{codesyntax}
\tsobj{\NewActivity} will create a new one, \tsobj[marg]{act-ID} will be the identifier of it. \tsobj{\ActivitySet} will set the \tsobj[marg]{name,acronym} of an activity. If not given, the current \tsobj[oarg]{act-ID} will be used.

The idea is to (normally) use one right after the other, though, once created with \tsobj{\NewActivity}, an activity can be initialized/changed at a later point using the optional argument from \tsobj{\ActivitySet}.

\end{codedescribe}
\begin{tsremark}
  Every time a \tsobj[pkg]{starray} is instantiated, up to two hashes are created: a numerical one (starting at one) and an ``user defined one''. In the \tsobj{\NewActivity} function above, the given argument is that hash, so the just created instance can be later referenced by it. Of course, it must be an unique ID/hash.
\end{tsremark}
\begin{tsremark}
  One thing to be noticed about \tsobj[pkg]{starrays}: every structure has an associated internal index (iterator). When you create a new instance, this iterator always points to the newly created one, therefore, sparing the use of an explicit index.
\end{tsremark}

\tscode*[stdemo]{ActivityFunctions}[1]

~

Similarly, one can define some functions to set the activity's coordinator. Of course, it's up to the package programmer to choose if one, two (or more) functions for this.

\begin{codedescribe}{\ActivitySetCoord,\ActivitySetCoordTitle}
\begin{codesyntax}
  \tsmacro{\ActivitySetCoord}[act-ID]{name}
  \tsmacro{\ActivitySetCoordTitle}[act-ID]{title}
\end{codesyntax}
For both, \tsobj{\ActivitySetCoord,\ActivitySetCoordTitle}, the optional argument \tsobj[oarg]{act-ID} refers to an already create activity, and, if not given, will use the current one.
\end{codedescribe}

\tscode*[stdemo]{ActivityFunctions}[2]



And the associated ``Calendar Events'', assuming there will be a fixed set of events (each semester/year), leaving the date to be set later on. 

\begin{codedescribe}{\ActivitySetNewEvent,\ActivitySetEventDay}
\begin{codesyntax}
  \tsmacro{\ActivitySetNewEvent}[act-ID]{event-ID,description}
  \tsmacro{\ActivitySetEventDay}[act-ID]{event-ID,date,week}
\end{codesyntax}
The optional argument \tsobj[oarg]{act-ID} refers to an already create activity, and, if not given, will use the current one. \tsobj[marg]{event-ID} can be any identifier. That way, the user can first define a set of events, and only later on, set the associated dates.
\end{codedescribe}


\tscode*[stdemo]{ActivityFunctions}[3]

~

\tscode*[stdemo]{ActivityFunctions}[4]



In many cases, it's desirable to have a ``check list''. What such list could entice is always up to debate, the idea behind the few next functions is to allow the end user to define which items such a list (as a matrix) might have.

\begin{codedescribe}{\checkdef,\checklist}
\begin{codesyntax}
  \tsmacro{\checkdef}{chkID,chkPos,chktext}
  \tsmacro{\checklist}[act-ID]{chkID-list}
\end{codesyntax}
  \tsobj[marg]{chkID} is just an ID to reference the check list item. \tsobj[marg]{chkPos} will relate the item to a position in a matrix (tabular environment, see \tsobj{\StudentCheckListTabLine}) and \tsobj[marg]{chktext} is the (assumed) short text.
  The command \tsobj{\checkdef} defines/create a new item, whilst \tsobj{\checklist} sets a list of \tsobj[marg]{chkID}s (note that \tsobj[marg]{chkID-list} is a csv list).
\end{codedescribe}
\begin{tsremark}
  In the implementation below, note that the check list items are associated with an activity, but the final list itself is a student by student one. Better said, each student will have a property list (constructed based on a student's unique ID) of the ``checked items'' (see \tsobj{\StudentCheckListTabLine}).
\end{tsremark}



\tscode*[stdemo]{ActivityFunctions}[5]

~

\subsubsection{Recovering/Using Activity's Data}

\begin{codedescribe}{\ActivitySelect}
  \begin{codesyntax}
    \tsmacro{\ActivitySelect}{act-ID}
  \end{codesyntax}
  This will just select an activity, identified by \tsobj[marg]{act-ID} as the current one. So that, in the following commands, one can avoid the first, optional, argument.
\end{codedescribe}


\tscode*[stdemo]{ActivityFunctions}[6]


\begin{codedescribe}{\Activity,\ActivityCoord}
\begin{codesyntax}
  \tsmacro{\Activity}[act-ID]{field}
  \tsmacro{\ActivityCoord}[act-ID]{field}
\end{codesyntax}
  \tsobj{\Activity} will recover the associated \tsobj[marg]{field} value (from \ref{Activity-set} it can (reasonably) be \tsobj[meta,sep=or]{name,acronym}).
  \tsobj{\ActivityCoord}, similarly, will recover the associated \tsobj[marg]{field} value for the activity's coordinator (field can be, from \ref{Activity-set},  \tsobj[meta,sep=or]{name,title}).
\end{codedescribe}

\tscode*[stdemo]{ActivityFunctions}[7]

\begin{codedescribe}{\ActivityCalendarIterate}
  \begin{codesyntax}
    \tsmacro{\ActivityCalendarIterate}{code}
  \end{codesyntax}
  This is a helper function, so that the end user is free to construct an ``Event Calendar'' with the (activity's) stored data. The suggested pattern is: 
  \begin{enumerate*}
    \item Select an activity with \tsobj{\ActivitySelect}, then
    \item execute the code for each item stored in the activity's calendar list.
  \end{enumerate*}. The user is supposed to use (in \tsobj[marg]{code}) \tsobj[code,sep=or]{\DataField,\DataGet} to retrieve and use the calendar's data.
\end{codedescribe}

\tscode*[stdemo]{ActivityFunctions}[8]


\subsection{Student's Functions}

\end{document} 