
%%%%%%%%%%%%%%%%%
%%%%%%%%%%%%%%%%%
%%%
%%%  Etapa 07
%%%
%%%%%%%%%%%%%%%%%
%%%%%%%%%%%%%%%%%



%%%%%%
%%%%%%
%
\classdef[Pro.Maquinas]{ENG10049}{4}{ACIONAMENTO DE MÁQUINAS ELÉTRICAS}

     \csummary{Seleção de motores elétricos. Comportamento e modelos dinâmicos de máquinas elétricas. Controle de velocidade e torque. Princípios de eletrônica potência, operação e componentes básicos de conversores estáticos. Acionamento de máquinas com conversores estáticos.}


      \Orgbibdef{B. K. Bose. Power electronics and variable frequency drives : technology and applications.. IEEE Press, 1997. ISBN 0780310845}
      \Orgbibdef{Edson Bin. Máquinas Elétricas e Acionamento. Campus/Elsevier, 2009. ISBN 9788535230291}
      \Orgbibdef{P. C. Krause. Analysis of electric machinery and drive systems.. Wiley, 2013. ISBN 9781118024294}

      \Orgbibdef[basic]{Austin Hughes. Electric motors and drives : fundamentals, types and applications. Elsevier, 2006. ISBN 9780750647182}
      \Orgbibdef[basic]{Ion Boldea. Electrical Drives. CRC Press, 1999. ISBN 0849325218}
      \Orgbibdef[basic]{S. K. Pillai. A First Course in Electrical Drives. John Willey, 1989. ISBN 047021399X}
      \Orgbibdef[basic]{W. Leonhard. Control of Electrical Drives. Springer Verlag, 1990. ISBN 3540136509}

      \Orgbibdef[compl]{D. C. White, H. H. Woodson. Electromechanical Energy Conversion. John Wiley, 1959. ISBN 978-0262230292}
      \Orgbibdef[compl]{Gordon R. Slemon. Electrical Machines and Drives. Addison-Wesley, 1992. ISBN 0201578859}
      \Orgbibdef[compl]{J. J. Cathey. Electrical Machines: Analysis and Design Applying Matlab. MCGraw-Hill, 2000. ISBN 0072423706}
      \Orgbibdef[compl]{J. M. D. Murphy. Power Electronics of AC Motors. Pergamon, 1988. ISBN 0080226833}

%%%%%%
%%%%%%
%
\classdef[Pro.Control]{ENG10005}{2}{LABORATÓRIO DE CONTROLE}

     \csummary{Métodos experimentais para ajuste de controladores PID (Ziegler-Nichols e similares). Verificação experimental de desempenho de malhas de controle. Projeto de controladores por métodos de controle clássico. Projeto prático de controladores para: processo térmico, processo mecânico, controle de velocidade de motor elétrico, controle de posição de motor elétrico.}

      \Orgbibdef{Bazanella, Alexandre Sanfelice; Gomes da Silva Junior, Joao Manoel. Sistemas de controle: princípios e métodos de projeto. UFRGS, 2005. ISBN 8570258496}

      \Orgbibdef[basic]{Ogata, Katsuhiko. Engenharia de controle moderno. Prentice Hall do Brasil, ISBN 8587918230}

      \Orgbibdef[compl]{Astrom, Karl Johan; Hagglund, Tore. Pid controllers: theory, design, and tuning. ISA, ISBN 978-1556175169}
      \Orgbibdef[compl]{Franklin, Gene F.; Powell, J. David; Emami-Naeini, Abbas. Feedback control of dynamic systems. Prentice Hall, ISBN 0131499300}

%%%%%%
%%%%%%
%
\classdef[Pro.Automacao]{ENG04475}{5}{MICROPROCESSADORES I}

     \csummary{Arquitetura de microprocessadores. Endereçamento e conjunto de instruções. Memória e adaptadores de interface de entrada e saída. Projeto lógico e elétrico de sistemas microprocessados. Sistemas supervisores. Programação e algoritmos.}


      \Orgbibdef{Cady, Fredrick M... Microcontrollers and Microcomputers: Principles of Software and Hardware Engineering.. New York.: Oxford University Press., 2010. ISBN 9780195371611}
      \Orgbibdef{Pont, Michael J... Embedded C.. Boston.: Addison-Wesley Professional., 2002. ISBN 9780201795233}
      \Orgbibdef{Susnea, Ioan; Mitescu, Marian.. Microcontrollers in Practice.. Berlin.: Springer-Verlag Berlin Heidelberg., 2005. ISBN 9783540283089}

      \Orgbibdef[basic]{Balch, Mark.. Complete Digital Design: A Comprehensive Guide to Digital Electronics and Computer System Architecture.. New York.: McGraw-Hill., 2003. ISBN 9780071409278}
      \Orgbibdef[basic]{McFarland, Grant.. Microprocessor Design: A Practical Guide from Design Planning to Manufacturing.. New York.: McGraw-Hill Education., 2006. ISBN 9780071459518}
      \Orgbibdef[basic]{Nicolosi, Denys Emílio Campion.. Microcontrolador 8051 detalhado.. São Paulo.: Erica., 2013. ISBN 9788571947214}
      \Orgbibdef[basic]{Nicolosi, Denys Emílio Campion; Silva, Caio Mario Da.. Laboratório de Microcontroladores Família 8051 : Treino de Instruções, Hardware e Software.. São Paulo.: Erica., 2014. ISBN 9788571948716}
      \Orgbibdef[basic]{Sen Gupta, Gourab.. Embedded Microcontroller Interfacing: Designing Integrated Projects.. Berlin.: Springer-Verlag Berlin Heidelberg., 2010. ISBN 9783642136368}
      \Orgbibdef[basic]{Silva Junior, Vidal Pereira da.. Aplicações práticas do microcontrolador 8051.. São Paulo.: Erica., 2004. ISBN 8571949395}
      \Orgbibdef[basic]{Silva Junior, Vidal Pereira da.. Microcontrolador 8051 : hardware e software.. São Paulo.: Erica., 1990. ISBN 8571940363}

      \Orgbibdef[compl]{Hennessy, John L.; Patterson, David A... Arquitetura de Computadores : Uma Abordagem Quantitativa.. Rio de Janeiro.: Elsevier Brasil., 2013. ISBN 9788535261226}
      \Orgbibdef[compl]{Hennessy, John L.; Patterson, David A... Computer Architecture : A Quantitative Approach.. Waltham, Mass.: Elsevier Morgan Kaufmann., 2012. ISBN 9780123838728}
      \Orgbibdef[compl]{Patterson, David A.; Hennessy, John L... Computer Organization and Design : The Hardware/Software Interface.. Amsterdam.: Elsevier Morgan Kaufmann., 2013. ISBN 9780124077263}
      \Orgbibdef[compl]{Patterson, David A.; Hennessy, John L... Organização e Projeto de Computadores : A Interface Hardware/Software.. Rio de Janeiro.: Elsevier Brasil., 2013. ISBN 9788535235852}


%%%%%%
%%%%%%
%
\classdef[Pro.Fabricacao]{ENG03021}{4}{PROCESSOS DISCRETOS DE PRODUÇÃO}

     \csummary{Introdução aos Materiais Empregados em Engenharia e Seleção de materiais. Fundição: princípios; principais tipos de moldes utilizados e processos de fabricação empregados, tais como fornos elétricos, por indução, etc. Conformação Mecânica: princípios; principais processos 
     empregados, tais como laminação, forjamento, etc., `a quente' e `a frio'. Usinagem: princípios; principais métodos empregados, tais como torno, retífica, etc. Soldagem e Técnicas Conexas: princípios; principais processos empregados, tais como eletrodo revestido, MIG/MAG, etc. Introdução 
     ao Planejamento das Operações de Manufatura, considerações econômicas e comparações de custos entre os processos descritos.}

      \Orgbibdef{GROOVER, Mikell P.. Fundamentals of Modern Manufacturing: Materials, Processes, and Systems. New York: John Wiley, 2012. ISBN 1118231465}

      \Orgbibdef[basic]{A. E. Diniz, F. C. Marcondes, N. L. Coppini. Tecnologia da usinagem dos materiais. São Paulo: Artliber, 2006. ISBN 8587296019}
      \Orgbibdef[basic]{J. M. G. de Carvalho Ferreira. Tecnologia da fundição. Lisboa: Fundação Calouste Gulbenkian, 1999. ISBN 9723108372}
      \Orgbibdef[basic]{P. R. Cetlin, H. Helman. Fundamentos da conformação mecânica dos metais. São Paulo: Artliber, 2005. ISBN 8588098288}
      \Orgbibdef[basic]{P. V. Marques, P. J. Modenesi, A. Q. Bracarense. Soldagem - fundamentos e tecnologia. Belo Horizonte: UFMG, 2009. ISBN 8570417489}

      \Orgbibdef[compl]{C. Lefteri. Como se faz: 82 técnicas de fabricação para design de produtos. Blucher, 2009. ISBN 978-85-212-0506-7}
      \Orgbibdef[compl]{G. E. Dieter. Metalurgia Mecânica. Rio de Janeiro: Guanabara Dois, 1981}


%%%%%%
%%%%%%
%
\classdef[Pro.Automacao]{ENG10048}{4}{PROTOCOLOS DE COMUNICAÇÃO}

     \csummary{Conceitos básicos de redes de computadores. Definição de sistemas abertos (modelo ISO/OSI). Nível físico, enlace de dados; algoritmos de detecção e correção de erro, redes, transporte e aplicação. Barramentos industriais para automação e instrumentação: IEEE448, Profibus, Fieldbus, CAN-BUS e outros protocolos de chão-de-fábrica.}

      \Orgbibdef{Tanembaum, Andrew S. Redes de Computadores. Campus, ISBN 8535211853}

      \Orgbibdef[basic]{Soares, L.F.G., Lemos, G. e Colcher, S. Redes de Computadores - Das LANs, MANs e WANs as Redes ATM. Campus, ISBN 857001998X}

%%%%%%
%%%%%%
%
\classdef[Pro.Control]{ENG10018}{4}{SISTEMAS DE CONTROLE II}

     \csummary{Análise e projeto de sistemas de controle por métodos freqüenciais. Sensibilidade e robustez de sistemas de controle. Análise de ciclo-limite em sistemas não-lineares. Modelagem, análise e projeto de sistemas de controle por variáveis de estado.}

      \Orgbibdef{A.S. Bazanella, J.M. Gomes da Silva Jr.. Sistemas de Controle: Princípios e Métodos de Projeto.. UFRGS, 2005. ISBN 85-7025-849-6}
      \Orgbibdef{Ogata, Katsuhiko; Maya, Paulo Alvaro. Engenharia de controle moderno. Rio de Janeiro: Prentice-Hall do Brasil, 2003. ISBN 8587918230; 9788597918239}

      \Orgbibdef[basic]{G.F. Franklin, J.D. Powell, A.Emami-Naeini. Feedback Control of Dynamic Systems. Prentice Hall, 2002. ISBN 0130323934}

      \Orgbibdef[compl]{Boldrini; Costa; Figueiredo; Wetzler. Álgebra Linear. Harbra, 1986. ISBN 8529402022}
      \Orgbibdef[compl]{Chen, C.T.. Linear System Theory.. Oxford University Press, 1998. ISBN 0195117778}
      \Orgbibdef[compl]{Lucíola Campestrini. Sintonia de controladores PID descentralizados baseada no método do ponto crítico. 2006}
      \Orgbibdef[compl]{Volnei Zanchin. Projetos de controladores para sistemas de potência utilizando LMI'S. 2003}


%%%%%%
%%%%%%
%
\classdef[Pro.Maquinas]{ENG03027}{4}{SISTEMAS HIDRÁULICOS E PNEUMÁTICOS}

     \csummary{Introdução à hidráulica e pneumática industrial. descrição de componentes para circuitos de comando e controle: atuadores, válvulas, cilindros, bombas e compressores. Características e propriedades dos fluidos hidráulicos. Elementos de mecatrônica.}


      \Orgbibdef{Prudente, F.. Automação Industrial Pneumática: Teoria e Aplicações. LTC, 2013. ISBN 9788521621195}
      \Orgbibdef{Rabie, M.. Fluid Power Engineering. McGraw-Hill, 2009. ISBN 0071622462}
      \Orgbibdef{Watton, J.. Fundamentos de Controle em Sistemas Fluidomecânicos. LTC, ISBN 9788521620259}

      \Orgbibdef[basic]{Capuamo, F. G., Idoeta, I. V. Elementos de Eletrônica Digital. Erica, ISBN 9788571940192}
      \Orgbibdef[basic]{Cundiff, J. S. Buckmaster, D. R.. Fluid Power Circuits and Controls: Fundamentals and Applications. Taylor and Francis, 2011. ISBN 9781439827819}
      \Orgbibdef[basic]{Linsingen, I. V.. Fundamentos de Sistemas Hidráulicos. UFSC, 2001. ISBN 9788532803986}

      \Orgbibdef[compl]{Bollmann, A. Fundamentos da automação Industrial Pneutrônica, Projetos de Comandos Binários Eletropneumáticos. São Paulo: ABHP ? Associação Brasileira de Hidráulica e Pneumática, 1996}
      \Orgbibdef[compl]{Bolton, W.. Pneumatic and Hydraulic Systems. Butterworth-Heinemann}
      \Orgbibdef[compl]{Martin, H.. The Design of Hydraulic Components and Systems. Elis Horwood Limited}
      \Orgbibdef[compl]{Merritt, Herbert E.. Hydraulic control systems. John Wiley, 1967. ISBN 0471596175}
      \Orgbibdef[compl]{Parr, A.. Hydraulics and Pneumatics ? A Technician?s and Engineer?s Guide. New York: Elsevier Ltd., 2007. ISBN 0750644192}
      \Orgbibdef[compl]{Yeaple, F.. Fluid Power Design Handbook. New York: Marcel Dekker, Inc., 1996. ISBN 0824795628}


%%%%%%%%%%%%%%%%%
%%%%%%%%%%%%%%%%%
%%%
%%%  Etapa 08
%%%
%%%%%%%%%%%%%%%%%
%%%%%%%%%%%%%%%%%



%%%%%%
%%%%%%
%
\classdef[Pro.Fabricacao]{ENG03045}{4}{ELEMENTOS DE MÁQUINAS}

     \csummary{Noções básicas sobre projeto mecânico. Fadiga dos materiais. Eixos de transmissão. Dimensionamento, seleção e aplicação de molas, 
     rolamentos, mancais de escorregamento, engrenagens, freios e embreagens, elementos flexíveis, parafusos de fixação, acoplamentos elásticos, elementos de transmissão de movimento.}

      \Orgbibdef{JUVINAL R. C., MARSHEK, K. M.. Fundamentos do Projeto de Componentes de Máquinas. 978-85-216-1578-1, 2008. ISBN 978-85-216-1578-1}

      \Orgbibdef[basic]{NORTON, R. L.. Projeto de Máquinas - Uma abordagem integrada. Porto Alegre: Bookman, 2004. ISBN 8582600224}
      \Orgbibdef[basic]{SHIGLEY, Joseph Edward; MISCHKE, Charles R.; BUDYNAS, Richard G.. Projeto de engenharia mecânica. Porto Alegre: Bookman, 2004. ISBN 85-363-0562-2}

      \Orgbibdef[compl]{HIBBELER, R. C.. Resistência dos Materiais. Pearson, 2010. ISBN 9788576053736}

%%%%%%
%%%%%%
%
\classdef[Pro.ContProc]{ENG07042}{4}{MODELAGEM E CONTROLE DE PROCESSOS INDUSTRIAIS}

     \csummary{Introdução à modelagem matemática de processos industriais. Aplicação das leis de conservação em regime estacionário e dinâmico. 
     Equações constitutivas. Simulação estática e dinâmica de processos. Malhas de controle típicas da indústria de processos. Projeto de controladores aplicados na indústria de processos.}

      \Orgbibdef{Bequette, B.W.. Process Dynamics: Modeling, Analysis, and Simulation. Oxford University Press, 1998. ISBN 0132068893}
      \Orgbibdef{Campos, M.; Teixeira, H.. Controles Típicos de Equipamentos e Processos Industriais. Rio de Janeiro: Edgar Blücher, 2006. ISBN 9788521205524}
      \Orgbibdef{Seborg, D. E.; Edgar, T. F.; Mellichamp, D. A.. Process Dynamics and Control. Wiley, 2003. ISBN 0470128674}

      \Orgbibdef[basic]{Edgar, T.F.. Optimization of Chemical Processes. McGraw-Hill, 2001. ISBN 0070393591}
      \Orgbibdef[basic]{Luyben, W. L.. Process Modeling, Simulation and Control for Chemical Engineers. McGraw-Hill, 1990. ISBN 0070391599}
      \Orgbibdef[basic]{Rice, R.G.. Applied Mathematics and Modeling for Chemical Engineers. John Wiley, 1995. ISBN 9781118024720}

      \Orgbibdef[compl]{Hangos, K.; Cameron, I.. Process Modelling and Model Analysis. London: Academic Press, 2001. ISBN 0121569314}
      \Orgbibdef[compl]{Ogunnaike, B. A.; Ray, W. H.. Process Dynamics, Modeling, and Control. Oxford: Oxford University Press, 1994. ISBN 9780195091199}

%%%%%%
%%%%%%
%
\classdef[Pro.Control]{ENG10019}{4}{SISTEMAS DE CONTROLE DIGITAIS}

     \csummary{Análise de Sistemas de Controle amostrados através da transformada Z. Digitalização de controladores analógicos. Identificação de sistemas pelo método dos mínimos quadrados. Projeto de controladores digitais para sistemas monovariáveis. Implementação de controladores digitais.}

      \Orgbibdef{A.S. Bazanella, L. Campestrini, D. Eckhard. Data-Driven Controller Design - the H2 Approach. Holanda: Springer, 2012. ISBN 978-94-007-2300-9}
      \Orgbibdef{Astrom, Karl Johan; Wittenmark, Bjorn. Computer-controlled systems :theory and design. London: Prentice-Hall International, 1997. ISBN 0137367872}
      \Orgbibdef{Franklin, Gene F.; Powell, J. David; Emami-Naeini, Abbas. Feedback control of dynamic systems. Upper Saddle River, N.J.: Pearson Prentice Hall, c2006. ISBN 0131499300}

      \Orgbibdef[basic]{Aguirre, Luis Antonio. Introdução à identificação de sistemas:técnicas lineares e não-lineares aplicadas a sistemas reais. Belo Horizonte: Editora UFMG, 2007. ISBN 9788570415844}
      \Orgbibdef[basic]{Kuo, Benjamin C.. Automatic control systems. Englewood Cliffs: Wiley, 2002. ISBN 0471134767}
      \Orgbibdef[basic]{Ogata, Katsuhiko. Discrete-time control systems. Upper Saddle River, N.J.: Prentice Hall, c1995. ISBN 0130342815}

      \Orgbibdef[compl]{Bazanella, Alexandre Sanfelice; Silva Junior, Joao Manoel Gomes da. Sistemas de controle:princípios e métodos de projeto. Porto Alegre: Editora da Universidade/UFRGS, 2005. ISBN 8570258496}
      \Orgbibdef[compl]{Constantine H. Houpis and Gary B. Lamont. Digital Control Systems. McGraw-Hill, 1991. ISBN 0070305005}
      \Orgbibdef[compl]{Elder M. Hemerly. Controle por Computador de Sistemas Dinâmicos. São Paulo: Edgard Blücher, 2000. ISBN 8521202660}
      \Orgbibdef[compl]{Geromel, José Claudio; Palhares, Alvaro Geraldo Badan. Análise linear de sistemas dinâmicos :teoria, ensaios práticos e exercícios. São Paulo: Edgard Blücher, 2004. ISBN 8521203357; 9788521203353}

%%%%%%
%%%%%%
%
\classdef[Pro.Fabricacao]{ENG03387}{4}{SISTEMAS DE FABRICAÇÃO}

     \csummary{Modos de produção e arranjo físico industrial. Modelos e métricas de produção. Análise de sistemas de produção: estações de trabalho operadas manualmente e automatizadas, análise de grupos de máquinas, linhas de montagem. Tecnologia de grupo: sistemas de codificação e 
     classificação, métodos para formação de famílias de peças e de células de manufatura. Manufatura celular. Movimentação interna de materiais e armazenamento. Introdução à fabricação CNC.}


      \Orgbibdef{Groover, Mikell P.. AUTOMAÇAO INDUSTRIAL E SISTEMAS DE MANUFATURA. Pearson Brasil, 2010. ISBN 8576058715}

      \Orgbibdef[basic]{Black, T.. O Projeto da Fábrica com Futuro. Porto Alegre: Bookman, 2000}

      \Orgbibdef[compl]{Groover, Mikell P.. Automation Production Systems and Computer integrated Manufacturing. Prentice-Hall, 2007. ISBN 0132393212}
      \Orgbibdef[compl]{Liker, Jeffrey K.. O modelo Toyota :14 princípios de gestão do maior fabricante do mundo. Porto Alegre: Bookman, 2005. ISBN 9788536304953}
      \Orgbibdef[compl]{Lorino, F. V.. Tecnologia de grupo e organização da manufatura}

%%%%%% 
%%%%%%
%
\classdef[Pro.Fabricacao]{ENG03386}{4}{FABRICAÇÃO AUXILIADA POR COMPUTADOR}

     \csummary{Processos de fabricação por usinagem: torneamento, furação e fresamento. Planejamento de processo e CAPP. Projeto orientado à manufatura e montagem (DFM/A). Fundamentos da usinagem CNC. Arquitetura de sistemas CNC: hardware e software. Linguagem ISSO. Programação 
     manual de centros de usinagem CNC. Programação manual de tornos CNC. CAD/CAM.}


      \Orgbibdef{GROOVER, Mikell P.. AUTOMAÇAO INDUSTRIAL E SISTEMAS DE MANUFATURA. Pearson Brasil, 2010. ISBN 8576058715}

      \Orgbibdef[basic]{CHANG, Tien-C.; WYSK, Richard A.; WANG, Hsu-P.. Computer-aided manufacturing. London: Prentice-Hall, 2005. ISBN 978-0131429192}
      \Orgbibdef[basic]{SAWHNEY, G. S.. Fundamentals of computer aided manufacturing. I K International Publishing House, 2007. ISBN 978-8189866372}

      \Orgbibdef[compl]{Amorim, H.J.. Furação - (Material desenvolvido para a disciplina ENG03386). 2012}
      \Orgbibdef[compl]{Amorim, H.J.. Torneamento (Material desenvolvido para a disciplina ENG03386). 2012}
      \Orgbibdef[compl]{Amorim. H.J.. Fresamento (Material desenvolvido para a disciplina ENG03386). 2013}
      \Orgbibdef[compl]{CORNELIUS, Leondes T.. Computer aided and integrated manufacturing systems: manufacturing processes. World Scientific Publishing Company, 2003. ISBN 978-9812389794}
      \Orgbibdef[compl]{GROOVER, Mikell P.. Automation, production systems, and computer-integrated manufacturing. London: Prentice Hall, 2007. ISBN 978-0132393218}

%%%%%%  
%%%%%%
%
\classdef[Pro.Control]{ENG03046}{4}{CONTROLE DE SISTEMAS FLUÍDO-MECÂNICOS}

     \csummary{Modelagem dinâmica de sistemas hidráulicos, pneumáticos e híbridos. Características não lineares de sistemas hidráulicos e pneumáticos: aspectos construtivos e análise por aproximações lineares. Servoatuadores hidráulicos e pneumáticos: análise, controle e aplicações.}


      \Orgbibdef{Manring, N.. Hydraulic Control Systems. Wiley, 2005. ISBN 9780471693116}
      \Orgbibdef{Slotine, J.-J; Li, W.. Applied Nonlinear Control. USA: Prentice-Hall, 1991. ISBN 9780130408907}
      \Orgbibdef{WATTON, John. Fundamentos de controle em sistemas fluidomecânicos. Rio de Janeiro: LTC, 2012. ISBN 9788521624516}

      \Orgbibdef[basic]{Costa, G. K., Sepehri, N.. Hydrostatic Transmissions and Actuators: Operation, Modelling and Applications. New York, USA: Wiley, 2015}
      \Orgbibdef[basic]{Cundiff, J. S., Buckmaster, D. R.. Fluid Power Circuits and Controls: Fundamentals and Applications. Boca Raton, FL, USA: CRC Press, 2013. ISBN 978-1439827819}
      \Orgbibdef[basic]{Linsingen, I. V.. Fundamentos de Sistemas Hidráulicos. UFSC, ISBN 9788532806468}
      \Orgbibdef[basic]{Merritt, H.E.. Hydraulic Control Systems. New York, USA: John Wiley, 1991. ISBN 978-0471596172}
      \Orgbibdef[basic]{Watton, J.. Fundamentos de Controle em Sistemas Fluidomecânicos. LTC-GEN, ISBN 9788521620259}

      \Orgbibdef[compl]{Akers, A., Gassman, M., Smith, R.. Hydraulic Power System Analysis. CRC - Taylor, 2006. ISBN 9780824799564}
      \Orgbibdef[compl]{Dorf, R. C., Bishop, R. H.. Modern control systems. Pearson Prentice Hall, ISBN 9780136024583}
      \Orgbibdef[compl]{Franklin, G. F., Powell, J. D., Emami-Naeini, A.. Sistemas de Controle para Engenharia. Bookman, 2013. ISBN 9788582600672}
      \Orgbibdef[compl]{Ogata, K.. Engenharia de Controle Moderno. Pearson - Prentice Hall, ISBN 9788576058106}
      \Orgbibdef[compl]{Parr, A.. Hydraulics and Pneumatics: A Technicians and Engineers Guide. Butterworth - Heinemann, ISBN 9780080966748, 0080966748}
      \Orgbibdef[compl]{Rabie, M.. Fluid Power Engineering. USA: McGraw-Hill, 2009. ISBN 0071622462}
      \Orgbibdef[compl]{Watton, J.. Modelling, Monitoring and Diagnostic Techniques for Fluid Power Systems. Springer-Verlag, 2007. ISBN 9781846283734}

%%%%%%  
%%%%%%
%
\classdef[Pro.Maquinas]{ENG10027}{4}{ELETRÔNICA FUNDAMENTAL II - B}

     \csummary{Amplificador operacional: modelamento e características. Circuitos não-lineares com amplificadores operacionais: conformadores, comparadores, detectores de pico, amostradores, conversores tensão-frequência, amplificadores logarítmicos, mono-estáveis, estáveis. Circuitos integrados especiais e aplicações. Conceitos básicos de comportamento em frequência de amplificadores.}


      \Orgbibdef{Sedra, Adel S.. Microeletrônica. Pearson Prentice Hall, 2007. ISBN 9788576050223}

      \Orgbibdef[basic]{Graeme, Jerald G.; Tobey, Gene E.; Huelsman, Laurence P.. Operational amplifiers :design and aplications. Auckland: Mcgraw-Hill International Book., [c1981]}
      \Orgbibdef[basic]{Razavi, Behzad. Fundamentos em microeletrônica. LTC, 2010. ISBN 9788521617327}
      \Orgbibdef[basic]{Sérgio Franco. Design with operational amplifiers and analog integrated circuits. WCB/McGraw-Hill, 1998. ISBN 0070218579}
      \Orgbibdef[basic]{Wait, John V.. Introduction to operational amplifier :theory and applications. --: Mcgraw-Hill, 1991}
      \Orgbibdef[basic]{Walt Jung. Op Amp Applications Handbook. USA: Newnes (Elsevier), 2004. ISBN 9780750678445}
      \Orgbibdef[basic]{Wong, Yu Jen. Function Circuits: Design and Applications. Mcgraw-Hill, ISBN 007071570X}

      \Orgbibdef[compl]{Don Lancaster. Active-Filter Cookbook. Howard Sams, ISBN 0-672-21168-8}
      \Orgbibdef[compl]{Ott, Henry W. Noise Reduction Techniques in Electronics Systems. John Wiley, ISBN 0471850683}


