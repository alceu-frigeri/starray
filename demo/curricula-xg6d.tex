
%%%%%%%%%%%%%%%%%
%%%%%%%%%%%%%%%%%
%%%
%%%  Etapa Eletivas
%%%
%%%%%%%%%%%%%%%%%
%%%%%%%%%%%%%%%%%



%%%%%%
%%%%%%
%
\classdef[Transv.integ]{CCA99008}{9}{PROJETO INTEGRADO I}

     \csummary{Atuação em equipes para analisar, propor e desenvolver soluções para problemas de Engenharia de interesse da sociedade, contemplando seus aspectos técnicos, 
     econômicos, socioambientais, de acessibilidade e de prevenção de desastres, entre outros. Os problemas serão colhidos 
     junto à sociedade via ação de extensão vinculada, sendo de natureza inerentemente aberta, prática e integradora, oportunizando que os 
     estudantes trabalhem simultaneamente os conteúdos aprendidos múltiplas disciplinas diferentes e em contextos realistas. Desenvolvimento de 
     habilidades de trabalho autônomo, comunicação, convívio social e respeito à diversidade através da atuação em grupos e do contato com questões
      e/ou indivíduos externos à Universidade. Dada a natureza integradora desta atividade, espera-se que os alunos de Projeto Integrado I, 
      interajam com os alunos de Projeto Integrado II e III no processo, permitindo a troca de experiência de alunos em diversos momentos na parte 
      profissionalizante do curso.}


      \Orgbibdef{Brockman, J.. Introduction to engineering modeling and problem solving. USA: Wiley, 2009}
      \Orgbibdef{Clarence W. de Silva. Mechatronics: an integrated Approach. Boca Raton: CRC, 2006. ISBN 0-8493-1274-4}
      \Orgbibdef{Robert C. Junival e Kurt M. Marshek. Fundamentos do Projeto de Componentes de Máquinas. Rio de Janeiro: LTC, 2008. ISBN 978-85-216-1578-1}

      \Orgbibdef[basic]{Joseph E. Shigley, Charles R. Mischke e Richard G. Budynas. Projeto de Engenharia Mecânica. Porto Alegre: Bookman, 2005. ISBN 85-363-0562-2}
      \Orgbibdef[basic]{Robert H. Bishop. The Mechatronics Handbook. Boca Raton: CRC, 2002. ISBN 0-8493-0066-5}

      \Orgbibdef[compl]{Clarence W. de Silva. Mechatronic Systems: devices, design, control, operation and monitoring. Boca Raton: CRC, 2008. ISBN 978-0-8496-0775-1}
      \Orgbibdef[compl]{Godfrey C. Onwubolu. Mechatronics: principles and applications. Oxford: Elsevier, 2005. ISBN 0-7506-6379-0}
      \Orgbibdef[compl]{Robert H. Bishop. Mechatronic System Control, Logic, and Data Acquisition. Boca Raton: CRC, 2008. ISBN 978-0-8493-9260-3}
      \Orgbibdef[compl]{Robert L. Norton. Projeto de Máquinas: uma abordagem integrada. Porto Alegre: Bookman, 2004. ISBN 85-363-0273-9}


%%%%%%
%%%%%%
%
\classdef[Transv.integ]{CCA99009}{9}{PROJETO INTEGRADO II}

     \csummary{Atuação em equipes para analisar, propor e desenvolver soluções para problemas de Engenharia de interesse da sociedade, contemplando seus aspectos técnicos, 
     econômicos, socioambientais, de acessibilidade e de prevenção de desastres, entre outros. Os problemas serão colhidos 
     junto à sociedade via ação de extensão vinculada, sendo de natureza inerentemente aberta, prática e integradora, oportunizando que os 
     estudantes trabalhem simultaneamente os conteúdos aprendidos múltiplas disciplinas diferentes e em contextos realistas. Desenvolvimento de 
     habilidades de trabalho autônomo, comunicação, convívio social e respeito à diversidade através da atuação em grupos e do contato com questões 
     e/ou indivíduos externos à Universidade.  Dada a natureza integradora desta atividade, espera-se que os alunos de Projeto Integrado II, 
     interajam com os alunos de Projeto Integrado I e III no processo, permitindo a troca de experiência de alunos em diversos momentos na parte profissionalizante do curso.}


      \Orgbibdef{Juvinall, Robert C.; Marshek, Kurt M.. Fundamentos do Projeto de Componentes de Máquinas. Rio de Janeiro: LTC, 2008. ISBN 9780849312748}
      \Orgbibdef{Silva, Clarence W. de. Mechatronics: an integrated Approach. Boca Raton: CRC, 2006. ISBN 9780849312748}

      \Orgbibdef[basic]{Bishop, Robert H.. The Mechatronics Handbook. Boca Raton: CRC, 2002. ISBN 9780849392573}
      \Orgbibdef[basic]{Shigley, Joseph E.; Mischke, Charles R.; Budynas, Richard G.. Projeto de Engenharia Mecânica. Porto Alegre: Bookman, 2005. ISBN 9780849392573}

      \Orgbibdef[compl]{Bishop, Robert H.. Mechatronic System Control, Logic, and Data Acquisition. Boca Raton: CRC, 2008. ISBN 978-0-8493-9260-3}
      \Orgbibdef[compl]{Norton, Robert L.. Projeto de Máquinas: uma abordagem integrada. Porto Alegre: Bookman, 2004. ISBN 9788582600221}
      \Orgbibdef[compl]{Onwubolu, Godfrey C.. Mechatronics: principles and applications. Oxford: Elsevier, 2005. ISBN 9780750663793}
      \Orgbibdef[compl]{Silva, Clarence W. de. Mechatronic Systems: devices, design, control, operation and monitoring. Boca Raton: CRC, 2008. ISBN 9780849607751}



%%%%%%
%%%%%%
%
\classdef[Transv.integ]{CCA99010}{9}{PROJETO INTEGRADO III}

     \csummary{Atuação em equipes para analisar, propor e desenvolver soluções para problemas de Engenharia de interesse da sociedade, contemplando 
     seus aspectos técnicos, econômicos, socioambientais, de acessibilidade e de prevenção de desastres, entre outros. Os problemas serão colhidos 
     junto à sociedade via ação de extensão vinculada, sendo de natureza inerentemente aberta, prática e integradora, oportunizando que os 
     estudantes trabalhem simultaneamente os conteúdos aprendidos múltiplas disciplinas diferentes e em contextos realistas. Desenvolvimento de 
     habilidades de trabalho autônomo, comunicação, convívio social e respeito à diversidade através da atuação em grupos e do contato com questões
      e/ou indivíduos externos à Universidade.  Dada a natureza integradora desta atividade, espera-se que os alunos de Projeto Integrado III, 
      interajam com os alunos de Projeto Integrado I e II no processo, permitindo a troca de experiência de alunos em diversos momentos na parte 
      profissionalizante do curso.}


      \Orgbibdef{Cetinkunt S.. Mechatronics with Experiments. New Delhi, India: Wiley, 2015. ISBN 9781118802465}
      \Orgbibdef{Horowitz P., Hill W.. The Art of Electronics. New York, USA: Cambridge, 2015. ISBN 9780521809269}
      \Orgbibdef{Russell K., Shen Q., Sodhi R.S.. Kinematics and Dynamics of Mechanical Systems: Implementation in MATLAB® and SimMechanics®. Boca Raton, FL, USA: CRC Press, 2015. ISBN 9781498724937}

      \Orgbibdef[basic]{Clarence W. de Silva. Mechatronics: an integrated Approach. Boca Raton: CRC, 2006. ISBN 0849312744}
      \Orgbibdef[basic]{Robert C. Junival e Kurt M. Marshek. Fundamentos do Projeto de Componentes de Máquinas. Rio de Janeiro: LTC, 2008}
      \Orgbibdef[basic]{Robert H. Bishop. The Mechatronics Handbook. Boca Raton: CRC, 2002. ISBN 0-8493-0066-5}

      \Orgbibdef[compl]{Clarence W. de Silva. Mechatronic Systems: devices, design, control, operation and monitoring. Boca Raton: CRC, 2008. ISBN 978-0-8496-0775-1}
      \Orgbibdef[compl]{Godfrey C. Onwubolu. Mechatronics: principles and applications. Oxford: Elsevier, 2005. ISBN 0-7506-6379-0}
      \Orgbibdef[compl]{Robert H. Bishop. Mechatronic System Control, Logic, and Data Acquisition. Boca Raton: CRC, 2008. ISBN 978-0-8493-9260-3}

%%%%%%
%%%%%%
%
\classdef[Pro.Automacao]{INF1017}{4}{APRENDIZADO DE MÁQUINA}

     \csummary{Fundamentos da área de aprendizado de máquina e algoritmos baseados em redes neurais e em abordagens estatísticas. Aplicações para a resolução de problemas de aprendizado supervisionado, não-supervisionado, e por reforço.}

      \Orgbibdef{Faceli, Katti; Lorena, Ana C.; Gama, João; Carvalho, Andre C.P. Inteligêcia Artificial: Uma Abordagem de Aprendizado de Máquina. Rio de Janeiro: LTC, 2011}
      \Orgbibdef{Russell, Stuart Jonathan; Norvig, Peter. Artificial intelligence: a modern approach. EUA: Prentice-Hall, 2010. ISBN 0136042597}
      \Orgbibdef{Sutton, Richard; Barto, Andrew. Reinforcement Learning: An Introduction. Cambridge: MIT Press, 1999. ISBN 0262193981}

      \Orgbibdef[basic]{Haykin, Simon; Engel, Paulo Martins. Redes neurais :princípios e prática. Porto Alegre: Bookman, 2001. ISBN 8573077182}

      \Orgbibdef[compl]{Aurélien Géron. Hands-On Machine Learning with Scikit-Learn, Keras, and Tensorflow. USA: O'Reilly Media, 2022. ISBN 9781098122485}
      \Orgbibdef[compl]{Haykin, Simon. Neural networks and learning machines. New York: Prentice Hall, c2009. ISBN 9780131471399}

%%%%%%
%%%%%%
%
\classdef[Pro.Automacao]{INF01037}{4}{COMPUTAÇÃO EVOLUTIVA}

     \csummary{Conceitos básicos sobre Vida Artificial. Introdução ao Paradigma de Algoritmos Genéticos. Aplicações de Algoritmos Genéticos. Programação evolutiva.}

      \Orgbibdef{Eiben, Agoston E.; Smith, James E.. Introduction to evolutionary computing. Berlin: Springer, c 2003. ISBN 97835401841}
      \Orgbibdef{Michalewicz, Zbigniew. Genetic algorithms data structures = evolution programs. Berlin: Springer-Verlag, 1999. ISBN 3540606769}
      \Orgbibdef{Peitgen, Heinz-Otto; Jurgens, Hartmut; Saupe, Dietmar. Chaos and fractals: new frontiers of science. New York: Springer-Verlag, c1992. ISBN 0387979034}

      \Orgbibdef[basic]{Barone,Dante. Sociedades Artificiais - A nova fronteira da Inteligência nas Máquinas. Porto Alegre: Bookman, 2003. ISBN 8536301244}
      \Orgbibdef[basic]{Fogel, David B.. Evolutionary computation :toward a new philosophy of machine intelligence. New York: IEEE, c2000. ISBN 078035379X}
      \Orgbibdef[basic]{Jones, M. Tim. Artificial Intelligence :a systems approach. Usa: Jones and Bartlett Publishers, 2008. ISBN 9780763773373}
      \Orgbibdef[basic]{Mitchell,Melanie. An Introduction to genetic algorithms. Cambridge: Mit Press, c1996. ISBN 3540606769}


%%%%%%
%%%%%%
%
\classdef[Transv.outros]{ADM01135}{2}{ENGENHARIA ECONÔMICA E AVALIAÇÕES}

     \csummary{Introdução à engenharia econômica. Engenharia de avaliações. Projetos econômicos.}

      \Orgbibdef{Casarotto Filho, Nelson; Kopittke, Bruno Hartmut. Análise de investimentos :matemática financeira, engenharia econômica, tomada de decisão, estratégia empresarial. São Paulo: Atlas, 2007. ISBN 9788522448012}

      \Orgbibdef[basic]{Blank, Leland T.; Tarquin, Anthony. Engenharia econômica. [S.l.]: McGrawHill, 2008. ISBN 8577260267; 9788577260263}
      \Orgbibdef[basic]{Dal Zot, Wili Alberto Brancks. Matemática financeira. [Porto Alegre]: Ed. da UFRGS, [2008]. ISBN 9788570259943; 9788570259950 (CD)}
      \Orgbibdef[basic]{Gitman, Lawrence J.; Sanvicente, Antonio Zoratto. Princípios de administração financeira. São Paulo: Pearson/Addison Wesley, 2004. ISBN 8588639122; 9788588639126}
      \Orgbibdef[basic]{Hirschfeld, Henrique. Engenharia econômica e análise de custos. São Paulo: Atlas, 1998. ISBN 8522417970}
      \Orgbibdef[basic]{Ross, Stephen A.; Westerfield, Randolph W.; Jaffe, Jeffrey F.; Sanvicente, Antonio Zoratto. Administração financeira. São Paulo: Atlas, 2002. ISBN 8522429421; 9788522429424}

%%%%%%
%%%%%%
%
\classdef[Pro.ContProc]{ENG07043}{2}{INSTRUMENTAÇÃO DE PROCESSOS INDUSTRIAIS}

     \csummary{Fluxograma de engenharia, normas para descrever estratégias de controle de processos industriais. Principais estratégias de controle 
     utilizadas para controlar colunas de destilação, reatores químicos, trocadores de calor, fornos, biorreatores e demais processos usados nas 
     indústrias de processos. Utilização industrial de malhas de controle feedback, cascata e feedforward. Dimensionamento de válvulas de controle e 
     atuadores. Apresentação dos principais instrumentos de medição utilizados no cenário industrial. Medidores de temperatura, pressão, vazão, 
     nível e composição/analisadores. Descrição e quantificação dos erros de medição. Desenvolvimento de inferidores para acompanhar variáveis de 
     difícil medição.}


      \Orgbibdef{Mário Campos e Herbert Teixeira. Controles Típicos de Equipamentos e Processos Indsutriais. São Paulo, SP: Editora Edgar Blücher, 2006. ISBN 85-212-0398-5}

      \Orgbibdef[basic]{Bela G. Lipták. Instrument Engineers' Handbook : Process Control. USA: Chilton Book Co, 1995. ISBN 0-8019-8242-1}
      \Orgbibdef[basic]{Bela G. Lipták (Ed.). Instrument Engineers' Handbook : Process Measurement and Analysis. USA: Chilton, 1995. ISBN 0-8019-8197-2}

%%%%%%
%%%%%%
%
\classdef[Pro.ContProc]{ENG07012}{3}{LABORATÓRIO DE CONTROLE E OPERAÇÃO DE PROCESSOS}

     \csummary{Aulas práticas de laboratório, contemplando experimentos, coleta de dados e interpretação de resultados, em assuntos abordados nas disciplinas instrumentação da indústria química, cálculo de reatores e controle de processos.}

      \Orgbibdef{B.A. Ogunnaike, W. H. Ray. Process Dynamics, Modeling, and Control. New York, Oxford: Oxford University Press, 1994. ISBN 0195091191}
      \Orgbibdef{Campos, Mario Cesar M. Massa de; Teixeira, Herbert C.G.. Controles típicos de equipamentos e processos industriais. São Paulo: Edgard Blücher, 2006. ISBN 8521203985}
      \Orgbibdef{D.E. Seborg, T.F. Edgar, D.A. Mellichamp. Process Dynamics and Control. John Wiley, 2004. ISBN ISBN: 978-0-471-00077-8}

      \Orgbibdef[basic]{Luyben, William L.. Process modeling, simulation, and control for chemical engineers. Boston, Mass.: Mcgraw-Hill, c1990. ISBN 0070391599}
      \Orgbibdef[basic]{W.S. Levine (Ed.). The Control Handbook. CRC Press, 1996}

      \Orgbibdef[compl]{Foust, Alan S.; Wenzel, Leonard A.; Clump, Curtis W.; Maus, Louis; Andersen, L. Bryce. Princípios das operações unitárias. Rio de Janeiro: Guanabara Dois, c1980, 1982. ISBN 8521610386, 9788521610380}
      \Orgbibdef[compl]{Macintyre, Archibald Joseph. Bombas e instalações de bombeamento. Rio de Janeiro: LTC, c1997. ISBN 8521610866; 9788521610861}
      \Orgbibdef[compl]{Mccabe, Warren L.; Smith, Julian C.; Harriott, Peter. Unit operations of chemical engineering. New York, N.Y.: McGraw-Hill, c2005. ISBN 0072848235}
      \Orgbibdef[compl]{Perry, Robert H.; Green, Don W.. Perry's chemical engineers handbook. New York: McGraw-Hill, 2007. ISBN 0071422943}
      \Orgbibdef[compl]{Welty, James R.; Wicks, Charles E.. Fundamentals of momentum, heat and mass transfer. New York: John Wiley, 2007. ISBN 0470128682}

%%%%%%
%%%%%%
%
\classdef[Transv.outros]{EDU03071}{2}{LÍNGUA BRASILEIRA DE SINAIS (LIBRAS)}

     \csummary{Aspectos linguísticos da Língua Brasileira de Sinais (LIBRAS). História das comunidades surdas, da cultura e das identidades surdas. Ensino básico da LIBRAS. Políticas linguísticas e educacionais para surdos.}

      \Orgbibdef{CAMPELLO, Ana Regina, REZENDE, Patrícia Luiza Ferreira. Em defesa da escola bilíngue para surdos: a história de lutas do movimento surdo brasileiro. Curitiba: Educ. rev. [online], 2014}
      \Orgbibdef{GESSER, Audrei. Libras? Que língua é essa?:Crenças e preconceitos em torno da língua de sinais e da realidade surda. São Paulo: Parábola, 2009. ISBN 9788579340017}
      \Orgbibdef{QUADROS, Ronice Muller; KARNOPP, Lodenir Becker. Língua de Sinais Brasileira. Porto Alegre: Artmed, 2004. ISBN 9788536303086}

      \Orgbibdef[basic]{BRASIL. Decreto nº 5.626, de 22 de dezembro de 2005. Regulamenta a Lei nº 10.436, de 24 de abril de 2002, que dispõe sobre a Língua Brasileira de Sinais - Libras, e o art. 18 da Lei nº 10.098, de 19 de dezembro de 2000. Brasília: Diário Oficial da União, 2005}
      \Orgbibdef[basic]{BRASIL. Lei Federal 12.319, de 1º de setembro de 2010 - Regulamenta a profissão de Tradutor e Intérprete da Língua Brasileira de Sinais - LIBRAS. Brasília: Diário Oficial da União, 2010}
      \Orgbibdef[basic]{BRASIL. Lei nº 10.436, de 24 de abril de 2002. Dispõe sobre a Língua Brasileira de Sinais-Libras e dá outras providências. Brasília: Diário Oficial da União, 2002}
      \Orgbibdef[basic]{BRASIL. Lei nº 13.146, de 6 de julho de 2015. Institui a Lei Brasileira de Inclusão da Pessoa com Deficiência (Estatuto da Pessoa com Deficiência). Brasília: Diário Oficial da União, 2015}
      \Orgbibdef[basic]{KARNOPP, Lodenir Becker. Língua de sinais brasileira: aspectos linguísticos. Porto Alegre: Moodle, 2014}
      \Orgbibdef[basic]{PONTIN, Bianca Ribeiro; ROSA, Emiliana Faria. Movimento, história e Educação de Surdos. Porto Alegre: Moodle, 2014}
      \Orgbibdef[basic]{PONTIN, Bianca Ribeiro; ROSA, Emiliana Faria. Surdos. Porto Alegre: Moodle, 2014}

      \Orgbibdef[compl]{GOMES, Anie Pereira Goularte. O que significa essa tal de `cultura surda'? In: GOMES, Anie Pereira Goularte; HEINZELMANN, Renata Ohlson (orgs.). Cadernos Conecta Libras. Rio de Janeiro: Editora Arara Azul, 2015}
      \Orgbibdef[compl]{KARNOPP, Lodenir Becker. Produções culturais de surdos: análise da literatura surda. In: Cadernos de Educação | FaE/PPGE/UFPel. Pelotas: UFPel, 2010}
      \Orgbibdef[compl]{THOMA, Adriana da Silva; KLEIN, Madalena. Experiências educacionais, movimentos e lutas surdas como condições de possibilidade para uma educação de surdos no Brasil. Cadernos de Educação. FaE/PPGE/UFPel. Pelotas: UFPel, 2010}
      \Orgbibdef[compl]{THOMA, Adriana da Silva; LOPES, Maura Corcini (Org). A invenção da surdez:cultura, identidade, identidades e diferença no campo da educação. Santa Cruz do Sul (RS): EDUNISC, 2004. ISBN 8575780794}



%%%%%%
%%%%%%
%
\classdef[Pro.Control]{ENG07062}{3}{OTIMIZAÇÃO APLICADA}

     \csummary{Os métodos de programação matemática (métodos de otimização) são apresentados aplicados à solução de diferentes classes de problemas, 
     tais como: síntese de processos, programação de produção e logística, estimação de parâmetros, otimização em tempo real, controle preditivo, 
     entre outras aplicações encontradas comumente na engenharia. O curso inicia com a revisão de conceitos básicos de otimização, tais como: 
     critérios de optimalidade, convexidade, linearidade, continuidade, etc. A seguir as diversas técnicas empregadas para resolver as diferentes 
     formulações de problemas de otimização são apresentadas, segundo a seguinte classificação comumente adotada: a) programação não linear (NLP) 
     com e sem restrições; b) programação linear (LP); c) programação quadrática (QP); d) Programação inteira mista linear (MILP); e) programação 
     inteira mista não linear (MINLP); f) programação dinâmica e g) otimização global. Cada uma dessas técnicas é apresentada tendo como ponto de partida uma aplicação real.}

      \Orgbibdef{Edwin K.P. Chong e Stanislaw H. Zak. An Introduction to Optimization. Wiley, 2001. ISBN 0-471-39126-3}
      \Orgbibdef{Michael Bartholomew - Biggs. Nonlinear Optimization with Engineering Applications. Springer, 2008. ISBN 978-0-387-78722-0}
      \Orgbibdef{Nocedal, Jorge, Wright, Stephen. Numerical Optimization. Springer, 2006. ISBN 978-0-387-40065-5}

      \Orgbibdef[basic]{Andreas Antoniou e Wu-Sheng Lu. Pratical Optimization -- Algorithms and Engineering Applications. Springer, 2007. ISBN 0-387-71106-6}
      \Orgbibdef[basic]{Edgar, Himmelblau. Optimization of Chemical Processes. McGraw-Hill, 2001. ISBN 0-07-039359-1}
      \Orgbibdef[basic]{Eligius M.T. Hendrix e Boglárka G. Tóth. Introduction to Nonliear and Global Optimization. Springer, 2010. ISBN 978-0-387-88669-5}
      \Orgbibdef[basic]{Lorenz T. Biegler. Nonlinear Programming: Concepts, Algorithms, and Applications to Chemical Processes. SIAM, 2010. ISBN 978-0-898717-02-0}

      \Orgbibdef[compl]{Frank Neumann e Carsten Witt. Bioinspired Computation in Combinatorial Optimization. Springer, 2010. ISBN 978-3-642-16543-6}
      \Orgbibdef[compl]{Stefan Schäffler. Global Optimization: A Stochastic Approach. Springer, 2012. ISBN 978-1-4614-3926-4}

