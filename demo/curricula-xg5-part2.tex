%%%%%%%%%%%%%%%%%
%%%%%%%%%%%%%%%%%
%%%
%%%  Etapa 09
%%%
%%%%%%%%%%%%%%%%%
%%%%%%%%%%%%%%%%%



%%%%%%
%%%%%%
%
\classdef[Pro.ContProc]{ENG07087}{3}{CONTROLE AVANÇADO DE PROCESSOS}

     \csummary{Introdução à análise e controle de sistemas MIMO. Técnicas de controle avançado de processos: Controle preditivo baseado em modelo (MPC). Estimadores de estados: Filtro de Kalman e Filtro de Kalman Estendido.}

%%%%%%
%%%%%%
%
\classdef[Transv.integ]{TCC/CCA - I}{2}{TRABALHO DE CONCLUSÃO DE CURSO / CCA - I}

     \csummary{Tema de livre escolha do aluno dentro do ramo da Engenharia de Controle e Automação. Cada aluno terá um professor orientador e o trabalho final será examinado por professores que atuam na parte profissionalizante e específica do curso. O trabalho consistirá de uma monografia preliminar, propondo, contextualizando e delineando um plano de solução para um problema de Engenharia de Controle e Automação, a ser completado até o final da atividade de TCC/CCA - II, e deverá consistir minimamente da pesquisa bibliográfica e estado da arte do problema proposto, bem como a análise de viabilidade técnica/econômica da solução pretendida.}

%%%%%%   
%%%%%%
%
\classdef[Pro.Automacao]{ENG10021}{4}{SISTEMAS A EVENTOS DISCRETOS}

     \csummary{Sistemas a eventos discretos: conceituação, propriedades. Redes de Petri: conceitos básicos e aplicações na modelagem e controle de sistemas a eventos discretos. Teoria de autômatos: modelos de autômatos e aplicações ao controle de sistemas a eventos discretos. Sistemas de supervisão: conceituação aplicações em sistemas de automação.}

      \bibdef{cardoso1990}
      \bibdef{cassandras2010}
      \bibdef{miyagi1996}
      \bibdef[basic]{cassandras1993}
      \bibdef[basic]{aguirre2007b}
      \bibdef[basic]{reisig2013}
      \bibdef[basic]{seatzu2013}
      \bibdef[basic]{zhou2007}
      \bibdef[compl]{hein2010}
      \bibdef[compl]{banks2005}
      \bibdef[compl]{jensen2009}
      \bibdef[compl]{popova2013}

%%%%%%  
%%%%%%
%
\classdef[Pro.Maquinas]{ENG03047}{4}{PROJETOS DE SISTEMAS MECÂNICOS}

     \csummary{Princípios de projeto de um sistema mecânico. Estudo de problemas do projeto mecânico em geral. Aplicações em diversas áreas com ênfase em: controle e supressão de vibrações; avaliação e seleção de atuadores e sensoriamento para robôs; cinemática e controle de trajetória de robôs.}

      \bibdef{siciliano2009}
      \bibdef{pahl2007}
      \bibdef[basic]{nwokah2002}
      \bibdef[compl]{dorf1994}
      \bibdef[compl]{rao2008}
      \bibdef[compl]{spong2005}

%%%%%% 
%%%%%%
%
\classdef[Pro.Maquinas]{ENG10050}{2}{LABORATÓRIO DE MÁQUINAS E ACIONAMENTOS}

     \csummary{Ferramentas de análise de campos elétricos e magnéticos. Características operacionais da máquina CC. Características operacionais de máquina CA. Controle de velocidade de máquinas elétricas. Acionamentos usando conversores estáticos.}

      \bibdef{fitzgerald2014}
      \bibdef{bim2014}
      \bibdef{kostenko1979}
      \bibdef[basic]{rezek2011}
      \bibdef[basic]{white1959}
      \bibdef[basic]{mohan2015}

%%%%%% 
%%%%%%
%
\classdef[Pro.Maquinas]{ENG10046}{2}{PRINCÍPIOS DE ELETRÔNICA DE POTÊNCIA}

     \csummary{Princípios de operação e componentes básicos de conversores estáticos. Conversores CA-CC (retificadores controlados e não-controlados), CA-CA, CC-CC, CC-CA (inversores, tipos de modulação). Aplicação de conversores para acionamento de máquinas elétricas.}

      \bibdef{vithayathil1995}
      \bibdef{rashid2015}
      \bibdef{mohan2003}
      \bibdef[basic]{bose1997}
      \bibdef[basic]{aburub2014}
      \bibdef[basic]{murphy1988}
      \bibdef[basic]{martins2000}
      \bibdef[basic]{martins2005}
      \bibdef[basic]{sen1989}
      \bibdef[compl]{barbi2006}
      \bibdef[compl]{elhawary2002}
      \bibdef[compl]{bollen2011}

%%%%%%%%%%%%%%%%%
%%%%%%%%%%%%%%%%%
%%%
%%%  Etapa 10
%%%
%%%%%%%%%%%%%%%%%
%%%%%%%%%%%%%%%%%



%%%%%%
%%%%%%

\classdef[Transv.outros]{ENG03010}{3}{CIÊNCIA, TECNOLOGIA E AMBIENTE}

     \csummary{Ecologia: conceitos básicos. A biosfera e seu equilíbrio, desenvolvimento sustentável. Ciência e tecnologia: conceitos básicos, efeitos da tecnologia sobre o equilíbrio ambiental, tecnologia e desenvolvimento sócio-econômico. O ambiente industrial, legislação ambiental brasileira, a preservação dos recursos naturais, aspectos internos e externos do ambiente industrial, geração e o impacto de resíduos (sólidos, líquidos e pastosos) industriais, o tratamento e disposição final dos resíduos industriais, planejamento ambiental da atividade industrial.}

      \bibdef{macedo2000}
      \bibdef{vesilind1991}
      \bibdef[basic]{ipt2000}
      \bibdef[basic]{aisse1982}
      \bibdef[basic]{azevedo1977}
      \bibdef[basic]{lima1991}
      \bibdef[basic]{menegat2006}
      \bibdef[basic]{valle1996}
      \bibdef[compl]{campbell1995}
      \bibdef[compl]{ely1990}
      \bibdef[compl]{franciss1980}
      \bibdef[compl]{jacobi1989}
      \bibdef[compl]{knijnik1994}
      \bibdef[compl]{mandelli1991}
      \bibdef[compl]{mota1981}

%%%%%%
%%%%%%
%
\classdef[Transv.outros]{ENG03048}{4}{GERÊNCIA E ADMINISTRAÇÃO DE PROJETOS}

     \csummary{Idéias, técnicas e metodologias avançadas para o planejamento, controle e desenvolvimento de projetos de sistemas. Apresentação de um processo disciplinado e estruturado de administração de projetos de sistemas, segundo uma visão de negócio, de forma a cumprir prazos, orçamentos e requisitos. Exploração dos principais componentes do processo de gerenciamento de projetos nas organizações, fornecendo ferramental para projetar, avaliar e medir a efetividade e os fatores de risco da implementação de projetos. Técnicas de elaboração de estimativas de custos, prazos e recursos nos projetos de desenvolvimento de sistemas. Controle e garantia da qualidade no desenvolvimento de sistemas.}

      \bibdef{carvalho2008}
      \bibdef{pmi2000}
      \bibdef[basic]{kerzner2006}
      \bibdef[basic]{carvalho2006}
      \bibdef[basic]{rabechini2009}

%%%%%%
%%%%%%
%
\classdef[Transv.integ]{TCC/CCA - II}{2}{TRABALHO DE CONCLUSÃO DE CURSO / CCA - II}

     \csummary{Tema de livre escolha do aluno dentro do ramo da Engenharia de Controle e Automação, continuação do trabalho iniciado em TCC/CCA - I. Cada aluno, sob orientação de um professor, deverá concluir a análise iniciada em TCC/CCA - I, desenvolvendo e implementando a solução do problema proposto. A solução será documentada sob a forma de monografia, a ser apresentada perante uma banca de professores que atuam na parte profissionalizante e específica do curso.} 