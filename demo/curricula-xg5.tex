
%%%%%%%%%%%%%%%%%
%%%%%%%%%%%%%%%%%
%%%
%%%  Etapa 09
%%%
%%%%%%%%%%%%%%%%%
%%%%%%%%%%%%%%%%%



%%%%%%
%%%%%%
%
\classdef[Pro.ContProc]{ENG07087}{3}{CONTROLE AVANÇADO DE PROCESSOS}

     \csummary{Introdução à análise e controle de sistemas MIMO. Técnicas de controle avançado de processos: Controle preditivo baseado em modelo (MPC). Estimadores de estados: Filtro de Kalman e Filtro de Kalman Estendido.}



%%%%%%
%%%%%%
%
\classdef[Transv.integ]{TCC/CCA - I}{2}{TRABALHO DE CONCLUSÃO DE CURSO / CCA - I}

     \csummary{Tema de livre escolha do aluno dentro do ramo da Engenharia de Controle e Automação. Cada aluno terá um professor orientador e o 
     trabalho final será examinado por  professores que atuam na parte profissionalizante e específica do curso. O trabalho consistirá de uma 
     monografia preliminar, propondo, contextualizando e delineando um plano de solução para um problema de Engenharia de Controle e Automação, a 
     ser completado até o final da atividade de TCC/CCA - II, e deverá consistir minimamente da pesquisa bibliográfica e estado da arte do problema 
     proposto, bem como a análise de viabilidade técnica/econômica da solução pretendida.}




%%%%%%   
%%%%%%
%
\classdef[Pro.Automacao]{ENG10021}{4}{SISTEMAS A EVENTOS DISCRETOS}

     \csummary{Sistemas a eventos discretos: conceituação, propriedades. Redes de Petri: conceitos básicos e aplicações na modelagem e controle de 
     sistemas a eventos discretos. Teoria de autômatos: modelos de autômatos e aplicações ao controle de sistemas a eventos discretos. Sistemas de supervisão: conceituação aplicações em sistemas de automação.}


      \Orgbibdef{CARDOSO, J.; VALETTE, R.;. Redes de Petri}
      \Orgbibdef{Christos G. Cassandras, Stéphane Lafortune.. Introduction to Discrete Event Systems. Springer, 2010. ISBN 978-1441941190}
      \Orgbibdef{Paulo Eigi Miyagi. Controle Programável: fundamentos do controle de sistemas a eventos discretos. Blucher, 1996. ISBN 852120079x}

      \Orgbibdef[basic]{Introduction to Discrete Event Systems. Discrete Event Systems: modeling and performance analysis. Irwin, 1993. ISBN 0-256-11212-6}
      \Orgbibdef[basic]{Luis Antonio Aguirre. Enciclopédia de Automática - Vol.1. Blucher, 2007. ISBN 9788521204084}
      \Orgbibdef[basic]{Reisig, Wolfgang. Understanding Petri Nets : Modeling Techniques, Analysis Methods, Case Studies. Heidelberg: Springer Berlin, 2013. ISBN 9783642332784, 9783642332777}
      \Orgbibdef[basic]{Seatzu, Carla. Control of Discrete-Event Systems : Automata and Petri Net Perspectives. London: Springer, 2013. ISBN 9781447142768}
      \Orgbibdef[basic]{Zhou, MengChu. Modeling and Control of Discrete-event Dynamic Systems : with Petri Nets and Other Tools. London: Springer-Verlag, 2007. ISBN 9781846288777}

      \Orgbibdef[compl]{James L. Hein. Discrete Structures, Logic and Computability. Jones and Bartlett, 2010. ISBN 9780763772062}
      \Orgbibdef[compl]{Jarry Banks,. Discrete-event System Simulation. Prentice Hall, ISBN 978-0131446793}
      \Orgbibdef[compl]{Jensen, Kurt. Coloured Petri Nets : Modelling and Validation of Concurrent Systems. Heidelberg: Springer-Verlag Berlin, 2009. ISBN 9783642002847}
      \Orgbibdef[compl]{Popova-Zeugmann, Louchka. Time and Petri Nets. Heidelberg: Springer Berlin, 2013. ISBN 9783642411151, 9783642411144}


%%%%%%  
%%%%%%   
%
\classdef[Pro.Maquinas]{ENG03047}{4}{PROJETOS DE SISTEMAS MECÂNICOS}

     \csummary{Princípios de projeto de um sistema mecânico. Estudo de problemas do projeto mecânico em geral. Aplicações em diversas áreas com 
     ênfase em: controle e supressão de vibrações; avaliação e seleção de atuadores e sensoriamento para robôs; cinemática e controle de trajetória de robôs.}


      \Orgbibdef{Bruno Siciliano, Lorenzo Sciavicco, Luigi Villani, Giuseppe Oriolo. Robotics: Modelling, Planning and Control. London: Springer-Verlag, 2009. ISBN 978-1-84628-642-1}
      \Orgbibdef{PAHL, G.; BEITZ, W. ; FELDHUSEN, J.; GROTE K.-H.. Engineering Design: A Systematic Approach. Springer, 2007. ISBN 9781846283185}

      \Orgbibdef[basic]{NWOKAH, O. D. I.; HURMUZLU, Y.. The Mechanical System Design Handbook: Modeling, Measurement and Control. Boca Raton: CRC Press, 2002. ISBN 0849385962}

      \Orgbibdef[compl]{DORF, R. C.; KUSIAC, A.. HANDBOOK OF DESIGN, MANUFACTURING AND AUTOMATION. Wiley-Interscience, 1994. ISBN 0471552186}
      \Orgbibdef[compl]{RAO, S.. Vibrações Mecânicas. Pearson Prentice Hall, 2008. ISBN 978-85-7605-200-5}
      \Orgbibdef[compl]{SPONG, M. W.; HUTCHINSON, S.; VIDYASAGAR, M.. Robot Modeling and Control. Wiley, 2005. ISBN 978-0-471-64990-8}

%%%%%% 
%%%%%%
%
\classdef[Pro.Maquinas]{ENG10050}{2}{LABORATÓRIO DE MÁQUINAS E ACIONAMENTOS}

     \csummary{Ferramentas de análise de campos elétricos e magnéticos. Características operacionais da máquina CC. Características operacionais de máquina CA. Controle de velocidade de máquinas elétricas. Acionamentos usando conversores estáticos.}


      \Orgbibdef{A. E. Fitzgerald, C . Kingsley Jr, S. D. Umans. Máquinas Elétricas. Porto Alegre: Bookman, 2014. ISBN 9788580553741}
      \Orgbibdef{Edson Bim. Máquinas Elétricas e Acionamento. Brasil: Elsevier - Campus, 2014. ISBN 85-352-7713-7}
      \Orgbibdef{M. Kostenko, L. Piotrovski. Máquinas Elétricas. Portugal: Lopes da Silva, 1979}

      \Orgbibdef[basic]{ANGELO JOSE JUNQUEIRA REZEK. FUNDAMENTOS BASICOS DE MAQUINAS ELETRICAS: TEORIA E ENSAIOS. Brasil: Synergia, 2011. ISBN 8561325690}
      \Orgbibdef[basic]{D. C . White, H. H. Woodson. Electromechanical Energy Conversion. EUA: Wiley, 1959}
      \Orgbibdef[basic]{Ned Moham. Máquinas Elétricas E Acionamentos: Curso Introdutório. Brasil: LTC, 2015. ISBN 8521627629}

%%%%%% 
%%%%%%
%
\classdef[Pro.Maquinas]{ENG10046}{2}{PRINCÍPIOS DE ELETRÔNICA DE POTÊNCIA}

     \csummary{Princípios de operação e componentes básicos de conversores estáticos. Conversores CA-CC (retificadores controlados e não-controlados), CA-CA, CC-CC, CC-CA (inversores, tipos de modulação). Aplicação de conversores para acionamento de máquinas elétricas.}


      \Orgbibdef{Joseph Vithayathil. Power electronics : principles and applications.. McGraw-Hill, 1995. ISBN 0070675554}
      \Orgbibdef{M. H. Rashid. Eletrônica de potência : dispositivos, circuitos e aplicações.. Pearson Education do Brasil, 2015. ISBN 9788543005942}
      \Orgbibdef{Ned Mohan. Power electronics : converters, applications, and design.. John Wiley, 2003. ISBN 9780471226932}

      \Orgbibdef[basic]{B. K. Bose. Power electronics and variable frequency drives : technology and applications.. IEEE Press, 1997. ISBN 0780310845}
      \Orgbibdef[basic]{Haitham Abu-Rub, Mariusz Malinowski, Kamal Al-Haddad. Power Electronics for Renewable Energy Systems, Transportation and Industrial Applications. New York: Wiley-IEEE Press, 2014. ISBN 978-1-118-63403-5}
      \Orgbibdef[basic]{J. M. D. Murphy. Power electronic control of AC motors.. Pergamon, 1988. ISBN 0080226833}
      \Orgbibdef[basic]{Martins, D. C.; Barbi, Ivo. Eletrônica de potência : conversores CC-CC básicos não isolados. Florianópolis: Florianópolis Ed. do Autor 2000., 2000. ISBN 859010463X}
      \Orgbibdef[basic]{Martins, D. C.; Barbi, Ivo. Introdução ao Estudo dos Conversores CC-CA. Florianópolis: INEP, 2005. ISBN 9788590520313}
      \Orgbibdef[basic]{P. C. Sen. Principles of electric machines and power electronics.. John Wiley, 1989. ISBN 047185845}

      \Orgbibdef[compl]{Barbi, Ivo. Eletrônica de Potência. Florianópolis: Do autor, 2006}
      \Orgbibdef[compl]{M. E. El-Hawary. Principles of electric machines with power electronic applications.. IEEE Press, 2002. ISBN 0471208124}
      \Orgbibdef[compl]{Math H. J. Bollen,? Fainan Hassan. Integration of Distributed Generation in the Power System. New York: IEEE Computer Society Press, 2011. ISBN 978-0470643372}


%%%%%%%%%%%%%%%%%
%%%%%%%%%%%%%%%%%
%%%
%%%  Etapa 10
%%%
%%%%%%%%%%%%%%%%%
%%%%%%%%%%%%%%%%%



%%%%%%
%%%%%%

\classdef[Transv.outros]{ENG03010}{3}{CIÊNCIA, TECNOLOGIA E AMBIENTE}

     \csummary{Ecologia: conceitos básicos. A biosfera e seu equilíbrio, desenvolvimento sustentável. Ciência e tecnologia: conceitos básicos, 
     efeitos da tecnologia sobre o equilíbrio ambiental, tecnologia e desenvolvimento sócio-econômico. O ambiente industrial, legislação ambiental 
     brasileira, a preservação dos recursos naturais, aspectos internos e externos do ambiente industrial, geração e o impacto de resíduos (sólidos, 
     líquidos e pastosos) industriais, o tratamento e disposição final dos resíduos industriais, planejamento ambiental da atividade industrial.}

      \Orgbibdef{Ricardo Kohn de Macedo. Gestão Ambiental. rio de janeiro: ABES, ISBN 8570221169}
      \Orgbibdef{Vesilind, P. Aarne/ Morgan, Susan M.. Introdução a Engenharia Ambiental. São Paulo: Cengage Learning, 1991. ISBN 978-85-221-0718-6}

      \Orgbibdef[basic]{Lixo municipal :manual de gerenciamento integrado. São Paulo: IPT, 2000. ISBN 8509001138}
      \Orgbibdef[basic]{Aisse, Miguel Mansur; Obladen, Nicolau Leopoldo. Tratamento de esgotos por biodigestão anaeróbia. [Curitiba: CNPq, 1982?]}
      \Orgbibdef[basic]{Azevedo Netto, Jose Martiniano de. Sistemas de esgotos sanitários. São Paulo: Companhia de Tecnologia de Saneamento Ambiental, 1977}
      \Orgbibdef[basic]{Lima, Luiz Mario Queiroz. Tratamento de lixo. Sao Paulo: Hemus, 1991}
      \Orgbibdef[basic]{Menegat, Rualdo; Porto, Maria Luiza; Carraro, Clóvis Carlos; Fernandes, Luís Alberto D'Ávila. Atlas ambiental de Porto Alegre. Porto Alegre: Editora da Universidade/UFRGS, 2006. ISBN 8570259123}
      \Orgbibdef[basic]{Valle, Cyro Eyer do. Como se preparar para as normas iso 14000 :qualidade ambiental. Sao Paulo: Pioneira, 1996. ISBN 8522100101}

      \Orgbibdef[compl]{Campbell, Stu. Manual de compostagem para hortas e jardins :como aproveitar o lixo organico domestico. Sao Paulo: Nobel, 1995. ISBN 8521308868}
      \Orgbibdef[compl]{Ely, Aloisio. Economia do meio ambiente :uma apreciação introdutória interdisciplinar da poluição, ecologia e qualidade ambiental. Porto Alegre: FEE, 1990}
      \Orgbibdef[compl]{Franciss, Fernando Olavo. Hidráulica de meios permeáveis :escoamento em meios porosos. Rio de Janeiro: Interciência, 1980}
      \Orgbibdef[compl]{Jacobi, Pedro Roberto. Movimentos sociais e políticas públicas :demandas por saneamento básico e saúde, São Paulo 1974-84. São Paulo: Cortez, 1989. ISBN 8524901713 (broch.)}
      \Orgbibdef[compl]{Knijnik, Roberto. Energia e meio ambiente em Porto Alegre :bases para o desenvolvimento. Porto Alegre: Dmae, 1994}
      \Orgbibdef[compl]{Mandelli, Suzana Maria de Conto; Lima, Luiz Mario Queiroz; Ojima, Mário K.. Tratamento de resíduos sólidos:compêndio de publicações. Caxias do Sul: Ed. do Autor, 1991}
      \Orgbibdef[compl]{Mota, Suetonio. Planejamento urbano e preservacao ambiental. Fortaleza: Ufc, 1981}

%%%%%%
%%%%%%
%
\classdef[Transv.outros]{ENG03048}{4}{GERÊNCIA E ADMINISTRAÇÃO DE PROJETOS}

     \csummary{Idéias, técnicas e metodologias avançadas para o planejamento, controle e desenvolvimento de projetos de sistemas. Apresentação de um processo disciplinado e estruturado de administração de projetos de sistemas, segundo uma visão de negócio, de forma a cumprir prazos, 
     orçamentos e requisitos. Exploração dos principais componentes do processo de gerenciamento de projetos nas organizações, fornecendo ferramental para projetar, avaliar e medir a efetividade e os fatores de risco da implementação de projetos. Técnicas de elaboração de 
     estimativas de custos, prazos e recursos nos projetos de desenvolvimento de sistemas. Controle e garantia da qualidade no desenvolvimento de sistemas.}

      
      \Orgbibdef{Marly Monteiro de Carvalho e Roque Rabechini Jr.. - CONSTRUINDO COMPETÊNCIAS PARA GERENCIAR PROJETOS: Teoria e Casos. São Paulo, SP: Editora Atlas, 2008. ISBN 9788522449248}
      \Orgbibdef{Project Management Institute. A Guide to the Project Management Body of Knowledge (PMBOK Guide). Pensilvania, EUA: Project Management Institute, 2000. ISBN 1880410230}

      \Orgbibdef[basic]{Harold Kerzner. Gestão de Projetos: as Melhores Práticas. BOOKMAN, 2006. ISBN 8536306181}
      \Orgbibdef[basic]{Marly Monteiro de Carvalho e Roque Rabechini Jr.. GERENCIAMENTO DE PROJETOS NA PRÁTICA: Casos Brasileiros - v. 1. Atlas, 2006. ISBN 9788522445233}
      \Orgbibdef[basic]{Roque Rabechini Jr. e Marly Monteiro de Carvalho. GERENCIAMENTO DE PROJETOS NA PRÁTICA: Casos Brasileiros - v. 2. Atlas, 2009. ISBN 9788522456987}



%
\classdef[Transv.integ]{TCC/CCA - II}{2}{TRABALHO DE CONCLUSÃO DE CURSO / CCA - II}

     \csummary{Tema de livre escolha do aluno dentro do ramo da Engenharia de Controle e Automação, continuação do trabalho iniciado em TCC/CCA - I. 
     Cada aluno, sob orientação de um professor, deverá concluir a análise iniciada em TCC/CCA - I, desenvolvendo e implementando a solução do 
     problema proposto. A solução será documentada sob a forma de monografia, a ser apresentada perante uma banca de professores que atuam na parte 
     profissionalizante e específica do curso.}


