
%%%%%%
%%%%%%
%
\classdef[Transv.outros]{ENG09023}{2}{PLANEJAMENTO ESTRATÉGICO DA PRODUÇÃO}

     \csummary{Administrar estrategicamente é um processo contínuo e interativo que visa manter a organização como um conjunto apropriadamente 
     integrado ao seu ambiente. Não é mais suficiente gerenciar a organização como um objeto específico; é preciso gerenciar o negócio da 
     organização, envolvendo fatores, influências, recursos e variáveis externas e internas, buscando competitividade. Com o planejamento 
     estratégico, não se pretende adivinhar o futuro. O intuito é traçar objetivos futuros viáveis e propor ações para alcançá-los. Na disciplina 
     são discutidos Negócio, Missão e Princípios organizacionais, Análise do Ambiente e identificação de oportunidades e ameaças, Definição de Visão 
     e objetivos a serem alcançados, além da Definição de Estratégias para atingir os objetivos, com ênfase na discussão de estratégias de produção.}

      \Orgbibdef{Müller, Cláudio J.. Modelo de Gestão Integrando Planejamento Estratégico, Sistemas de Avaliação de Desempenho e Gerenciamento de Processos (MEIO: Modelo de Estratégia, Indicadores e Operações). 2003}

      \Orgbibdef[compl]{Ansoff, H. Igor; Declerck, Roger P.; Hayes, Robert L.. Do planejamento estratégico a administração estratégica. São Paulo: Atlas, 1990, c1985}
      \Orgbibdef[compl]{Ansoff, H. Igor; Mcdonnell, Edward J.. Implantando a administração estratégica. São Paulo: Atlas, 1993. ISBN 8522409544}
      \Orgbibdef[compl]{Campos, Vicente Falconi. Gerenciamento pelas diretrizes :(Hoshin Karin) : o que todo membro da alta administração precisa saber para entrar no terceiro milênio. Nova Lima: INDG, 2004. ISBN 8598254150}
      \Orgbibdef[compl]{Certo, Samuel C.; Peter, J. Paul; Marcondes, Reynaldo Cavalheiro; Cesar, Ana Maria Roux. Administração estratégica :planejamento e implantação da estratégia. São Paulo: Pearson Prentice Hall, 2005. ISBN 8576050250; 9788576050254}
      \Orgbibdef[compl]{Kaplan, Robert S.; Norton, David P.. A estratégia em ação. Rio de Janeiro: Campus/Elsevier, 1997. ISBN 8535201491; 9788535201499}
      \Orgbibdef[compl]{Kaplan, Robert S.; Norton, David P.. Organização orientada para estratégia :como as empresas que adotam o Balanced Scorecard prosperam no novo ambiente de negócios. Rio de Janeiro: Campus, c2001. ISBN 8535207090; 9788535207095}
      \Orgbibdef[compl]{Mintzberg, Henry; Ahlstrand, Bruce; Lampel, Joseph; Montingelli Júnior, Nivaldo; Rossi, Carlos Alberto Vargas. Safári de estratégia :um roteiro pela selva do planejamento estratégico. Porto Alegre: Bookman, 2000. ISBN 8573075414; 9788573075410}
      \Orgbibdef[compl]{Naisbitt, John. Megatendencias, asia :oito megatendencias asiaticas que estao transformando o mundo. Rio de Janeiro: Campus, 1997. ISBN 853520119X}
      \Orgbibdef[compl]{Oliveira, Djalma de Pinho Reboucas de. Excelência na administração estratégica :a competitividade para administrar o futuro das empresas : com depoimentos de executivos. São Paulo: Atlas, 1999. ISBN 8522423903}
      \Orgbibdef[compl]{Paiva, Ely Laureano; Carvalho Junior, Jose Mario de; Fensterseifer, Jaime Evaldo; Hayes, Robert H.. Estratégia de produção e de operações :conceitos, melhores práticas, visão de futuro. Porto Alegre: Bookman, 2009. ISBN 9788577804948}
      \Orgbibdef[compl]{Popcorn, Faith; Marigold, Lys. Click :16 tendências que irão transformar sua vida, seu trabalho e seus negócios no futuro. Rio de Janeiro: Campus, 1997. ISBN 8535201084}
      \Orgbibdef[compl]{Porter, Michael E.. Vantagem competitiva :criando e sustentando um desempenho superior. Rio de Janeiro: Campus, c1989. ISBN 8570015585; 9788570015587}
      \Orgbibdef[compl]{Porter, Michael E.; Braga, Elizabeth Maria de Pinho; Gomez, Jorge A. Garcia. Estratégia competitiva. Rio de Janeiro: Elsevier/Campus, c2004. ISBN 8535215263; 9788535215267}
      \Orgbibdef[compl]{Prahalad, C. K.; Hamel, Gary. Competindo pelo futuro :estratégias inovadoras para obter o controle do seu setor e criar os mercados de amanhã. Rio de Janeiro: Elsevier, c2005. ISBN 9788535215441; 8535215441}
      \Orgbibdef[compl]{Scott, Cynthia D.. Visão, valores e missão organizacional :construindo a organização do futuro. Rio de Janeiro: Qualitymark, 1998. ISBN 8573031891}
      \Orgbibdef[compl]{Slack, Nigel; Corrêa, Sônia Maria; Correa, Henrique Luiz. Vantagem competitiva em manufatura :atingindo competitividade nas operações industriais. São Paulo, SP: Atlas, 2002. ISBN 8522432600}
      \Orgbibdef[compl]{Thompson Jr., Artur A.. Administração estratégica. São Paulo: McGraw-Hill, 2008. ISBN 8586804908}
      \Orgbibdef[compl]{Valadares, Maurício C.B.. Planejamento estratégico empresarial: foco em clientes e pessoas. Rio de Janeiro: Qualitymark, 2002. ISBN 8573033274}
      \Orgbibdef[compl]{Vasconcellos Filho, Paulo de; Pagnoncelli, Dernizo. Construindo estratégias para vencer! :um método prático, objetivo e testado para o sucesso da sua empresa. Rio de Janeiro: Elsevier, 2001. ISBN 8535207678; 9788535207675}


%%%%%%
%%%%%%
%
\classdef[Avan.topic]{CCA99005}{2}{TÓPICOS ESPECIAIS EM ENGENHARIA DE CONTROLE E AUTOMAÇÃO I}

     \csummary{Disciplina com tema variado sempre dentro da área de Engenharia de Controle a Automação.}


%%%%%%
%%%%%%
%
\classdef[Avan.topic]{CCA99006}{4}{TÓPICOS ESPECIAIS EM ENGENHARIA DE CONTROLE E AUTOMAÇÃO II}

     \csummary{Disciplina com tema variado sempre dentro da área de Engenharia de Controle e Automação.}


%%%%%%
%%%%%%
%
\classdef[Avan.topic]{CCA99007}{6}{TÓPICOS ESPECIAIS EM ENGENHARIA DE CONTROLE E AUTOMAÇÃO III}

     \csummary{Disciplina com tema variado sempre dentro da área de Engenharia de Controle e Automação.}


%%%%%%
%%%%%%
%
\classdef[Pro.Robotica]{ENG10051}{4}{DINÂMICA E CONTROLE DE ROBÔS}

     \csummary{Modelagem dinâmica de robôs: modelos de Lagrange e Newton-Euter. Controle Independente por juntas. Controle de robôs: por toque calculado, no espaço cartesiano, por realimentação variante no tempo, por realimentação não-suave. Aspectos de implementação.}

      \Orgbibdef{Aaron Martinez and Enrique Fernández. Learning ROS for Robotics Programming. Birmingham, UK: Packt Publishing, 2013. ISBN 978-1-78216-144-8}
      \Orgbibdef{King Sun Fu and R. C. Gonzales and C. S. George Lee. Robotics Control, Sensing, Vision and Intelligence. McGraw-Hill, 1987. ISBN 0-07-022625-3}

      \Orgbibdef[basic]{Anis Koubaa. Robot Operating System (ROS): The Complete Reference. Switzerland: Springer International Publishing, ISBN 978-3-319-26052-5}
      \Orgbibdef[basic]{Jason M. O'Kane. A Gentle Introduction to ROS. CreateSpace Independent Publishing Platform, 2013. ISBN 978-1492143239}
      \Orgbibdef[basic]{Patrick Goebel. ROS by Example. Raleigh, NC: Lulu, 2013}


%%%%%%
%%%%%%
%
\classdef[Pro.Robotica]{ENG10052}{4}{LABORATÓRIO DE ROBÓTICA}

     \csummary{Ambientes de programação e simulação de robôs. Projeto e concepção de células robotizadas. Integração software-hardware do sistema robótico. Protocolos de comunicação como o robô. Normas de segurança.}

      \Orgbibdef{John J. Craig. Robótica. São Paulo: Pearson, 2012. ISBN 978-85-8143-128-4}
      \Orgbibdef{NOF, S. Y. Handbook of Industrial Robotics. John Wyley, ISBN 0-471-17783-0}

      \Orgbibdef[basic]{Sciavicco, L; Siciliano, B. Modelling and Control of Robot Manipulators. Springer, ISBN 1852332212}

      \Orgbibdef[compl]{Fu, K.S.; Gonzalez, R. C.; Lee, C.S.G.. Robotics Control, Sensing, Vision an Intelligence. McGraw-Hill, ISBN 0-07-022625-3}
      \Orgbibdef[compl]{Groover, M. P, Weiss, M., Nagel, R. N.; Odrey, N. G.. Industrial Robotics Technology, Programming and Applications. McGraw- Hill Inc, ISBN 0-07-024989}




%%%%%%
%%%%%%
%
\classdef[Pro.Robotica]{ENG10xxa}{4}{ROBÓTICA MÓVEL}

     \csummary{Comportamento não-holonômico, modelagem cinemática e dinâmica de robôs móveis, controle de robôs móveis, localização, mapeamento de ambiente, localização e mapeamento simultâneos, aspectos de implementação, plataformas para robótica móvel.}

      \Orgbibdef{LATOMBE, J.-C. Robot Motion Planning. Boston, MA: Kluwer Academic Publishers, 1991. n.124. (Kluwer International Series in Engineering and Computer Science).}
      \Orgbibdef{NEWMAN, W. A Systematic Approach to Learning Robot Programming with ROS. Boca Raton, FL: CRC Press, 2018.}
      \Orgbibdef{THRUN, S.; BURGARD, W.; FOX, D. Probabilistic Robotics. Cambridge, MA: MIT Press, 2005. (Intelligent Robotics and Autonomous Agents Series).}

      \Orgbibdef[basic]{FU, K. S.; GONZALES, R. C.; LEE, C. S. G. Robotics Control, Sensing, Vision and Intelligence. New York: McGraw-Hill, 1987. (Industrial Engineering Series).}
      \Orgbibdef[basic]{MURRAY, R. M.; LI, Z.; SASTRY, S. S. Mathematical Introduction to Robotic Manipulation.Boca Raton, FL: CRC Press, 1994.}

      \Orgbibdef[compl]{BROWN, R. G. Introduction to Random Signal Analysis and Kalman Filtering. New York: John Wiley \& Sons, 1983.}
      \Orgbibdef[compl]{EVERETT, H. R. Sensors for Mobile Robots: theory and application. Natick, MA: A. K. Peters, 1995.}

%%%%%%
%%%%%%
%
\classdef[Pro.Robotica]{ENG10xxB}{4}{SISTEMAS DE TEMPO REAL}

     \csummary{Caracterização de sistemas tempo-real. Sistemas operacionais tempo-real: métodos de escalonamento. Linguagens de programação para sistemas tempo-real.}

      \Orgbibdef{A. Burns and A. Wellings. Real-Time Systems and Programming Languages. Reading, MA: Addison-Wesley, 2001. ISBN 0201729881.}
      \Orgbibdef[basic]{Andrew S. Tanenbaum. Modern Operating Systems. Englewood-Clifs, NJ: Prentice-Hall, 2001. ISBN 0130313580.}
      \Orgbibdef[basic]{W. Richard Stevens , Stephen A. Rago. Advanced Programming in the UNIX Environment. Reading, MA: Addison-Wesley, 2005. ISBN 0201433079.}
      \Orgbibdef[compl]{Bjarne Stroustrup. The C Programming Language. Reading, MA: Addison-Wesley, 1997. ISBN 0-201-32755-4.}
      \Orgbibdef[compl]{Bjarne Stroustrup. The Design and Evolution of C. Reading, MA: Addision-Wesley, 1993. ISBN 0201543303.}
      \Orgbibdef[compl]{M. Ben-ari. Principles of Concurrent Programming. Reading, MA: Addison-Wesley, 2006. ISBN 0-321-31283-X.}


%%%%%%
%%%%%%
%
\classdef[Pro.Control]{ELE223}{4}{Sistemas Lineares – A}

     \csummary{Fundamentos de álgebra linear. Sistemas dinâmicos lineares de tempo contínuo e de tempo discreto: definições e propriedades. 
     Representações entrada-saída: equações diferenciais, convolução, função de transferência. Representação por variáveis de estado. Realizações. 
     Análise de sistemas lineares e invariantes no tempo: estabilidade, controlabilidade e observalidade. Realimentação de estados. Observadores de 
     estado.}

      \Orgbibdef[basic]{Chen, C.T. - Linear System Theory and Design. 3ª edição, Oxford University Press, 1999}
      \Orgbibdef[basic]{Lipschutz, S. - Álgebra Linear. 2ª Edição, McGraw-Hill, 1978}
      \Orgbibdef[basic]{Strang, G. ? Linear Álgebra and its Applications, 3a edição, Harcourt Brace Jovanovich College Publishers, 1988}

      \Orgbibdef[compl]{Zadeh, L.A; Desoer, C.A. - Linear Systems Theory. Springer-Verlag, 1979}
      \Orgbibdef[compl]{Kailath, T. ? Linear Systems. Prentice Hall, Rnglewood Cliffs, N.J., 1980}
      \Orgbibdef[compl]{Haykin, S.; Van Veen, B. - Sinais e Sistemas. Bookman, Porto Alegre, 2001}
      \Orgbibdef[compl]{Boldrini, J.L.; Costa, S.I.R.; Figueiredo, V.L.; Wetzler, H.G. - Álbebra Linear. 3ª Edição, Harbra, 1984}
      \Orgbibdef[compl]{Steinbruch, A.; Winterle, P. - Álgebra Linear. McGraw-Hill, 1987}

%%%%%%
%%%%%%
%
\classdef[Pro.Control]{ELE312}{4}{Sistemas Não-Lineares – B}

     \csummary{Equilíbrios, ciclos-limite e atratores; definições de estabilidade; caracterização de domínios de atração. Método indireto de 
     Liapunov. Método direto de Liapunov em sistemas autônomos: funções de Liapunov, princípio de invariância, estimação de domínios de atração. 
     Estabilidade absoluta: critério do círculo, critério de Popov. Passividade; o lema positivo real.}

      \Orgbibdef[basic]{H. K. Khalilk. Nonlinear Systems. 2ª Edição, Prentice-Hall, 1996}
      \Orgbibdef[basic]{S. Sastry. Nonlinear Systems. Springer, 1999}

      \Orgbibdef[compl]{R. Sepulchre ans M. Jankovic and P. Kokotovic. Constructive Nonlinear Control. Springer, 1997}
      \Orgbibdef[compl]{R. Seydel. Practical bifurcation and stability analysis. Springer, 1994}
      \Orgbibdef[compl]{J. J. Slotine and W. Li. Applied Nonlinear Control. Prentice-Hall, 1991}
      \Orgbibdef[compl]{E. J. Davison and E. M. Kurak. A computacional method for determining quadratic Lyapunov functions for nonlinear systems. Automatica, 7:627-636, 1971}
      \Orgbibdef[compl]{R. Genesio and M. Tartaglia and A. Vicino. On the estimation of asymptotic stability regions: state of the art and nem proposals. IEEE Transactions on Automatic Control, 30:747-755, 1985}
      \Orgbibdef[compl]{H.-D. Chiang and IF.F Wu and P.P Varaya. Foundations of direct methods for power system transiente stability analysis. IEEE Transactions on Circuits and Systems, 34, 1987}
      \Orgbibdef[compl]{A. S. Bazanella and C. L. Conceicao. Transient stability improvement through excitation control. International Journal of Robust and Nonlinear Control, 14:891-910, 2004}
      \Orgbibdef[compl]{D. F. Coutinho and A. S. Bazanella amnd A. Trofino and A. Silveira e Silva. Stability analisys and controlo f a class of a differential-algebraiic nonlinear systems. International Journal of Robust and Nonlinear Control, 14:1301-1326, 2004}
      \Orgbibdef[compl]{A. S. Bazanella, A. S. e Silva, and P. V. Kokotovic. Lyapunov design of excitation control for synchronous machines. Decision and Control, 1997. Proceedings of the 36th IEEE Conference on, 1:211-216 vol. 1, 10-12 Dec 1997}
      \Orgbibdef[compl]{A. S. Bazanella, P. Kokotovic, and A.S. e Silva. A dynamics extension for LgV controllers. IEEE Transaction on Automatic Control, 44:588-592, 1999}
      \Orgbibdef[compl]{A. S. Bazanella and R. Reginatto. Robustness margins for indirect field oriented controlo f induction motors. IEEE Transactions on Automatic Control, 45:1226-1231, 2000}
      \Orgbibdef[compl]{Jr. Gomes da Silva, J. M., S. Tarbouriech, and R. Reginatto. Analysis of regions of
stability for linear systems with saturating inputs through na anti-windup scheme. Control Applications, 2002. Proceedings of the 2002 International Conference on, 2:1106-1111 vol. 2, 2002}
      \Orgbibdef[compl]{J. M. Gomes da Silva Jr. And M. Z. Oliveira and D. F. Coutinho and S. Tarbouriech. Static anti-windup design for a classe of nonlinear systems. International Journal of Robust and Nonlinear Control, 24:793:810, 2014}
      \Orgbibdef[compl]{M. G . LArios, R. Ortega, and A. S. Bazanella. An energy-shaping approach to the design of excitation controlo f synchronous generators. Automatica, 39:111-119, 2003}
      \Orgbibdef[compl]{R. Reginatto and A. S. Bazanella. Robust tuning of the speed loop in indirect field oriented control of induction motors. Automatic, 37:1811-1818, 2001}


%%%%%%
%%%%%%
%
\classdef[Pro.Control]{ELE216}{4}{ELE216 – Controle Multivariável}

     \csummary{Realimentação de estados. Realimentação estática e dinâmica de saída. Problema de seguimento de referência e rejeição de perturbações. Controle linear quadrático. Introdução ao controle robusto.}

      \Orgbibdef[basic]{C.T. Chen. Linear Systems Theory and Design. Holt, Rinehart and Wiston, 1984}
      \Orgbibdef[basic]{T. Kailath. Linear Systems. Prentice-Hall-Inc., 1980}
      \Orgbibdef[basic]{J.M. Maciejowski. Multivariable Feedback Design. Adison-Wesley, 1990}
      \Orgbibdef[basic]{H. Kwakernak, R. Sivan. Linear Optimal Control. Wiley, N.Y., 1972}
      \Orgbibdef[basic]{W.M. Wonham. Linear Multivariable Control, a geometric Approach. Springer-Verlag, 1979}
      \Orgbibdef[basic]{P. Dorato, C. Abdallah, V. Cerone. Linear-Quadratic Control - An Introduction. Prentice-Hall, 1995}
      \Orgbibdef[basic]{K. Zhou, J. Doyle, K. Glover. Robust and Optimal Control. Prentice-Hall-Inc}

%%%%%%
%%%%%%
%
\classdef[Pro.Control]{ELExxx}{4}{Processos Estocásticos}

     \csummary{to be written}




