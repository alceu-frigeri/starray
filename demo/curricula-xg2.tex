
%%%%%%%%%%%%%%%%%
%%%%%%%%%%%%%%%%%
%%%
%%%  Etapa 03
%%%
%%%%%%%%%%%%%%%%%
%%%%%%%%%%%%%%%%%



%%%%%%
%%%%%%
%
\classdef[BaseEletro]{ENG10001}{4}{CIRCUITOS ELÉTRICOS I - C}

     \csummary{Análise de circuitos resistivos. Quadripolos resistivos. Análise de circuitos de primeira e segunda ordem de domínio do tempo.}

      \Orgbibdef{  ALEXANDER, C. K., SADIKU, N.O.M.. Fundamentos de Circuitos Elétricos. Bookman, ISBN 8536302496}
      \Orgbibdef{NILSSON J. W., RIEDEL S. A.. Circuitos Elétricos Editora. LTC, ISBN 8521613636}

      \Orgbibdef[basic]{IRWIN D. J.. Análise Básica de Circuitos para Engenharia. LTC, ISBN 8534606935}
      \Orgbibdef[basic]{SCOTT, R.E.. Elements of Linear Circuits. Addison-Wesley, ISBN 0201068435}

      \Orgbibdef[compl]{DESOER, Charles A. e KUH, Ernest S.. Teoria Básica de Circuitos. Guanabara Dois, ISBN 0070165750}
      \Orgbibdef[compl]{DORF, R.C., SVODOBA, J. A.. Introdução aos Circuitos Elétricos. LTC, ISBN 9788521615828}
      \Orgbibdef[compl]{FOERSTER G., TREGNAGO R.. Circuitos Elétricos. Editora da Universidade, ISBN 8570251378}

%%%%%%
%%%%%%
%
\classdef[BaseMatematica]{MAT01167}{6}{EQUAÇÕES DIFERENCIAIS II}

     \csummary{Equações diferenciais ordinárias e lineares. Elementos de séries de Fourier, polinômios de Legendre e funções de Bessel. Equações diferenciais lineares a derivadas parciais (problemas de contorno: equações da Física Clássica).}

      \Orgbibdef{  C. H. Edwards Jr., D.E. Penney.. Equações Elementares com Problemas de Contorno. Rio de Janeiro: LTC, ISBN 9788570540577}
      \Orgbibdef{William E. Boyce, Richard C. DiPrima. Equações Diferenciais Elementares e Problemas de Valores de Contorno. LTC, 2015. ISBN 9788521627357}
      \Orgbibdef{Zill, Dennis G.. Equações diferenciais com aplicações em modelagem. São Paulo: Thomson, 2003. ISBN 8522103143; 9788522103140}

      \Orgbibdef[basic]{Eduardo Brietzke. Notas de aula de Equações Diferenciais II. Porto Alegre}

      \Orgbibdef[compl]{Asmar, Nakhle. Partial differential equations and boundary value problems. New Jersey: Prentice-Hall, c2005. ISBN 0131480960}
      \Orgbibdef[compl]{Boyce, William E.; DiPrima, Richard C.. Equações diferenciais elementares e problemas de valores de contorno. Rio de Janeiro: LTC, c2006. ISBN 8521614993}
      \Orgbibdef[compl]{Churchill, Ruel Vance; Carvalho, Carlos Alberto Aragao de. Series de Fourier e problemas de valores de contorno. Rio de Janeiro: Guanabara Dois, 1978}
      \Orgbibdef[compl]{Figueiredo, Djairo Guedes de. Análise de Fourier e equações diferenciais parciais. Rio de Janeiro: IMPA, 2003. ISBN 9788524401206}
      \Orgbibdef[compl]{Kreyszig, Erwin. Advanced engineering mathematics. Hoboken, NJ: John Wiley, c2006. ISBN 0471488852}
      \Orgbibdef[compl]{Simmons, George F.. Differential equations with applications and historical notes. New York: McGraw-Hill, c1972}
      \Orgbibdef[compl]{Solow, Daniel; Borrelli, Robert L.; Coleman, Courtney S.. Differential equationsa modeling perspective and how to read and do proof. New York: Wiley, 1998. ISBN 0471314129}
      \Orgbibdef[compl]{Spiegel, Murray Ralph. Analise de fourier. Sao Paulo: Mcgraw-Hill do Brasil, 1976}
      \Orgbibdef[compl]{Tenenbaum, Morris; Pollard, Harry. Ordinary differential equations:an elementary texbook for students of mathematics, engineering, and the sciences.. New York: Harper e Row, 1963}
      \Orgbibdef[compl]{Zill, Dennis G.; Cullen, Michael R.. Equações diferenciais. Makron Books: São Paulo, c2001}
      \Orgbibdef[compl]{Zill, Dennis G.; Cullen, Michael R.. Equações diferenciais. Makron Books: São Paulo, c2001}


%%%%%%
%%%%%%
%
\classdef[BaseMatematica]{FIS01183}{6}{FÍSICA III-C}

     \csummary{Temperatura. Calor. Teoria cinética dos gases. Termodinâmica. Física ondulatória: ondas mecânicas e eletro-magnéticas. Reflexão e refração.}

      \Orgbibdef{ C. H. Edwards Jr., D.E. Penney.. Equações Elementares com Problemas de Contorno. Rio de Janeiro: LTC, ISBN 9788570540577}
      \Orgbibdef{William E. Boyce, Richard C. DiPrima. Equações Diferenciais Elementares e Problemas de Valores de Contorno. LTC, 2015. ISBN 9788521627357}
      \Orgbibdef{Zill, Dennis G.. Equações diferenciais com aplicações em modelagem. São Paulo: Thomson, 2003. ISBN 8522103143; 9788522103140}

      \Orgbibdef[basic]{Alonso, Marcelo; Finn, Edward J.. Fisica. Harlow: Addison-Wesley, c1999. ISBN 8478290273}
      \Orgbibdef[basic]{Alonso, Marcelo; Finn, Edward J.; Moscati, Giorgio. Fisica um Curso Universitario :campos e Ondas. Sao Paulo: Edgard Blucher, c1967}
      \Orgbibdef[basic]{Halliday, David; Resnick, Robert; Walker, Jearl. Fundamentos de física. Rio de Janeiro: Livros Técnicos e Científicos, 2006-2007. ISBN 8521614845 (V.1); 9788521614845 (v.1); 8521614853 (V.2); 9788521614869 (V.3); 9788521614876 (V.4)}
      \Orgbibdef[basic]{Mckelvey, John P.; Grotch, Howard; Nunes, Frederico Dias. Fisica. Sao Paulo: Ed. Harper, c1979}
      \Orgbibdef[basic]{Nussenzveig, Hersh Moyses. Curso de física básica. Sao Paulo: Ed. Edgar Blucher, c2002. ISBN 8521202989 (v.1); 8521202997 (v.2); 8521201346 (v.3); 852120163X (v.4)}
      \Orgbibdef[basic]{Sears, Francis Weston. Fisica. Rio de Janeiro: Livros Tecnicos e Cientificos, 1983-1985}
      \Orgbibdef[basic]{Tipler, Paul A.; Mosca, Gene. Física :para cientistas e engenheiros. Rio de Janeiro: LTC, 2009. ISBN 9788521617105 (v.1); 9788521617112 (v.2); 9788521617129 (v.3)}

%%%%%%
%%%%%%
%
\classdef[Base.Mat]{ENG03043}{4}{MATERIAIS PARA ENGENHARIA A}

     \csummary{Materiais e aplicações principais em engenharia. Correlação entre estrutura e propriedades dos materiais. Microestrutura e suas relações com o comportamento mecânico. Materiais metálicos: classificação e aplicações específicas, metalografia, tratamentos térmicos e termoquímicos. Influência da microestrutura no comportamento mecânico. Processamento, microestrutura e comportamento mecânico dos materiais cerâmicos, poliméricos e conjugados.}

      \Orgbibdef{  William D. Callister Jr., David G. Rethwisch. Ciência e engenharia de materiais : uma introdução. Rio de Janeiro: LTC, 2020. ISBN 9788521637288}
      \Orgbibdef[basic]{Donald R. Askeland, Wendelin J. Wright. Ciência e Engenharia dos Materiais. São Paulo: Editora Cengage Learning, 2019. ISBN 9788522128112}
      \Orgbibdef[compl]{James F. Shackelford. Ciência dos materiais. São Paulo: Pearson Prentice Hall, 2008. ISBN 9788576051602}

%%%%%%
%%%%%%
%
\classdef[BaseMec]{ENG03042}{4}{MECÂNICA APLICADA II}

     \csummary{Cinemática do ponto material. 2ª. Lei de Newton. Energia e quantidade de movimento. Sistemas de pontos materiais. Cinemática de corpos rígidos. Princípios de conservação de energia e quantidade de movimento. Movimento de corpos rígidos em duas e três dimensões.}

      \Orgbibdef{Beer, Ferdinand Pierre; Johnston Jr., E. Russell; Cornwell, Phillip J.; Self, Brian P.; Sanghi, Sanjeev. Mecanica Vetorial para Engenheiros. Dinâmica. Rio de Janeiro: AMGH, 2019. ISBN 9788580556179 (Impresso), 9788580556186 (E-book)}
      \Orgbibdef{Hibbeler, Russell Charles. Dinâmica. Mecânica para Engenharia. São Paulo: Pearson, 2017. ISBN 9788543016252}
      \Orgbibdef{Meriam, James L.; Kraige, L. Glenn. Mecânica para Engenharia. Dinâmica. Rio de Janeiro: LTC, 2016. ISBN 9788521630142}

      \Orgbibdef[basic]{Gray, Gary L.; Constanzo, Francesco; Plesha, Michael E.. Mecânica para Engenharia. Dinâmica. Porto Alegre: Grupo A, 2014. ISBN 9788565837002 (Impresso), 9788565837293 (E-book)}
      \Orgbibdef[basic]{Nelson, E.W. Nelson; Best, Charles L.; McLean, W.G.; Potter, Merle C.. Engenharia Mecânica: Dinâmica Coleção Schaum. São Paulo: Grupo A, 2013. ISBN 9788582600412 (E-book)}
      \Orgbibdef[basic]{Rade, Domingos. Cinemática e Dinâmica para Engenharia. Rio de Janeiro: GEN LTC, 2017. ISBN 9788535281866}
      \Orgbibdef[basic]{Tenenbaum, Roberto A.. Dinâmica Aplicada. São Paulo: Manole, 2016. ISBN 9788520446775}
      \Orgbibdef[basic]{Tongue, Benson H. ; Sheppard, Sheri D.. Dinâmica: Análise e Projeto de Sistemas em Movimento. Rio de Janeiro: LTC, 2007. ISBN 9788521615422}


%%%%%%
%%%%%%
%
\classdef[BaseMatematica]{MAT02219}{4}{PROBABILIDADE E ESTATÍSTICA}

     \csummary{Probabilidade: Conceito e teoremas fundamentais. Variáveis aleatórias. Distribuições de probabilidade. Estatística descritiva. Noções de amostragem. Inferência estatística: Teoria da estimação e Testes de hipóteses. Regressão linear simples. Correlação.}

      \Orgbibdef{Barbetta, Pedro Alberto; Reis, Marcelo Menezes; Bornia, Antonio Cezar. Estatística :para cursos de engenharia e informática. São Paulo, SP: Atlas, 2008. ISBN 9788522449897}
      \Orgbibdef{Jay L. Devore. Probabilidade e estatística para engenharia e ciências. Cengage Learning, ISBN ISBN-10: 8522111839 ISBN-13: 9788522111831}
      \Orgbibdef{Montgomery, Douglas C.; Runger, George C.. Estatística aplicada e probabilidade para engenheiros. Rio de Janeiro: LTC, 2009. ISBN 8521616643; 9788521616641}

      \Orgbibdef[basic]{Costa Neto, Pedro Luiz de Oliveira. Estatística. São Paulo: Edgard Blücher, 2002. ISBN 8521203004}
      \Orgbibdef[basic]{Fonseca, Jairo Simon da; Martins, Gilberto de Andrade. Curso de estatística. São Paulo: Atlas, 1996. ISBN 8522414718}
      \Orgbibdef[basic]{Magalhães, Marcos Nascimento. Noções de probabilidade e estatística. São Paulo: Edusp, 2005. ISBN 8531406773}
      \Orgbibdef[basic]{Meyer, Paul L.. Probabilidade: aplicações à estatística. Rio de Janeiro: LTC, 2000. ISBN 8521602944}
      \Orgbibdef[basic]{Morettin, Pedro Alberto; Bussab, Wilton de Oliveira. Estatística básica. São Paulo: Saraiva, 2009. ISBN 8502034979}
      \Orgbibdef[basic]{Spiegel, Murray Ralph. Probabilidade e estatística. São Paulo: Pearson, 2004. ISBN 8534613001}

%%%%%%
%%%%%%
%
\classdef[Base.Digit]{ENG10042}{4}{SISTEMAS DIGITAIS}

     \csummary{Fundamentos. Sistemas de numeração e códigos binários. Aritmética binária. Álgebra booleana. Circuitos combinacionais. Flip-Flops, registradores, memórias e contadores. Relógios. Circuitos sequenciais. Arranjos lógicos programáveis.}

      \Orgbibdef{Brown, Stephen D.. Fundamentals of Digital Logic with VHDL Design. McGraw-Hill, 2008. ISBN 978-0077221430}
      \Orgbibdef{Floyd, Thomas L.. Sistemas digitais : fundamentos e aplicações. Bookman, ISBN 9788560031931}
      \Orgbibdef{Wakerly, John F.. Digital Design: principles and practices. --: Pearson, 2005. ISBN 978-0131863897}

      \Orgbibdef[basic]{Mano, M. Morris. Logic and computer design fundamentals. Pearson Prentice Hall, ISBN 9780131989269}
      \Orgbibdef[basic]{Ronald J. Tocci, Neal S. Widmer e Gregory L. Moss. Sistemas Digitais: Princípios e Aplicações. São Paulo: Prentice-Hall, 2011. ISBN 9788576059226}
      \Orgbibdef[basic]{Tokheim, Roger L. Fundamentos de eletrônica digital. HMGH, ISBN 9788580551921 (v. 1) e 9788580551945 (v.2)}
      \Orgbibdef[basic]{Vahid, Frank. Digital Design. Hoboken: John Wiley, ISBN 9780470044377}

      \Orgbibdef[compl]{FLETCHER. An Engineering Approach to Digital Design. Prentice Hall}
      \Orgbibdef[compl]{Karris, Steven T. Digital Circuit Analysis and Design with Simulink Modeling and Introduction to CPLDs and FPGAs. Orchard Publications, ISBN 13: 978-1-934404-05-8; 10: 1-934404-05-5}
      \Orgbibdef[compl]{Sandige, Richard S. Fundamentals of digital and computer design with VHDL. McGraw-Hill, ISBN 9780073380698}
      \Orgbibdef[compl]{Wagner, Flavio Rech. Fundamentos de circuitos digitais. Bookman, ISBN 9788577803453}


%%%%%%%%%%%%%%%%%
%%%%%%%%%%%%%%%%%
%%%
%%%  Etapa 04
%%%
%%%%%%%%%%%%%%%%%
%%%%%%%%%%%%%%%%%



%%%%%%
%%%%%%
%
\classdef[BaseMatematica]{MAT01169}{6}{CÁLCULO NUMÉRICO}

     \csummary{Sistemas de numeração. Zeros de funções. Métodos numéricos de Álgebra Linear. Interpolação. Derivação e integração numérica. Aproximação de funções, ajustamento de dados. Solução numérica de equações diferenciais ordinárias.}

      \Orgbibdef{Borche, Alejandro. Métodos Numéricos. Porto Alegre: Ed. da UFRGS, 2008. ISBN 9788570259783}
      \Orgbibdef{Burden, Richard L.; Faires, J. Douglas. Análise numérica. São Paulo: Pioneira Thomson Learning, 2003. ISBN 852210297X}

      \Orgbibdef[basic]{Bortoli, Álvaro e outros.. Introdução ao Cálculo Numérico - caderno de apoio didático B59. Instituto de Matemática: IM - UFRGS, 2001}
      \Orgbibdef[basic]{Ruggiero, M; Lopes, V.. Cálculo Numérico - Aspectos Teóricos e Computacionais. Pearson, 1996. ISBN 9788534602044}

      \Orgbibdef[compl]{Barroso, Leônidas. Cálculo numérico :com aplicações. São Paulo: Harbra, c1987. ISBN 8529400895; 9788529400891}
      \Orgbibdef[compl]{Conte, Samuel. Elementos de Análise Numérica. Porto Alegre: Globo, 1965-1971}
      \Orgbibdef[compl]{R.Burden, J. Faires. Numerical Analysis. London: Thompson Learning, 2005. ISBN 0534392008;0534404995}
      \Orgbibdef[compl]{Roque, Waldir Leite. Introdução ao cálculo numérico :um texto integrado ao cálculo numérico DERIVE. São Paulo: Atlas, 2000. ISBN 8522427224}
      \Orgbibdef[compl]{Sperandio, Décio; Mendes, João Teixeira; Silva, Luiz Henry Monken e. Cálculo numérico :características matemáticas e computacionais dos métodos numéricos. São Paulo: Pearson, c2003. ISBN 8587918745}

%%%%%%
%%%%%%
%
\classdef[BaseEletro]{ENG10002}{4}{CIRCUITOS ELÉTRICOS II - C}

     \csummary{Análise fasorial. Potência. Acoplamento magnético e transformadores. Circuitos polifásicos. introdução básica de Fourier e Laplace. Análise de circuitos no domínio da frequência. Aplicação de Transformadas de Fourier e de Laplace em circuitos.}

      \Orgbibdef{Charles K. Alexander e Matthew N. O. Sadiku. Fundamentos de circuitos elétricos.. McGrawHill, 2003. ISBN 978-85-86804-97-7}
      \Orgbibdef{J. David Irwin. Análise básica de circuitos para engenharia. LTC Editora, 2003. ISBN 85-216-1374-1}
      \Orgbibdef{James W. Nilsson e Susan A. Riedel. Circuitos elétricos. LTC Editora, 2003. ISBN 85-216-1374-1   }

      \Orgbibdef[basic]{J. Bird. Circuitos Elétricos - Teoria e Tecnologia. Editora Campus, 2009. ISBN 978-85-352-2771-0}

      \Orgbibdef[compl]{Charles A. Desoer e Ernest S. Kuh. Teoria básica de circuitos lineares. Guanabara Dois, 1979}
      \Orgbibdef[compl]{R. C. Dorf, J. A. Svoboda. Introdução aos Circuitos Elétricos. LTC, 2003. ISBN 85-216-1367-9}
      \Orgbibdef[compl]{W. H. Hayt Jr., J. E. Kemmerly, S. M. Durbin. Análise de Circuitos em Engenharia. McGraw-Hill, 2008. ISBN 978-85-7726-021-8}

%%%%%%
%%%%%%
%
\classdef[Base.Digit]{ENG10043}{2}{LABORATÓRIO DE SISTEMAS DIGITIAIS}

     \csummary{Síntese de circuitos digitais: circuitos aritméticos, contadores, registradores e máquinas de estados. Ferramentas computacionais de projeto e simulação. Circuitos integrados Digitais. Arranjos lógicos programáveis.}

      \Orgbibdef{Brown, Stephen D.. Fundamentals of Digital Logic with VHDL Design. McGraw-Hill, ISBN 9780073529530}
      \Orgbibdef{Floyd, Thomas L.. Sistemas Digitais: Fundamentos e Aplicações. Bookman, ISBN 9788560031931}
      \Orgbibdef{Wakerly, John F.. Digital Design: Principles and Practices. Prentice Hall, ISBN 978-0131863897}

      \Orgbibdef[basic]{Mano, M. Morris. Logic and Computer Design Fundamentals. Pearson - Prentice Hall, ISBN 9780131989269}
      \Orgbibdef[basic]{Ronald J. Tocci, Neal S. Wildmer e Gregory L. Moss. Sistemas Digitais: Princípios e Aplicações. Prentice-Hall, ISBN 9788576050957 (10ª Edição); 9788576059226 (11ª Edição)}
      \Orgbibdef[basic]{Vahid, Frank. Digital Design. Hoboken, ISBN 9780470044377}

      \Orgbibdef[compl]{Karris, Steven T. Digital Circuit Analysis and Design with Simulink Modeling and Introduction to CPLDs and FPGAs. Orchard Publications, ISBN 978-1-934404-05-8}

%%%%%%
%%%%%%
%
\classdef[BaseMatematica]{MAT01168}{6}{MATEMÁTICA APLICADA II}

     \csummary{Séries de Fourier. Integral de Fourier. Transformadas de Fourier e de Laplace. Análise vetorial.}

      \Orgbibdef{Anton, Howard; Bivens, Irl; Davis, Stephen; Doering, Claus Ivo. Cálculo. Porto Alegre: Bookman, 2007. ISBN 9788560031634 (V.1); 9788560031801 (V.2)}
      \Orgbibdef{Hwei P. Hsu. Sinais e Sistemas. Porto Alegre: Bookman Cia. Editora, 2011. ISBN 978-85-7780-938-7}

      \Orgbibdef[basic]{Hsu, Hwei P.. Análise de Fourier. Rio de Janeiro: Livros Técnicos e Científicos, 1973}
      \Orgbibdef[basic]{Irene Strauch. Notas de aula: Análise Vetorial, Transformada de Laplace, Análise de Fourier}
      \Orgbibdef[basic]{Kreyszig, Erwin. Matemática superior. Rio de Janeiro: Livros Técnicos e Científicos, 1983-1986. ISBN 8521601816(v.1); 852160355X(v.3); 8521603738(v.4); 8521601808(obra completa)}
      \Orgbibdef[basic]{Spiegel, Murray Ralph. Análise vetorial :com introdução à análise tensorial. São Paulo: McGraw-Hill, c1972}
      \Orgbibdef[basic]{Spiegel, Murray Ralph. Schaum?s outline of theory and problems of complex variables : with an introduction to conformal mapping and its applications. Nova Iorque: McGraw-Hill, ISBN 978-0071615693}
      \Orgbibdef[basic]{Zill, Dennis G.. Equações diferenciais com aplicações em modelagem. São Paulo: Thomson, 2003. ISBN 8522103143; 9788522103140}

      \Orgbibdef[compl]{Asmar, Nakhle. Partial differential equations and boundary value problems. New Jersey: Prentice-Hall, c2005. ISBN 0131480960}
      \Orgbibdef[compl]{O'Neil, Peter V.. Advanced engineering mathematics. New York: Brooks/Cole Pub. Co., 2003. ISBN 9780534401306}
      \Orgbibdef[compl]{Spiegel, Murray Ralph. Transformadas de Laplace :resumo da teoria, 263 problemas resolvidos, 614 problemas propostos. São Paulo: McGraw-Hill do Brasil, c1978}
      \Orgbibdef[compl]{Strang, Gilbert. Calculus. Cambridge: Wellesley-Cambridge Press, 1991. ISBN 0961408820}
      \Orgbibdef[compl]{Stroud, K.A.; Booth, Dexter J.. Advanced engineering mathematics :a new edition of further engineering mathematics. New York: Palgrave Macmillan, c2003. ISBN 1403903123}
      \Orgbibdef[compl]{Zill, Dennis G.; Cullen, Michael R.. Equações diferenciais. Makron Books: São Paulo, c2001}


%%%%%%
%%%%%%
%
\classdef[Base.Mat]{ENG03092}{4}{MECÂNICA DOS SÓLIDOS I-A}

     \csummary{Introdução à Mecânica dos Sólidos. Solicitações internas. Tensões e deformações. Esforço axial. Torção. Flexão simples. Cisalhamento em vigas. Solicitações compostas. Análise e transformação de tensões. Análise e transformação de deformações. Critérios de falha. Noções de coeficiente de segurança.}

      \Orgbibdef{Beer, F. P.; Johnston, E. R., Jr., DeWolf, J. T. e Mazurek, D. F.. Estática e Mecânica dos Materiais. McGraw-Hill, 2013. ISBN 978-85-8055-164-8}
      \Orgbibdef{E. P. Popov. Introdução à Mecânica dos Sólidos. Blücher, 1978. ISBN 978-85-212-0094-9}
      \Orgbibdef{Hibbeler, Russel C.. Resistência dos Materiais. PEARSON, 2010. ISBN 978-85-7605-373-6}

      \Orgbibdef[basic]{Gere, James M.. Mecânica dos Materiais. Thomson Pioeira, ISBN 8522103135; 978-85-2210-313-3}
      \Orgbibdef[basic]{J.N. Reddy. Principles of Continuum Mechanics: A study of conservation principles with applications. Cambridge, ISBN 978-0-521-51369-2}
      \Orgbibdef[basic]{Paulo de Tarso R. Mendonça e Eduardo A. Fancello. O MÉTODO DE ELEMENTOS FINITOS APLICADO À MECÂNICA DOS SÓLIDOS. Florianópolis: Orsa Maggiore, 2019. ISBN 9788590715313}

      \Orgbibdef[compl]{Gordon, J.E.. Structures: Or Why Things Don't Fall Down. Editora Penguin Books, ISBN 0306812835}
      \Orgbibdef[compl]{Pilkey, Walter D.; Chang, Pin Yu.. Modern formulas for statics and dynamics :a stress-and-strain approach. Editora McGraw-Hill, ISBN 0070499985}
      \Orgbibdef[compl]{Shames, Irving H.. Introdução à mecânica dos sólidos. Editora Prentice Hall do Brasil, ISBN 8570540019}


%%%%%%
%%%%%%
%
\classdef[BaseMec]{ENG03316}{4}{MECANISMOS I}

     \csummary{Introdução a análise de mecanismos: Conceito e classificação de mecanismos. Cadeias Cinemáticas. Análise cinemática dos mecanismos. Cames. Teoria das engrenagens. Forças de inércia em máquinas. Balanceamento estático e dinâmico. Aplicações Industriais ou em Equipamentos.}


      \Orgbibdef[basic]{NORTON, Robert L.. Cinemática e Dinâmica dos Mecanismos. McGraw-Hill ? AMGH Editora Ltda, ISBN 9788563308191}
      \Orgbibdef[basic]{UICKER, John J.; PENNOCK, Gordon R.; SHIGLEY, Joseph E.. Theory of Machines and Mechanisms. Oxford University Press, 2011. ISBN 9780195371239}

      \Orgbibdef[compl]{MABIE, Hamilton H.; REINHOLTZ, Charles F.. Mechanisms and Dynamics of Machinery. John Wiley, 1987. ISBN 0471802379}
      \Orgbibdef[compl]{NORTON, Robert L.. DESIGN OF MACHINERY. Porto Alegre: McGraw Hill, 2007. ISBN 007329098x}


%%%%%%
%%%%%%
%
\classdef[BaseMec]{ENG03044}{4}{MODELAGEM DE SISTEMAS MECÂNICOS}

     \csummary{Modelagem e modelos. Tipos de modelos. Modelagem em computador. Estimativas e aproximações. Modelagem sistemática de sistemas mecânicos, elétricos, fluídicos e térmicos. Analogias elétricas. Sistemas híbridos. Técnicas de representação de modelos matemáticos. Respostas transitória e permanente de sistemas dinâmicos. Análise no domínio frequência. Simulação de resposta de sistemas dinâmicos a excitações típicas.}


      \Orgbibdef{Close C. M., Frederick D. K., Newell J. C.. Modeling and Analysis of Dynamic Systems. USA: Wiley, 2001. ISBN 9780471394426}
      \Orgbibdef{Kluever, C. A.. Dynamic Systems: Modeling, Simulation, and Control. New York, USA: Wiley, 2015. ISBN 978-1118289457}
      \Orgbibdef{Palm III, W. J.. System Dynamics. McGraw-Hill, 2013. ISBN 9780073398068}


      \Orgbibdef[basic]{Kulakowski, B. T., Gardner, J. F., Shearer, J. L.. Dynamic Modeling and Control of Engineering Systems. Cambridge, UK: Cambridge University Press, 2012. ISBN 978-1107650442}
      \Orgbibdef[basic]{Lu, B., Esfandiari, R. S.. Modeling and Analysis of Dynamic Systems. Boca Raton, FL, USA: CRC Press, 2014. ISBN 9781466574939}
      \Orgbibdef[basic]{Palm III, W.J.. Modeling, Analysis and Control of Dynamic Systems. USA: John Wiley, 1999. ISBN 9780471073703}
      \Orgbibdef[basic]{Rao, S. S.. Vibrações Mecânicas. Pearson, 2008. ISBN 9788576052005}

      \Orgbibdef[compl]{Das, S.. Mechatronic Modeling and Simulation Using Bond Graphs. CRC Press, 2009. ISBN 9781420073140}
      \Orgbibdef[compl]{Franklin, G., Powell, J. D., Emami-Naeini, A.. Sistemas de Controle para Engenharia. Bookman, 2013. ISBN 9788582600672}
      \Orgbibdef[compl]{Golnaraghi, M. F., Kuo, B. C.. Sistemas de controle automático. LTC-GEN, ISBN 9788521606727}
      \Orgbibdef[compl]{Ogata, K.. Engenharia de Controle Moderno. Pearson Prentice Hall, ISBN 9788576058106}


