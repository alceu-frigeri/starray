
%%%%%%%%%%%%%%%%%
%%%%%%%%%%%%%%%%%
%%%
%%%  Etapa 05
%%%
%%%%%%%%%%%%%%%%%
%%%%%%%%%%%%%%%%%



%%%%%%
%%%%%%
%
\classdef[BaseEletro]{ENG10044}{4}{ELETRÔNICA FUNDAMENTAL I-B}

     \csummary{Amplificadores operacionais, diodos, circuitos conformadores, transistores de junção e efeito de campo: características, polarização, estabilidade térmica e resposta em frequência. Amplificadores de um ou mais estágios, realimentação e o teorema do elemento extra: princípios de análise de estabilidade e resposta em frequência.}


      \Orgbibdef{Sedra, Adel S.; Smith, Kenneth C.. Microeletrônica. São Paulo, SP: Pearson Universidades, 2007. ISBN 9788576050223}

      \Orgbibdef[basic]{Millman, J.; Halkias C.; Parikh, C.D.. MILLMAN'S INTEGRATED ELECTRONICS. McGraw-Hill, 2010. ISBN 0070151423}
      \Orgbibdef[basic]{Schilling, Donald L.; Belove, Charles. Circuitos eletrônicos :discretos e integrados. Guanabara Dois}

      \Orgbibdef[compl]{Behzad Razavi. Fundamentos de Microeletrônica. LTC - Grupo GEN, ISBN 9788521617327}
      \Orgbibdef[compl]{Boylestad, Robert L.; Nashelsky, Louis. Dispositivos eletrônicos e teoria de circuitos. Prentice-Hall do Brasil, ISBN 8587918222}

%%%%%%
%%%%%%
%
\classdef[BaseEletro]{ENG10003}{2}{LABORATÓRIO DE CIRCUITOS ELÉTRICOS}

     \csummary{Instrumentos de medida e conceitos fundamentais de medição. Ferramentas Computacionais de análise, e simulação. Aplicação de análise de circuitos resistivos. Aplicação de análise de circuitos de 1ª e 2ª ordem. Resposta em frequência de circuitos. Análise fasorial. Medição de potência em Circuitos Trifásicos. Transformadores.}

      \Orgbibdef{C. K. Alexander, M. N. O. Sadiku. Fundamentos de Circuitos Elétricos. McGraw-Hill, 2008. ISBN 978-85-86804-97-7}
      \Orgbibdef{J. W. Nilsson. Circuitos Elétricos. Rio de Janeiro: Pearson Prentice-Hall, 1999. ISBN 978-85-7605-159-6}

      \Orgbibdef{J David Irwin. Análise Básica de Circuitos para Engenharia. LTC, 2003. ISBN 85-216-1374-1}

      \Orgbibdef[compl]{C. A. Desoer, E. S. Kuh. Teoria Básica de Circuitos. Guanabara Dois, 1979}
      \Orgbibdef[compl]{J. Bird. Circuitos Elétricos - Teoria e Tencnologia. Campus, 2009. ISBN 978-85-352-2771-0}
      \Orgbibdef[compl]{W. H. Hayt Jr., J. E. Kemmerly, S. M. Durbin. Análise de Circuitos em Engenharia. McGraw-Hill, 2008. ISBN 978-85-7726-021-8}
      \Orgbibdef[compl]{Yannis Tsividis. A First Lab in Circuits and Electronics. Wiley, 2001. ISBN 978-0-471-38695-7}

%%%%%%
%%%%%%
%
\classdef[Base.Mat]{ENG03004}{4}{MECÂNICA DOS SÓLIDOS II}

     \csummary{Análise de tensões. Teorias estruturais. Análise de flexão de vigas. Métodos clássicos de análise de vigas. Métodos de solução de problemas estaticamente indeterminados. Introdução à análise limite em vigas. Princípios energéticos. Flambagem de colunas. Introdução à elasticidade.}

      \Orgbibdef{Hibeller. Análise de Estruturas. Brasil: Pearson, 2017. Disponível em: ps://www.amazon.com.br/Análise-das-Estruturas-R-C-Hibbeler/dp/8581431275?tag=goog0ef-20}

      \Orgbibdef[basic]{Beer, Ferdinand Pierre; Johnston, E. Russell, Jr.; DeWolf, John T.. Resistência dos materiais :mecânica dos materiais. São Paulo: McGraw-Hill, c2006. ISBN 8586804835; 9788586804830}
      \Orgbibdef[basic]{Salvadori, Mario G.; Heller, Robert. Estructuras para arquitectos. Buenos Aires: Cp67, 1987. ISBN 9509575143}
      \Orgbibdef[basic]{Steffen, Julio Cezar; Tamagna, Alberto. Prática de sistemas estruturais. São Leopoldo: UNISINOS, 1982}
      \Orgbibdef[basic]{Sussekind, Jose Carlos. Curso de analise estrutural. Sao Paulo: Globo, 1994. ISBN 8525002267}


%%%%%%
%%%%%%
%
\classdef[Pro.Control]{ENG10017}{6}{SISTEMAS E SINAIS}

     \csummary{Técnicas de modelagem e análise de sistemas lineares e sistemas amostrados. Introdução a sistemas não lineares.}

      \Orgbibdef{Haykin, Simon; Van Veen, Barry; Laschuk, Anatolio. Sinais e sistemas. Porto Alegre: Bookman, 2001. ISBN 8573077417; 9788573077414}

      \Orgbibdef[basic]{Franklin, Gene F.; Powell, J. David; Emami-Naeini, Abbas. Feedback control of dynamic systems. Upper Saddle River, N.J.: Pearson Prentice Hall, c2006. ISBN 0131499300}
      \Orgbibdef[basic]{Lathi, B. P.. Sinais e sistemas lineares. Porto Alegre: Bookman, 2007. ISBN 0195158334 (obra original); 9788560031139}
      \Orgbibdef[basic]{Oppenheim, Alan V.; Willsky, Alan S.; Nawab, Syed Hamid. Signals. Upper Saddle River, N.J.: Prentice Hall, c1997. ISBN 0138147574}

      \Orgbibdef[compl]{Geromel, José Claudio; Palhares, Alvaro Geraldo Badan. Análise linear de sistemas dinâmicos :teoria, ensaios práticos e exercícios. São Paulo: Edgard Blücher, 2004. ISBN 8521203357; 9788521203353}
      \Orgbibdef[compl]{Hsu, Hwei P.. Sinais e Sistemas. Porto Alegre: Bookman, 2004. ISBN 8536303603}
      \Orgbibdef[compl]{Olivier, J. C.. Linear Systems and Signals: A Primer. Norwood, MA: Artech House, 2019}

%%%%%%
%%%%%%
%
\classdef[Base.FenTrans]{ENG07086}{5}{TERMODINÂMICA E TRANSFERÊNCIA DE CALOR}

     \csummary{Propriedades termodinâmicas de substâncias puras e misturas. Energia, trabalho e calor e as Leis da Termodinâmica. Termodinâmica dos sistemas abertos. Eficiência de processos térmicos. Mecanismos de transferência de calor. Condução de calor em regime estacionário e transiente.}


      \Orgbibdef{C. Borgnakke, R. E. Sonntag, G. J. Van Wylen. Fundamentos da Termodinâmica. São Paulo: Edgard Blucher, 2009. ISBN 9788521204909}
      \Orgbibdef{J. M. Smith, H. C. Van Ness, M. M. Abbott. Introdução à Termodinâmica da Engenharia Química. Rio de Janeiro: LTC - Livros Técnicos e Científicos, 205. ISBN 9788521615538}

      \Orgbibdef[basic]{Milo D. Koretsky. Termodinâmica para a Engenharia Química. Rio de Janeiro: LTC, 2017. ISBN 9788521615309}

%%%%%%
%%%%%%
%
\classdef[Pro.Robotica]{ENG10026}{4}{ROBÓTICA-A}

     \csummary{Estrutura de robô: características, acionamento, controle, manipuladores e sensores. Capacidade do robô. Aplicações do robô. Noções de cinemática e dinâmica. Programação do robô. Sistemas de programação. Sistema controlador - periféricos-robô.}


      \Orgbibdef[basic]{Craig, John J.. Introduction to robotics :mechanics and control. Upper Saddle River, N.J.: Pearson Prentice Hall, c2005. ISBN 0201543613; 9780201543612}
      \Orgbibdef[basic]{Fu, K. S., Gonzales, R. C., Lee, C. S. G.. Robotics Control, Sensing, Vision and Intelligence. New York: McGraw-Hill, 1987}

      \Orgbibdef[compl]{Asada, Haruhiko; Slotine, Jean-Jacques E.. Robot analysis and control. New York: John Wiley, c1986. ISBN 471830291; 9780471830290}
      \Orgbibdef[compl]{Goebel, P.. ROS by Example - INDIGO. Raleigh, NC: Lulu, 2015. ISBN 9781312392663}
      \Orgbibdef[compl]{Martinez, A., Fernández, E.. Learning ROS for Robotics Programming. Birmingham, UK: Packt Publishing, 2013. ISBN 978-1-78216-144-8}
      \Orgbibdef[compl]{O'Kane, J. M.. A Gentle Introduction to ROS. CreateSpace, 2013. ISBN 978-1492143239}
      \Orgbibdef[compl]{Romano, Vitor Ferreira. Robótica industrial:Vaplicação na indústria de manufatura e de processos. São Paulo: Edgard Blücher, c2002. ISBN 8521203152}


%%%%%%
%%%%%% FOI 100% omitida pelo grok.
%
\classdef[Pro.Robotica]{ENG03380}{4}{ROBÓTICA}

     \csummary{Configurações físicas de robôs, movimentos básicos, características técnicas, programação elementar, tipos de linguagens, efetuadores finais, controle da célula de trabalho. Aplicação, dados de projeto.}


      \Orgbibdef[basic]{John J. Craig. Robótica. Pearson, 2013. ISBN 9788581431284}
      \Orgbibdef[basic]{Saeed Benjamin Niku. Introdução á Robótica - Análise, Controle, Aplicações. Rio de Janeiro: LTC, 2013. ISBN 9788521622376}

      \Orgbibdef[compl]{Bruno Siciliano, Lorenzo Sciavicco, Luigi Villani, Giuseppe Oriolo. Robotics Modelling, Planning and Control. Springer, ISBN 9781846286414}
      \Orgbibdef[compl]{Fernando Pazos. Automação de Sistemas. Rio de Janeiro: Axcel Books, 2002. ISBN 8573231718}
      \Orgbibdef[compl]{João Maurício Rosário. Princípios de Mecatrônica. São Paulo: Pearson Prentice Hall, 2005. ISBN 8576050102; 9788576050100}
      \Orgbibdef[compl]{Mark W. Spong, Seth Hutchinson e M. Vidyasagar. Robot Modeling and Control. John Wiley, 2005. ISBN 9780471649908}
      \Orgbibdef[compl]{Mikell P. Groover,. Industrial robotics: technology, programming and applications. Editora McGraw-Hill, ISBN 007024989x}


%%%%%%%%%%%%%%%%%
%%%%%%%%%%%%%%%%%
%%%
%%%  Etapa 06
%%%
%%%%%%%%%%%%%%%%%
%%%%%%%%%%%%%%%%%




%%%%%%
%%%%%%
%
\classdef[Pro.Maquinas]{ENG10047}{4}{FUNDAMENTOS DE MÁQUINAS ELÉTRICAS}

     \csummary{Princípios de conversão eletromecânica de energia. Dispositivos eletromagnéticos. Máquinas de corrente contínua. Máquinas de corrente alternada. Modelos de dispositivos em regime permanente. Características operacionais em regime permanente.}

      \Orgbibdef{A. E. Fitzgerald, C . Kingsley Jr, S. D. Umans. Máquinas Elétricas. Bookman, ISBN 978-85- 60031-04-7}
      \Orgbibdef{D. C . White, H. H. Woodson. Electromechanical Energy Conversion. John Wiley, ISBN 978- 0262230292}
      \Orgbibdef{Edson Bim. Máquinas Elétricas e Acionamento. Campus/Elsevier, ISBN 9788535230291}

      \Orgbibdef[basic]{A. Ivanon-Smolensky. Electrical Machines. MIR, 1980}
      \Orgbibdef[basic]{M. Kostenko, L. Piotrovski. Máquinas Eléctricas. Lopes da Silva, 1979}
      \Orgbibdef[basic]{P. C . Krause. Analysis of electric machinery and drive systems. Wiley, ISBN 9781118024294}
      \Orgbibdef[basic]{Syed A. Nasar. Electric Machines and Power Systems. McGraw-Hill, ISBN 978-0071135269}

      \Orgbibdef[compl]{C harles A. Gros. Electric Machines. CRC Press, ISBN 9780849385810}
      \Orgbibdef[compl]{Kay Hameier. Numerical Modelling and Design of Electrical Machines and Devices. WIT Press, ISBN 978-1853126260}
      \Orgbibdef[compl]{S. J. C hapman. Electric Machinery Fundamentsoals. McGraw-Hill, ISBN 9780072465235}


%%%%%%
%%%%%%
%
\classdef[Pro.Maquinas]{ENG10022}{4}{INSTRUMENTAÇÃO FUNDAMENTAL PARA CONTROLE E AUTOMAÇÃO}

     \csummary{Medidas em processos industriais. Precisão, erros e sua propagação. Transdutores para medição de grandezas físicas. Condicionamento de sinais e interfaceamento. Métodos indiretos de medida.}


      \Orgbibdef{Balbinot A., Brusamarello, V. J.. Instrumentação e Fundamentos de Medidas. GEN, 2011. ISBN 9788521615637}
      \Orgbibdef{Balbinot, A., Brusamarello V. J.. Instrumentação e Fundamentos de Medidas. GEN, ISBN 8521614969}

      \Orgbibdef[basic]{DOEBELIN, O.. Measurement Systems. McGraw-Hill, ISBN 9780071077606}
      \Orgbibdef[basic]{FRADEN, J.. Handbook of Modern Sensors. Springer AIP Press, 2010. ISBN 1441964657}
      \Orgbibdef[basic]{PALLÀS-ARENI R., WEBSTER J. G.. Sensors and Signal Conditioning. John Wiley, ISBN 0471332321}

      \Orgbibdef[compl]{CONSIDINE, D. A.. Process Instruments and Controls Handbook.. Mc Graw-Hill Book Company, ISBN 0070125821}
      \Orgbibdef[compl]{HOLMAN J. P.. Experimental Methods for Engineers. McGraw-Hill, Inc, ISBN 9780073660554}

%%%%%%
%%%%%%
%
\classdef[BaseEletro]{ENG10045}{2}{LABORATÓRIO DE ELETRÔNICA}

     \csummary{Instrumentos de medida e conceitos fundamentais de medição. Ferramentas computacionais de análise e simulação de circuitos não-lineares: diodos, transistores de junção e efeito de campo. Resposta em frequência de circuitos ativos. Circuitos conformadores, amplificadores de um e de diversos estágios realimentados. Amplificadores operacionais.}

      \Orgbibdef{Sedra, Adel S.; Smith, Kenneth C.. Microeletrônica. Editora Pearson Prentice Hall, ISBN (ISBN: 9788576050223)}
      \Orgbibdef{Silva, Manuel de Medeiros da. Circuitos com Transitores Bipolares e MOS. Lisboa: Fundação Calouste Gulbenkian, 2008. ISBN 978-972-31-0840-8}

      \Orgbibdef[basic]{Desoer, Charles A.; Kuh, Ernest S.. Basic Circuit Theory.. Érica, 2006. ISBN 0-07-085183-2}

      \Orgbibdef[compl]{Cordell, Bob. Designing Audio Power Amplifiers.. McGraw-Hill/TAB, ISBN 978-0071640244}

%%%%%%
%%%%%%
%
\classdef[Base.FenTrans]{ENG07069}{2}{PRINCÍPIOS DA MECÂNICA DE FLUIDOS}

     \csummary{Princípios de transferência de quantidade de movimento. Equações de conservação nas formas integral e diferencial. Estática dos fluidos. Camada limite. Equações de projeto para sistemas de transporte de fluidos.}


      \Orgbibdef{Merle C. Potter, David C. Wiggert. Mecânica dos fluidos. Rio de Janeiro: Cengage, 2013. ISBN 8522103097}
      \Orgbibdef{Robert W. Fox, Alan T. McDonald, John C. Leylegian. Introdução à mecânica dos Fluidos. Rio de Janeiro: LTC, 2017}
      \Orgbibdef{Welty, James R.. Fundamentos de transferência de momento, de calor e de massa. Rio de Janeiro: LTC, 2017}

      \Orgbibdef[basic]{Bird, R. Byron; Stewart, Warren E.; Lightfoot, Edwin N. Fenômenos de Transporte. Rio de Janeiro: LTC, 2004. ISBN 8521613938}

%%%%%%
%%%%%%
%
\classdef[Pro.Automacao]{ENG10023}{4}{SISTEMAS DE AUTOMAÇÃO}

     \csummary{Sistemas de automação industrial e de controle de processos. Técnicas de Modelagem e Metodologia de Desenvolvimento de Sistemas de Automação Industrial (Clássica e Orientada a objetos), Sistemas de Tempo Real (Linguagens de Programação, Sistemas Operacionais).}

      \Orgbibdef{Paul T. Ward. Structured Development for Real-Time Systems, Vol. III: Implementation Modeling Techniques. Prentice Hall, ISBN 9780138548032}
      \Orgbibdef{Selic, Bran; Gullekson, Garth; Ward, Paul T.. Real-time object-oriented modeling. John Wiley, ISBN 0471599174}

      \Orgbibdef[compl]{Awad, Maher; Kuusela, Juha; Ziegler, Jurgen. Object-oriented technology for real-time systems :a practical approach using omt and fusion. Prentice Hall, ISBN 0132279436}


%%%%%%
%%%%%%
%
\classdef[Pro.Control]{ENG10004}{4}{SISTEMAS DE CONTROLE I - B}

     \csummary{Modelagem e identificação de sistemas dinâmicos. Conceitos básicos e problemas fundamentais em sistemas de controle. Controladores PID: Teoria e ajuste. Projeto de controladores para sistemas monovariáveis via método do lugar das raízes. Aspectos não-lineares em sistemas de controle.}


      \Orgbibdef{Bazanella, Alexandre Sanfelice; Gomes da Silva Junior, Joao Manoel. Sistemas de controle: princípios e métodos de projeto. UFRGS, 2005. ISBN 8570258496}

      \Orgbibdef[basic]{Astrom, Karl Johan; Hagglund, Tore. Pid controllers: theory, design, and tuning. ISA, 1995. ISBN 978-1556175169}
      \Orgbibdef[basic]{Franklin, Gene F.; Powell, J. David; Emami-Naeini, Abbas. Feedback control of dynamic systems. Prentice Hall, ISBN 0131499300}
      \Orgbibdef[basic]{Ogata, Katsuhiko. Engenharia de controle moderno. Prentice Hall do Brasil, ISBN 8587918230}

      \Orgbibdef[compl]{Dorf, Richard. Modern control systems. Prentice Hall, ISBN 0131457330}
      \Orgbibdef[compl]{Kuo, Benjamin C.. Automatic control systems. Wiley, ISBN 0471134767}



