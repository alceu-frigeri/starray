%%%%I t%%%%%%%%%%%%%
%%%%%%%%%%%%%%%%%
%%%
%%%  Etapa Eletivas
%%%
%%%%%%%%%%%%%%%%%
%%%%%%%%%%%%%%%%%

%%%%%%
%%%%%%
%
\classdef[Transv.integ]{CCA99008}{9}{PROJETO INTEGRADO I}

     \csummary{Atuação em equipes para analisar, propor e desenvolver soluções para problemas de Engenharia de interesse da sociedade, contemplando seus aspectos técnicos, 
     econômicos, socioambientais, de acessibilidade e de prevenção de desastres, entre outros. Os problemas serão colhidos 
     junto à sociedade via ação de extensão vinculada, sendo de natureza inerentemente aberta, prática e integradora, oportunizando que os 
     estudantes trabalhem simultaneamente os conteúdos aprendidos múltiplas disciplinas diferentes e em contextos realistas. Desenvolvimento de 
     habilidades de trabalho autônomo, comunicação, convívio social e respeito à diversidade através da atuação em grupos e do contato com questões
      e/ou indivíduos externos à Universidade. Dada a natureza integradora desta atividade, espera-se que os alunos de Projeto Integrado I, 
      interajam com os alunos de Projeto Integrado II e III no processo, permitindo a troca de experiência de alunos em diversos momentos na parte 
      profissionalizante do curso.}

      \bibdef{Brockman2009}
      \bibdef{Silva2006}
      \bibdef{Junival2008}
      \bibdef[basic]{Shigley2005}
      \bibdef[basic]{Bishop2002}
      \bibdef[compl]{Silva2008}
      \bibdef[compl]{Onwubolu2005}
      \bibdef[compl]{Bishop2008}
      \bibdef[compl]{Norton2004}
%%%%%%
%%%%%%
%
\classdef[Transv.integ]{CCA99009}{9}{PROJETO INTEGRADO II}

     \csummary{Atuação em equipes para analisar, propor e desenvolver soluções para problemas de Engenharia de interesse da sociedade, contemplando seus aspectos técnicos, 
     econômicos, socioambientais, de acessibilidade e de prevenção de desastres, entre outros. Os problemas serão colhidos 
     junto à sociedade via ação de extensão vinculada, sendo de natureza inerentemente aberta, prática e integradora, oportunizando que os 
     estudantes trabalhem simultaneamente os conteúdos aprendidos múltiplas disciplinas diferentes e em contextos realistas. Desenvolvimento de 
     habilidades de trabalho autônomo, comunicação, convívio social e respeito à diversidade através da atuação em grupos e do contato com questões 
     e/ou indivíduos externos à Universidade.  Dada a natureza integradora desta atividade, espera-se que os alunos de Projeto Integrado II, 
     interajam com os alunos de Projeto Integrado I e III no processo, permitindo a troca de experiência de alunos em diversos momentos na parte profissionalizante do curso.}

      \bibdef{Juvinall2008}
      \bibdef{Silva2006b}
      \bibdef[basic]{Bishop2002b}
      \bibdef[basic]{Shigley2005b}
      \bibdef[compl]{Bishop2008b}
      \bibdef[compl]{Norton2004b}
      \bibdef[compl]{Onwubolu2005b}
      \bibdef[compl]{Silva2008b}
%%%%%%
%%%%%%
%
\classdef[Transv.integ]{CCA99010}{9}{PROJETO INTEGRADO III}

     \csummary{Atuação em equipes para analisar, propor e desenvolver soluções para problemas de Engenharia de interesse da sociedade, contemplando 
     seus aspectos técnicos, econômicos, socioambientais, de acessibilidade e de prevenção de desastres, entre outros. Os problemas serão colhidos 
     junto à sociedade via ação de extensão vinculada, sendo de natureza inerentemente aberta, prática e integradora, oportunizando que os 
     estudantes trabalhem simultaneamente os conteúdos aprendidos múltiplas disciplinas diferentes e em contextos realistas. Desenvolvimento de 
     habilidades de trabalho autônomo, comunicação, convívio social e respeito à diversidade através da atuação em grupos e do contato com questões
      e/ou indivíduos externos à Universidade.  Dada a natureza integradora desta atividade, espera-se que os alunos de Projeto Integrado III, 
      interajam com os alunos de Projeto Integrado I e II no processo, permitindo a troca de experiência de alunos em diversos momentos na parte 
      profissionalizante do curso.}

      \bibdef{Cetinkunt2015}
      \bibdef{Horowitz2015}
      \bibdef{Russell2015}
      \bibdef[basic]{Silva2006c}
      \bibdef[basic]{Junival2008b}
      \bibdef[basic]{Bishop2002c}
      \bibdef[compl]{Silva2008c}
      \bibdef[compl]{Onwubolu2005c}
      \bibdef[compl]{Bishop2008c}
%%%%%%
%%%%%%
%
\classdef[Pro.Automacao]{INF1017}{4}{APRENDIZADO DE MÁQUINA}

     \csummary{Fundamentos da área de aprendizado de máquina e algoritmos baseados em redes neurais e em abordagens estatísticas. Aplicações para a resolução de problemas de aprendizado supervisionado, não-supervisionado, e por reforço.}

      \bibdef{Faceli2011}
      \bibdef{Russell2010}
      \bibdef{Sutton1999}
      \bibdef[basic]{Haykin2001b}
      \bibdef[compl]{Geron2022}
      \bibdef[compl]{Haykin2009}
%%%%%%
%%%%%%
%
\classdef[Pro.Automacao]{INF01037}{4}{COMPUTAÇÃO EVOLUTIVA}

     \csummary{Conceitos básicos sobre Vida Artificial. Introdução ao Paradigma de Algoritmos Genéticos. Aplicações de Algoritmos Genéticos. Programação evolutiva.}

      \bibdef{Eiben2003}
      \bibdef{Michalewicz1999}
      \bibdef{Peitgen1992}
      \bibdef[basic]{Barone2003}
      \bibdef[basic]{Fogel2000}
      \bibdef[basic]{Jones2008}
      \bibdef[basic]{Mitchell1996}
%%%%%%
%%%%%%
%
\classdef[Transv.outros]{ADM01135}{2}{ENGENHARIA ECONÔMICA E AVALIAÇÕES}

     \csummary{Introdução à engenharia econômica. Engenharia de avaliações. Projetos econômicos.}

      \bibdef{Casarotto2007}
      \bibdef[basic]{Blank2008}
      \bibdef[basic]{DalZot2008}
      \bibdef[basic]{Gitman2004}
      \bibdef[basic]{Hirschfeld1998}
      \bibdef[basic]{Ross2002}
%%%%%%
%%%%%%
%
\classdef[Pro.ContProc]{ENG07043}{2}{INSTRUMENTAÇÃO DE PROCESSOS INDUSTRIAIS}

     \csummary{Fluxograma de engenharia, normas para descrever estratégias de controle de processos industriais. Principais estratégias de controle 
     utilizadas para controlar colunas de destilação, reatores químicos, trocadores de calor, fornos, biorreatores e demais processos usados nas 
     indústrias de processos. Utilização industrial de malhas de controle feedback, cascata e feedforward. Dimensionamento de válvulas de controle e 
     atuadores. Apresentação dos principais instrumentos de medição utilizados no cenário industrial. Medidores de temperatura, pressão, vazão, 
     nível e composição/analisadores. Descrição e quantificação dos erros de medição. Desenvolvimento de inferidores para acompanhar variáveis de 
     difícil medição.}

      \bibdef{Campos2006}
      \bibdef[basic]{Liptak1995a}
      \bibdef[basic]{Liptak1995b}
%%%%%%
%%%%%%
%
\classdef[Pro.ContProc]{ENG07012}{3}{LABORATÓRIO DE CONTROLE E OPERAÇÃO DE PROCESSOS}

     \csummary{Aulas práticas de laboratório, contemplando experimentos, coleta de dados e interpretação de resultados, em assuntos abordados nas disciplinas instrumentação da indústria química, cálculo de reatores e controle de processos.}

      \bibdef{Ogunnaike1994}
      \bibdef{Campos2006b}
      \bibdef{Seborg2004}
      \bibdef[basic]{Luyben1990}
      \bibdef[basic]{Levine1996}
      \bibdef[compl]{Foust1980}
      \bibdef[compl]{Macintyre1997}
      \bibdef[compl]{Mccabe2005}
      \bibdef[compl]{Perry2007}
      \bibdef[compl]{Welty2007}
%%%%%%
%%%%%%
%
\classdef[Transv.outros]{EDU03071}{2}{LÍNGUA BRASILEIRA DE SINAIS (LIBRAS)}

     \csummary{Aspectos linguísticos da Língua Brasileira de Sinais (LIBRAS). História das comunidades surdas, da cultura e das identidades surdas. Ensino básico da LIBRAS. Políticas linguísticas e educacionais para surdos.}

      \bibdef{Campello2014}
      \bibdef{Gesser2009}
      \bibdef{Quadros2004}
      \bibdef[basic]{Brasil2005}
      \bibdef[basic]{Brasil2010}
      \bibdef[basic]{Brasil2002}
      \bibdef[basic]{Brasil2015}
      \bibdef[basic]{Karnopp2014}
      \bibdef[basic]{Pontin2014a}
      \bibdef[basic]{Pontin2014b}
      \bibdef[compl]{Gomes2015}
      \bibdef[compl]{Karnopp2010}
      \bibdef[compl]{Thoma2010}
      \bibdef[compl]{Thoma2004}
%%%%%%
%%%%%%
%
\classdef[Pro.Control]{ENG07062}{3}{OTIMIZAÇÃO APLICADA}

     \csummary{Os métodos de programação matemática (métodos de otimização) são apresentados aplicados à solução de diferentes classes de problemas, 
     tais como: síntese de processos, programação de produção e logística, estimação de parâmetros, otimização em tempo real, controle preditivo, 
     entre outras aplicações encontradas comumente na engenharia. O curso inicia com a revisão de conceitos básicos de otimização, tais como: 
     critérios de optimalidade, convexidade, linearidade, continuidade, etc. A seguir as diversas técnicas empregadas para resolver as diferentes 
     formulações de problemas de otimização são apresentadas, segundo a seguinte classificação comumente adotada: a) programação não linear (NLP) 
     com e sem restrições; b) programação linear (LP); c) programação quadrática (QP); d) Programação inteira mista linear (MILP); e) programação 
     inteira mista não linear (MINLP); f) programação dinâmica e g) otimização global. Cada uma dessas técnicas é apresentada tendo como ponto de partida uma aplicação real.}

      \bibdef{Chong2001}
      \bibdef{Bartholomew2008}
      \bibdef{Nocedal2006}
      \bibdef[basic]{Antoniou2007}
      \bibdef[basic]{Edgar2001}
      \bibdef[basic]{Hendrix2010}
      \bibdef[basic]{Biegler2010}
      \bibdef[compl]{Neumann2010}
      \bibdef[compl]{Schaffler2012}

